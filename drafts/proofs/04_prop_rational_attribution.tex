% Proof of Proposition: Rational Attribution Equilibrium and Bayesian Game Equivalence
% For inclusion in ABE manuscript appendix
% Condensed version: 2026-01-14 (details moved to Online Appendix OA.2.3)

\paragraph{The Equivalent Bayesian Game $\Gamma^B$.}
Given the psychological game $\Gamma$ and a candidate equilibrium $\sigma^*$ satisfying Rational Attribution Equilibrium (RAE), construct:
\begin{itemize}
    \item Players: $N = N_H \cup N_A$.
    \item Type spaces: $\mathcal{T}_i = T_i \times \Omega_i$ for humans; $\mathcal{T}_j = \Theta_j$ for AI.
    \item Human payoff: $u_i^B(s, \tau) = \pi_i(s) + \psi_i(s, \sigma_i^*, \sigma_i^*)$ (beliefs equal equilibrium strategy under RAE).
    \item AI payoff: $u_j^B(s, \theta_j) = U_j^A(s; \theta_j)$.
\end{itemize}

\begin{proof}[Proof of Proposition~\ref{prop:rational-attribution}]

\noindent\textbf{Part (a): ABE with RAE $\Rightarrow$ BNE.}

Let $(\sigma^*, h^*, \tilde{h}^*)$ be an ABE where $(\sigma^*, \phi)$ satisfy RAE. Under RAE, all second-order beliefs---both genuine and attributed---equal $\sigma_i^*$. (See Online Appendix OA.2.3 for detailed belief characterization.)

Human $i$'s ABE utility coincides with Bayesian game payoff:
\begin{equation}
    U_i^H(\sigma_i, \sigma_{-i}^*, h_i^*, \tilde{h}_i^*) = \pi_i(\sigma_i, \sigma_{-i}^*) + \psi_i(\sigma_i, \sigma_{-i}^*, \sigma_i^*, \sigma_i^*) = u_i^B(\sigma_i, \sigma_{-i}^*).
\end{equation}
Since $\sigma_i^*$ maximizes $U_i^H$, it also maximizes $u_i^B$. AI payoffs are identical in both frameworks. Hence $\sigma^*$ is a BNE of $\Gamma^B$.

\medskip
\noindent\textbf{Part (b): BNE $\Rightarrow$ ABE with RAE.}

Let $\sigma^*$ be a BNE of $\Gamma^B$. Construct beliefs by $h_i^{*(1,k)} = \sigma_k^*$, $h_i^{*(2,k)} = \sigma_i^*$, and $\tilde{h}_i^{*(2,j)} = \phi_i(\theta_j, x_j, \omega_i) = \sigma_i^*$ (RAE condition). All ABE conditions are satisfied by construction.

\medskip
\noindent\textbf{Conclusion.}
The mappings establish a correspondence: the same strategy profile appears in both ABE and BNE when RAE holds.
\end{proof}
