% Proof of Proposition: Rational Attribution Equilibrium and Bayesian Game Equivalence
% For inclusion in ABE manuscript appendix
% When attribution is rational, ABE reduces to Bayes-Nash Equilibrium

\paragraph{The Equivalent Bayesian Game $\Gamma^B$.}
Given the psychological game $\Gamma$ and a candidate equilibrium $\sigma^*$ satisfying Rational Attribution Equilibrium (RAE), construct the Bayesian game $\Gamma^B$ as follows:

\textbf{1. Players.} $N = N_H \cup N_A$ (identical to $\Gamma$).

\textbf{2. Type spaces.}
\begin{itemize}
    \item For human $i \in N_H$: $\mathcal{T}_i = T_i \times \Omega_i$, where $T_i$ contains psychological parameters $(\beta_i, \gamma_i, \ldots)$ and $\Omega_i = [0,1]$ is the anthropomorphism space.
    \item For AI $j \in N_A$: $\mathcal{T}_j = \Theta_j$ (design parameters).
\end{itemize}

\textbf{3. Strategy spaces.} $S_k$ for all $k \in N$ (identical to $\Gamma$).

\textbf{4. Prior.} $\mu \in \Delta(\mathcal{T})$ derived from $p$ in $\Gamma$.

\textbf{5. Payoff functions.}
\begin{itemize}
    \item For human $i \in N_H$ with type $\tau_i = (t_i, \omega_i)$:
    \begin{equation}
    \label{eq:bayesian-human-payoff}
        u_i^B(s, \tau) = \pi_i(s) + \psi_i(s, \sigma_i^*, \sigma_i^*)
    \end{equation}
    Under Rational Attribution Equilibrium (RAE), all second-order beliefs (genuine and attributed) equal $\sigma_i^*$. The psychological payoff $\psi_i$ is evaluated at these consistent beliefs.

    \item For AI $j \in N_A$ with type $\tau_j = \theta_j$:
    \begin{equation}
        u_j^B(s, \theta_j) = U_j^A(s; \theta_j)
    \end{equation}
\end{itemize}

\begin{remark}[Equilibrium-Dependence]
The Bayesian game construction uses the candidate equilibrium $\sigma^*$ in the payoff function \eqref{eq:bayesian-human-payoff}. This is not circular because we construct $\Gamma^B$ for a \emph{given} candidate $\sigma^*$ and then verify that $\sigma^*$ is indeed a BNE of the resulting game. The equivalence shows that existence of such equilibria coincides across frameworks.
\end{remark}

\begin{proof}[Proof of Proposition~\ref{prop:rational-attribution}]
We prove parts (a) and (b) separately.

\medskip
\noindent\textbf{Part (a): ABE with Rational Attribution Equilibrium $\Rightarrow$ BNE.}

Let $(\sigma^*, h^*, \tilde{h}^*)$ be an ABE of $\Gamma$ where $(\sigma^*, \phi)$ satisfy Rational Attribution Equilibrium (RAE). We show that $\sigma^*$ is a BNE of $\Gamma^B$.

\emph{Step 1: Characterize equilibrium beliefs under RAE.}

By the ABE belief consistency conditions (ABE3):
\begin{align}
    h_i^{*(1,k)} &= \sigma_k^* \quad \text{(first-order beliefs match equilibrium play)} \\
    h_i^{*(2,k)} &= h_k^{*(1,i)} = \sigma_i^* \quad \text{(second-order beliefs: $k$ expects $i$ to play $\sigma_i^*$)}
\end{align}
The second equality uses ABE3 twice: first, second-order consistency $h_i^{*(2,k)} = h_k^{*(1,i)}$; second, first-order consistency $h_k^{*(1,i)} = \sigma_i^*$.

By attribution consistency (ABE4) and RAE:
\begin{equation}
    \tilde{h}_i^{*(2,j)} = \phi_i(\theta_j, x_j, \omega_i) = \sigma_i^*
\end{equation}
The last equality is the RAE condition.

Therefore, under RAE, all second-order beliefs---both genuine ($h_i^{*(2,k)}$) and attributed ($\tilde{h}_i^{*(2,j)}$)---equal the equilibrium strategy $\sigma_i^*$.

\emph{Step 2: Verify human optimality in $\Gamma^B$.}

In the ABE, human $i$'s utility at equilibrium is:
\begin{align}
    U_i^H(\sigma_i, \sigma_{-i}^*, h_i^*, \tilde{h}_i^*)
    &= \pi_i(\sigma_i, \sigma_{-i}^*) + \psi_i(\sigma_i, \sigma_{-i}^*, h_i^{*(2)}, \tilde{h}_i^{*(2)}) \\
    &= \pi_i(\sigma_i, \sigma_{-i}^*) + \psi_i(\sigma_i, \sigma_{-i}^*, \sigma_i^*, \sigma_i^*)
\end{align}
where the second line uses Step 1: all second-order beliefs equal $\sigma_i^*$.

In the Bayesian game $\Gamma^B$, human $i$'s payoff is:
\[
    u_i^B(s, \tau) = \pi_i(s) + \psi_i(s, \sigma_i^*, \sigma_i^*)
\]
which coincides with the ABE utility under RAE.

Since $\sigma_i^*$ maximizes $U_i^H$ in the ABE, it also maximizes $u_i^B$ in $\Gamma^B$. Human optimality in $\Gamma^B$ is satisfied.

\emph{Step 3: Verify AI optimality in $\Gamma^B$.}

In the ABE, for each AI $j \in N_A$:
\begin{equation}
    \sigma_j^* \in \arg\max_{\sigma_j \in \Delta(S_j)} U_j^A(\sigma_j, \sigma_{-j}^*; \theta_j)
\end{equation}

In $\Gamma^B$, AI $j$'s payoff is $u_j^B(s, \theta_j) = U_j^A(s; \theta_j)$, which is identical. Hence AI optimality is satisfied.

\emph{Step 4: Conclude.}

Since both human and AI optimality in $\Gamma^B$ are satisfied by $\sigma^*$, we have that $\sigma^*$ is a BNE of $\Gamma^B$.

\medskip
\noindent\textbf{Part (b): BNE $\Rightarrow$ ABE with Rational Attribution Equilibrium.}

Let $\sigma^*$ be a BNE of the Bayesian game $\Gamma^B$ constructed above, where the payoff function uses $\sigma^*$ itself (i.e., $\sigma^*$ is a fixed point). We construct an ABE of $\Gamma$.

\emph{Step 1: Construct the belief system.}

Define genuine beliefs by:
\begin{align}
    h_i^{*(1,k)} &:= \sigma_k^* \quad \text{for } k \in N_H \cup N_A \\
    h_i^{*(2,k)} &:= \sigma_i^* \quad \text{for } k \in N_H
\end{align}

Define attributed beliefs by:
\begin{equation}
    \tilde{h}_i^{*(2,j)} := \phi_i(\theta_j, x_j, \omega_i) = \sigma_i^*
\end{equation}
where the last equality is the RAE condition.

\emph{Step 2: Verify ABE conditions.}

\textbf{(ABE1) Human Optimality.} Since $\sigma^*$ is a BNE of $\Gamma^B$, and the payoffs coincide under RAE (Step 2 of Part (a)), human optimality in ABE is satisfied.

\textbf{(ABE2) AI Optimality.} Since $\sigma^*$ is a BNE and AI payoffs are identical in $\Gamma$ and $\Gamma^B$, AI optimality is satisfied.

\textbf{(ABE3) Genuine Belief Consistency.} By construction:
\begin{align}
    h_i^{*(1,k)} &= \sigma_k^* \quad \text{(first-order consistency)} \\
    h_i^{*(2,k)} &= \sigma_i^* = h_k^{*(1,i)} \quad \text{(second-order consistency)}
\end{align}

\textbf{(ABE4) Attribution Consistency.} By construction:
\begin{equation}
    \tilde{h}_i^{*(2,j)} = \phi_i(\theta_j, x_j, \omega_i)
\end{equation}

All four ABE conditions are satisfied. Therefore, $(\sigma^*, h^*, \tilde{h}^*)$ is an ABE of $\Gamma$.

\medskip
\noindent\textbf{Conclusion.}

The mappings in Parts (a) and (b) are consistent: the same strategy profile $\sigma^*$ appears in both the ABE and the BNE when RAE holds. This establishes the correspondence.
\end{proof}
