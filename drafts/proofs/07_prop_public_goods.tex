% Proof of Proposition: Public Goods ABE
% Appendix material for main manuscript
% Condensed version: 2026-01-14 (remarks moved to Online Appendix OA.5.2)

\begin{proof}[Proof of Proposition~\ref{prop:public-goods}]
The proposition requires (A2') attribution monotonicity, (I) indignation dominance, and $\beta_i > 0$ for all humans.

The human utility function is $U_i^H = \pi_i + \psi_i^{IND}$, where:
\begin{align}
    \pi_i(c) &= E - c_i + \frac{m}{n} \sum_{k \in N} c_k, \\
    \psi_i^{IND} &= -\beta_i \cdot \mathbf{1}_{c_i < c^*} \cdot \left[ \sum_{k \in N_H \setminus \{i\}} h_i^{(2,k)}(c^*) + \lambda_i^{IND} \sum_{j \in N_A} \tilde{h}_i^{(2,j)}(c^*) \right].
\end{align}

%------------------------------------------------------------------------------
% PART (i): COOPERATION EQUILIBRIUM
%------------------------------------------------------------------------------

\medskip
\noindent\textbf{Proof of (i): Cooperation equilibrium.}

Under conditions (A2') and (I), if AI contribute $c_A = E$ and humans attribute sufficiently high expectations ($\omega_i \geq \bar{\omega}_i$), the symmetric profile $c^* = (E, \ldots, E)$ constitutes an ABE.

\emph{Step 1.} Consider $c_k^* = E$ for all $k \in N$. Genuine belief consistency requires $h_i^{(2,k)}(E) = 1$.

\emph{Step 2.} By dispositional attribution: $\tilde{h}_i^{(2,j)}(E) = \omega_i \cdot \bar{h}_H$ for $j \in N_A$.

\emph{Step 3.} If human $i$ cooperates: $U_i^H(E) = mE$ (no psychological cost).

\emph{Step 4.} If human $i$ deviates to $c_i = 0$:
\begin{equation}
    U_i^H(0) = E\left(1 + \frac{m(n-1)}{n}\right) - \beta_i \left[ (n_H - 1) + \lambda_i^{IND} n_A \omega_i \bar{h}_H \right].
\end{equation}

\emph{Step 5.} Cooperation is optimal iff $U_i^H(E) \geq U_i^H(0)$:
\begin{equation}\label{eq:coop-ic}
    \beta_i \left[ (n_H - 1) + \lambda_i^{IND} n_A \omega_i \bar{h}_H \right] \geq E\left(1 - \frac{m}{n}\right).
\end{equation}

\emph{Step 6.} Solving for the threshold:
\begin{equation}\label{eq:omega-threshold}
    \bar{\omega}_i = \frac{E(1 - m/n) - \beta_i(n_H - 1)}{\beta_i \lambda_i^{IND} n_A \bar{h}_H}.
\end{equation}
Condition (I) ensures $\bar{\omega}_i \leq 1$. When $\omega_i \geq \bar{\omega}_i$, cooperation is optimal. \hfill $\square$

%------------------------------------------------------------------------------
% PART (ii): DEFECTION EQUILIBRIUM
%------------------------------------------------------------------------------

\medskip
\noindent\textbf{Proof of (ii): Defection equilibrium.}

Under low anthropomorphism with defecting AI, symmetric defection is the unique symmetric ABE.

\emph{Step 1.} Consider $c_k^* = 0$ for all $k \in N$. In symmetric defection: $h_i^{(2,k)}(E) = 0$ and $\tilde{h}_i^{(2,j)}(E) = \omega_i \cdot \bar{h}_L$.

\emph{Step 2.} The indignation cost of defection: $|\psi_i^{IND}| = \beta_i \lambda_i^{IND} n_A \omega_i \bar{h}_L \to 0$ as $\omega_i \to 0$.

\emph{Step 3.} Defection is preferred iff:
\begin{equation}
    E\left(1 - \frac{m}{n}\right) > \beta_i \lambda_i^{IND} n_A \omega_i \bar{h}_L.
\end{equation}
This holds for sufficiently low $\omega_i$. \hfill $\square$

%------------------------------------------------------------------------------
% PART (iii): POPULATION SHARE EFFECTS
%------------------------------------------------------------------------------

\medskip
\noindent\textbf{Proof of (iii): Population share effects.}

Higher $n_A$ affects equilibrium through two channels: (1) material---diluting MPCR, increasing free-rider temptation; (2) psychological---more sources of attributed expectations.

The material channel has diminishing marginal effect:
\begin{equation}
    \frac{\partial (\Delta\pi_i)}{\partial n_A} = \frac{mE}{(n_H + n_A)^2} > 0.
\end{equation}

The psychological channel has constant marginal effect:
\begin{equation}
    \frac{\partial \Psi_i}{\partial n_A} = \beta_i \lambda_i^{IND} \omega_i \bar{h}(c_A) \geq 0.
\end{equation}

For sufficiently large $n_A$, the psychological channel dominates. See Online Appendix OA.5.2 for detailed derivations. \hfill $\square$
\end{proof}

\begin{remark}
Extended analysis including connection to standard PGT, role of attenuation, boundary cases, empirical implications, and threshold interpretation appears in Online Appendix OA.5.2.
\end{remark}
