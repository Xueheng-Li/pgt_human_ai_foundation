% Proof of Proposition: Reduction to Nash Equilibrium
% For inclusion in ABE manuscript appendix
% Condensed version: 2026-01-14 (details moved to Online Appendix OA.2.2)

\noindent\textit{Incomplete information}: When the game involves incomplete information about types, the material game $\Gamma^M$ becomes a Bayesian game where types affect only AI utilities $U_j^A(s; \theta_j)$. The result holds ex-post for each type realization.

\begin{proof}
The proof establishes a bijection between ABE of $\Gamma$ (when $\psi_i \equiv 0$) and Nash equilibria of $\Gamma^M$.

\medskip
\noindent\textbf{Part 1: ABE $\Rightarrow$ Nash.} When $\psi_i \equiv 0$, human utility reduces to $U_i^H = \pi_i(s)$. The ABE optimality conditions become Nash best-response conditions. Hence $s^*$ is a Nash equilibrium of $\Gamma^M$. (Details in Online Appendix OA.2.2.)

\medskip
\noindent\textbf{Part 2: Nash $\Rightarrow$ ABE.} Given Nash equilibrium $s^*$, construct beliefs by $h_i^{*(1,k)} = s_k^*$, $h_i^{*(2,k)} = s_i^*$, and $\tilde{h}_i^{*(2,j)} = \phi_i(\theta_j, x_j, \omega_i)$. All ABE conditions are satisfied. (Details in Online Appendix OA.2.2.)

\medskip
\noindent\textbf{Part 3:} The belief systems are uniquely determined by the consistency conditions.
\end{proof}
