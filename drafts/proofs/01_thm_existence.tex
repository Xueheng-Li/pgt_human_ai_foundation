% Proof of Theorem: Existence of ABE
% Appendix material for main manuscript
% Notation follows sec_framework.tex and sec_equilibrium.tex
% Revised: 2026-01-14 to address reviewer feedback (CR2)
% Further condensed: 2026-01-14 (topology details moved to Online Appendix OA.1)

%==============================================================================
% LEMMA A.1: BOUNDEDNESS OF PSYCHOLOGICAL PAYOFFS
%==============================================================================

\begin{lemma}[Boundedness of Psychological Payoffs]
\label{lem:bounded-psi}
Under Assumptions A1 (Regularity) and A2 (Attribution Continuity), for each human type $t_i \in T_i$, there exists $M(t_i) < \infty$ such that
\begin{equation}
    |\psi_i(s, h_i^{(2)}, \tilde{h}_i^{(2)}; t_i)| \leq M(t_i)
\end{equation}
for all strategy profiles $s \in S$, all genuine second-order beliefs $h_i^{(2)} \in \mathcal{H}_i^{(2)}$, and all attributed beliefs $\tilde{h}_i^{(2)} \in \tilde{\mathcal{H}}_i^{(2)}$. Moreover, by compactness of the type space, there exists a uniform bound $M = \sup_{t_i \in T_i} M(t_i) < \infty$.
\end{lemma}

\noindent\emph{Proof.} See Online Appendix OA.1.1 for the complete proof, which establishes boundedness via Weierstrass's theorem applied to the continuous function $\psi_i$ on the compact domain $S \times \mathcal{H}_i^{(2)} \times \tilde{\mathcal{H}}_i^{(2)} \times T_i$. \hfill $\square$

\medskip
\noindent With boundedness established, we proceed to the main existence result.

%==============================================================================
% MAIN THEOREM: EXISTENCE OF ABE
%==============================================================================

\begin{proof}[Proof of Theorem~\ref{thm:existence}]
The proof applies Kakutani's fixed-point theorem to the best-response correspondence on the mixed strategy space. The key insight: both genuine and attributed beliefs are \emph{functions} of strategies, not independent equilibrium variables, so the fixed-point argument operates on the finite-dimensional space $\Sigma$ alone.

\medskip
\noindent\textbf{Step 1: Strategy space construction.}
The space $\Sigma = \prod_{i \in N} \Delta(S_i)$ is compact and convex. See Online Appendix OA.1.2 for topology details.

\medskip
\noindent\textbf{Step 2: Belief computation.}
Given $\sigma \in \Sigma$, genuine beliefs are $h_i^{(1,k)}(\sigma) = \sigma_k$ and $h_i^{(2,k)}(\sigma) = \sigma_i$ (coordinate projections, continuous). Under exogenous attribution (signal-based or dispositional), $\tilde{h}_i^{(2,j)} = \phi_i(\theta_j, x_j, \omega_i)$ is constant in $\sigma$---the key simplification: attributed beliefs do not create additional equilibrium conditions. See Online Appendix OA.1.2 for the three attribution approaches.

\medskip
\noindent\textbf{Step 3: Best-response correspondence.}
For each human $i \in N_H$, define $BR_i^H: \Sigma \rightrightarrows \Delta(S_i)$ by
\[
    BR_i^H(\sigma) = \arg\max_{\sigma_i' \in \Delta(S_i)} \mathbb{E}_{\sigma_i', \sigma_{-i}}\left[ U_i^H(s, h_i(\sigma), \tilde{h}_i) \right]
\]
where beliefs $h_i(\sigma), \tilde{h}_i$ are computed as in Step 2.

For each AI $j \in N_A$, define $BR_j^A: \Sigma \rightrightarrows \Delta(S_j)$ by
\[
    BR_j^A(\sigma) = \arg\max_{\sigma_j' \in \Delta(S_j)} \mathbb{E}_{\sigma_j', \sigma_{-j}}\left[ U_j^A(s; \theta_j) \right].
\]
By A1c (continuous payoffs), $U_j^A(s; \theta_j)$ is continuous in $\sigma$, ensuring well-defined best responses.

We verify that each $BR_k$ (for $k \in N$) satisfies the conditions for Kakutani's theorem:
\begin{enumerate}
    \item[(i)] \emph{Non-empty valued}: The simplex $\Delta(S_k)$ is compact and the objective is continuous in $\sigma_k'$, so the maximum is attained by the Weierstrass extreme value theorem.

    \item[(ii)] \emph{Convex valued}: Expected utility is multilinear in the strategy profile, hence linear in $\sigma_k'$ for fixed $\sigma_{-k}$. Any convex combination of maximizers is therefore a maximizer.

    \item[(iii)] \emph{Upper hemicontinuous}: The constraint set $\Delta(S_k)$ is constant (hence continuous). The objective is continuous in $(\sigma_k', \sigma)$: the material component $\pi_i(s)$ is continuous by A1c; the psychological component $\psi_i(s, h_i^{(2)}(\sigma), \tilde{h}_i^{(2)})$ is continuous by A1c (continuity of $\psi_i$) and continuity of the belief mappings (coordinate projections for genuine beliefs, constants for attributed beliefs). By Berge's Maximum Theorem, $BR_k$ is upper hemicontinuous.
\end{enumerate}

\medskip
\noindent\textbf{Step 4: Fixed-point application.}
Define the joint correspondence $BR: \Sigma \rightrightarrows \Sigma$ by $BR(\sigma) = \prod_{k \in N} BR_k(\sigma)$. The product of non-empty, convex-valued correspondences inherits these properties.

We verify that $BR$ has a closed graph. Let $(\sigma^n, \hat{\sigma}^n) \to (\sigma, \hat{\sigma})$ with $\hat{\sigma}^n \in BR(\sigma^n)$ for all $n$. For each player $k$, we have $\hat{\sigma}_k^n \in BR_k(\sigma^n)$, meaning $\hat{\sigma}_k^n$ maximizes the objective over $\Delta(S_k)$ given $\sigma^n$. Since $\Delta(S_k)$ is compact and the objective is continuous in $(\sigma_k', \sigma)$, the sequence $\hat{\sigma}_k^n \to \hat{\sigma}_k$ and $\hat{\sigma}_k$ maximizes the objective given $\sigma$ (by Berge's theorem applied to each component). Thus $\hat{\sigma} \in BR(\sigma)$, and $BR$ has a closed graph.

By Kakutani's fixed-point theorem: $\Sigma$ is non-empty, compact, and convex; $BR(\sigma)$ is non-empty and convex for all $\sigma$; $BR$ has a closed graph. Therefore, there exists $\sigma^* \in \Sigma$ with $\sigma^* \in BR(\sigma^*)$.

\medskip
\noindent\textbf{Step 5: Equilibrium verification.}
We verify that the fixed point $\sigma^*$, together with beliefs computed from Step 2, satisfies all four ABE conditions.

\emph{(ABE1) Human Optimality}: For $i \in N_H$, the fixed-point property gives $\sigma_i^* \in BR_i^H(\sigma^*)$, so $\sigma_i^*$ maximizes $U_i^H$ given $\sigma_{-i}^*$ and the induced beliefs.

\emph{(ABE2) AI Optimality}: For $j \in N_A$, we have $\sigma_j^* \in BR_j^A(\sigma^*)$, so $\sigma_j^*$ maximizes $U_j^A$ given $\sigma_{-j}^*$.

\emph{(ABE3) Genuine Belief Consistency}: Define equilibrium beliefs by:
\begin{align*}
    h_i^{*(1,k)} &:= h_i^{(1,k)}(\sigma^*) = \sigma_k^* \\
    h_i^{*(2,k)} &:= h_i^{(2,k)}(\sigma^*) = \sigma_i^*
\end{align*}
First-order consistency holds: $h_i^{*(1,k)} = \sigma_k^*$ (beliefs equal actual strategies). For second-order consistency, we need $h_i^{*(2,k)} = h_k^{*(1,i)}$. By construction:
\[
    h_i^{*(2,k)} = \sigma_i^* = h_k^{(1,i)}(\sigma^*) = h_k^{*(1,i)}
\]
Thus $i$'s belief about what $k$ expected from $i$ equals what $k$ actually expected from $i$.

\emph{(ABE4) Attribution Consistency}: Define $\tilde{h}_i^{*(2,j)} := \phi_i(\theta_j, x_j, \omega_i)$. This is exactly the attribution consistency condition: attributed beliefs are determined by the attribution function.

All four ABE conditions are satisfied. Therefore, $(\sigma^*, h^*, \tilde{h}^*)$ is an Attributed Belief Equilibrium.
\end{proof}

\begin{remark}
Technical details including the boundedness lemma proof, topology considerations, attribution approaches, and extensions to behavioral attribution appear in Online Appendix OA.1.
\end{remark}
