% Proof of Theorem: Existence of ABE
% Appendix material for main manuscript
% Notation follows sec_framework.tex and sec_equilibrium.tex
% Revised: 2026-01-14 to address reviewer feedback (CR2)

%==============================================================================
% LEMMA A.1: BOUNDEDNESS OF PSYCHOLOGICAL PAYOFFS
%==============================================================================

\begin{lemma}[Boundedness of Psychological Payoffs]
\label{lem:bounded-psi}
Under Assumptions A1 (Regularity) and A2 (Attribution Continuity), for each human type $t_i \in T_i$, there exists $M(t_i) < \infty$ such that
\begin{equation}
    |\psi_i(s, h_i^{(2)}, \tilde{h}_i^{(2)}; t_i)| \leq M(t_i)
\end{equation}
for all strategy profiles $s \in S$, all genuine second-order beliefs $h_i^{(2)} \in \mathcal{H}_i^{(2)}$, and all attributed beliefs $\tilde{h}_i^{(2)} \in \tilde{\mathcal{H}}_i^{(2)}$. Moreover, by compactness of the type space, there exists a uniform bound $M = \sup_{t_i \in T_i} M(t_i) < \infty$.
\end{lemma}

\begin{proof}
We establish boundedness by showing that $\psi_i$ is a continuous function on a compact domain.

\medskip
\noindent\textit{Step 1: Compactness of the strategy domain.}
By A1a (finite strategy spaces), each strategy space $S_i$ is finite. Hence the strategy profile space $S = \prod_{i \in N} S_i$ is finite and therefore compact in the discrete topology.

\medskip
\noindent\textit{Step 2: Compactness of the belief domains.}
For genuine second-order beliefs, $h_i^{(2,k)} \in \Delta(\Delta(S_{-k}))$ for each $k \neq i$. Since $S_{-k}$ is finite by A1a, the simplex $\Delta(S_{-k})$ is a compact subset of $\mathbb{R}^{|S_{-k}|}$. By Prohorov's theorem, the space of probability distributions over this compact simplex, $\Delta(\Delta(S_{-k}))$, is compact in the weak* topology. Since we work in finite dimensions, weak* convergence coincides with Euclidean convergence. Thus $\mathcal{H}_i^{(2)} = \prod_{k \neq i} \Delta(\Delta(S_{-k}))$ is compact as a finite product of compact spaces.

For attributed beliefs, $\tilde{h}_i^{(2,j)} \in \Delta(S_i)$ for each AI agent $j \in N_A$. Since $S_i$ is finite, $\Delta(S_i)$ is a compact simplex. Thus $\tilde{\mathcal{H}}_i^{(2)} = \prod_{j \in N_A} \Delta(S_i)$ is compact.

\medskip
\noindent\textit{Step 3: Boundedness for fixed type.}
For fixed type $t_i = (\beta_i, \gamma_i, \omega_i, \lambda_i^{IND}, \lambda_i^{GUILT}, \ldots) \in T_i$, the psychological payoff function $\psi_i(\cdot; t_i): S \times \mathcal{H}_i^{(2)} \times \tilde{\mathcal{H}}_i^{(2)} \to \mathbb{R}$ is continuous. For the indignation component:
\[
    \psi_i^{IND}(s, h_i^{(2)}, \tilde{h}_i^{(2)}; t_i) = -\beta_i \cdot \mathbf{1}_{s_i = D} \cdot \left[ \sum_{k \in N_H} h_i^{(2,k)}(C) + \lambda_i^{IND} \sum_{j \in N_A} \tilde{h}_i^{(2,j)}(C) \right]
\]
This is linear in the belief arguments, hence continuous. Since $S$ is finite, continuity in $s$ is automatic. By A1c (continuous payoffs), $\psi_i$ is continuous in all arguments. The domain $S \times \mathcal{H}_i^{(2)} \times \tilde{\mathcal{H}}_i^{(2)}$ is compact. By the Weierstrass extreme value theorem, $|\psi_i(\cdot; t_i)| \leq M(t_i)$ for some $M(t_i) < \infty$.

\medskip
\noindent\textit{Step 4: Uniform boundedness over types.}
By A1b (compact type spaces), $T_i$ is compact. Since $\psi_i$ is jointly continuous in $(s, h_i^{(2)}, \tilde{h}_i^{(2)}, t_i)$ by A1c, and the product space $(S \times \mathcal{H}_i^{(2)} \times \tilde{\mathcal{H}}_i^{(2)}) \times T_i$ is compact, another application of Weierstrass gives uniform boundedness: $|\psi_i| \leq M_i$ for some $M_i < \infty$. Taking $M = \max_{i \in N_H} M_i$ yields the global bound.
\end{proof}

\begin{remark}
Lemma~\ref{lem:bounded-psi} shows that Assumption A3 (Bounded Psychological Payoffs) follows from A1 and A2. The boundedness arises from three sources: finiteness of strategy spaces (A1a), compactness of type spaces (A1b), and beliefs lying in compact simplices.
\end{remark}

\medskip
\noindent With boundedness established, we proceed to the main existence result.

%==============================================================================
% MAIN THEOREM: EXISTENCE OF ABE
%==============================================================================

\begin{proof}[Proof of Theorem~\ref{thm:existence}]
The proof applies Kakutani's fixed-point theorem to the best-response correspondence on the mixed strategy space. The key insight: both genuine and attributed beliefs are \emph{functions} of strategies, not independent equilibrium variables, so the fixed-point argument operates on the finite-dimensional space $\Sigma$ alone.

\medskip
\noindent\textbf{Step 1: Strategy space construction.}
We work on the product space $\Sigma = \prod_{i \in N} \Delta(S_i)$, where each $\Delta(S_i)$ denotes the simplex of probability distributions over player $i$'s finite action set $S_i$. Since each $S_i$ is finite with $|S_i| = n_i$ by A1a, the simplex $\Delta(S_i)$ is the standard $(n_i - 1)$-dimensional simplex:
\[
\Delta(S_i) = \left\{ \sigma_i \in \mathbb{R}^{n_i} : \sigma_i(s_i) \geq 0 \ \forall s_i \in S_i, \; \sum_{s_i \in S_i} \sigma_i(s_i) = 1 \right\}.
\]
We equip each $\Delta(S_i)$ with the Euclidean (relative) topology inherited from $\mathbb{R}^{n_i}$, and $\Sigma$ with the product topology. Each $\Delta(S_i)$ is compact (closed and bounded in $\mathbb{R}^{n_i}$). By Tychonoff's theorem, the finite product $\Sigma$ is compact. Moreover, $\Sigma$ is convex since it is a product of convex sets.

\medskip
\noindent\textbf{Step 2: Belief computation.}
Given a strategy profile $\sigma \in \Sigma$, we compute beliefs for each player type.

\emph{Genuine beliefs (humans about humans):} For $i, k \in N_H$, define first-order beliefs $h_i^{(1,k)}(\sigma) = \sigma_k$ and second-order beliefs $h_i^{(2,k)}(\sigma) = \sigma_i$. The second-order construction reflects the equilibrium interpretation: in equilibrium with correct beliefs, human $k$'s expectation about human $i$'s play equals $i$'s actual equilibrium strategy $\sigma_i$. Thus $h_i^{(2,k)}(\sigma) = \sigma_i$ captures ``what $i$ believes $k$ expected from $i$.'' Each belief is a coordinate projection, hence continuous in $\sigma$.

\emph{Attributed beliefs (humans about AI):} For human $i \in N_H$ and AI $j \in N_A$, the attributed second-order belief $\tilde{h}_i^{(2,j)}$ represents what $i$ believes about $j$'s ``expectation'' regarding $i$'s strategy. The framework admits three attribution approaches:
\begin{enumerate}
    \item[(i)] \emph{Signal-based attribution}: $\phi_i^{sig}(\theta_j, x_j, \omega_i) = g(x_j, \omega_i)$ depends only on observable AI signals $x_j$ and anthropomorphism tendency $\omega_i$.

    \item[(ii)] \emph{Dispositional attribution}: $\phi_i^{disp}(\theta_j, x_j, \omega_i) = \omega_i \cdot \bar{h} + (1-\omega_i) \cdot \underline{h}$ depends only on $\omega_i$ and fixed reference distributions $\bar{h}, \underline{h} \in \Delta(S_i)$.

    \item[(iii)] \emph{Behavioral attribution}: $\phi_i^{beh}(\theta_j, x_j, \sigma_j, \omega_i) = g(\sigma_j, \omega_i)$ depends on AI's observed strategy $\sigma_j$, creating a feedback loop between attributed beliefs and equilibrium play.
\end{enumerate}

\noindent\emph{This theorem covers Cases (i) and (ii): exogenous attribution.} Under signal-based or dispositional attribution, the attribution function takes the form
\[
    \tilde{h}_i^{(2,j)} = \phi_i(\theta_j, x_j, \omega_i)
\]
where the arguments $(\theta_j, x_j, \omega_i)$ are exogenous parameters---AI design, observable signals, and human traits---none of which depend on equilibrium strategies $\sigma$. Therefore, $\tilde{h}_i^{(2,j)}$ is a \emph{constant} (independent of $\sigma$), determined before the game is played.

\emph{Attribution validity.} By A2 (Attribution Continuity), the attribution function maps to valid probability distributions: $\phi_i: \Theta_j \times X \times \Omega_i \to \Delta(S_i)$. For the linear specification $\tilde{h}_i^{(2,j)}(C) = \omega_i(\rho_j + \eta x_j)$, validity requires the parameter restriction $\rho_j + \eta x_j \leq 1$. Under this restriction, since $\omega_i \in [0,1]$, we have $\tilde{h}_i^{(2,j)}(C) \in [0,1]$ and thus $\tilde{h}_i^{(2,j)} \in \Delta(S_i)$. Alternative specifications (truncation, logistic) satisfy validity without parameter restrictions; see Section~\ref{sec:framework}.

This constancy is the key simplification: attributed beliefs do not create additional equilibrium conditions. Human $i$'s optimization problem takes $\tilde{h}_i^{(2,j)}$ as given, just as she would take AI design parameters as given. The fixed-point argument operates on strategies $\Sigma$ alone, without jointly solving for beliefs.

\medskip
\noindent\textbf{Step 3: Best-response correspondence.}
For each human $i \in N_H$, define $BR_i^H: \Sigma \rightrightarrows \Delta(S_i)$ by
\[
    BR_i^H(\sigma) = \arg\max_{\sigma_i' \in \Delta(S_i)} \mathbb{E}_{\sigma_i', \sigma_{-i}}\left[ U_i^H(s, h_i(\sigma), \tilde{h}_i) \right]
\]
where beliefs $h_i(\sigma), \tilde{h}_i$ are computed as in Step 2.

For each AI $j \in N_A$, define $BR_j^A: \Sigma \rightrightarrows \Delta(S_j)$ by
\[
    BR_j^A(\sigma) = \arg\max_{\sigma_j' \in \Delta(S_j)} \mathbb{E}_{\sigma_j', \sigma_{-j}}\left[ U_j^A(s; \theta_j) \right].
\]
By A1c (continuous payoffs), $U_j^A(s; \theta_j)$ is continuous in $\sigma$, ensuring well-defined best responses.

We verify that each $BR_k$ (for $k \in N$) satisfies the conditions for Kakutani's theorem:
\begin{enumerate}
    \item[(i)] \emph{Non-empty valued}: The simplex $\Delta(S_k)$ is compact and the objective is continuous in $\sigma_k'$, so the maximum is attained by the Weierstrass extreme value theorem.

    \item[(ii)] \emph{Convex valued}: Expected utility is multilinear in the strategy profile, hence linear in $\sigma_k'$ for fixed $\sigma_{-k}$. Any convex combination of maximizers is therefore a maximizer.

    \item[(iii)] \emph{Upper hemicontinuous}: The constraint set $\Delta(S_k)$ is constant (hence continuous). The objective is continuous in $(\sigma_k', \sigma)$: the material component $\pi_i(s)$ is continuous by A1c; the psychological component $\psi_i(s, h_i^{(2)}(\sigma), \tilde{h}_i^{(2)})$ is continuous by A1c (continuity of $\psi_i$) and continuity of the belief mappings (coordinate projections for genuine beliefs, constants for attributed beliefs). By Berge's Maximum Theorem, $BR_k$ is upper hemicontinuous.
\end{enumerate}

\medskip
\noindent\textbf{Step 4: Fixed-point application.}
Define the joint correspondence $BR: \Sigma \rightrightarrows \Sigma$ by $BR(\sigma) = \prod_{k \in N} BR_k(\sigma)$. The product of non-empty, convex-valued correspondences inherits these properties.

We verify that $BR$ has a closed graph. Let $(\sigma^n, \hat{\sigma}^n) \to (\sigma, \hat{\sigma})$ with $\hat{\sigma}^n \in BR(\sigma^n)$ for all $n$. For each player $k$, we have $\hat{\sigma}_k^n \in BR_k(\sigma^n)$, meaning $\hat{\sigma}_k^n$ maximizes the objective over $\Delta(S_k)$ given $\sigma^n$. Since $\Delta(S_k)$ is compact and the objective is continuous in $(\sigma_k', \sigma)$, the sequence $\hat{\sigma}_k^n \to \hat{\sigma}_k$ and $\hat{\sigma}_k$ maximizes the objective given $\sigma$ (by Berge's theorem applied to each component). Thus $\hat{\sigma} \in BR(\sigma)$, and $BR$ has a closed graph. (Note: for correspondences with compact range, closed graph is equivalent to upper hemicontinuity.)

By Kakutani's fixed-point theorem: $\Sigma$ is non-empty, compact, and convex; $BR(\sigma)$ is non-empty and convex for all $\sigma$; $BR$ has a closed graph. Therefore, there exists $\sigma^* \in \Sigma$ with $\sigma^* \in BR(\sigma^*)$.

\medskip
\noindent\textbf{Step 5: Equilibrium verification.}
We verify that the fixed point $\sigma^*$, together with beliefs computed from Step 2, satisfies all four ABE conditions.

\emph{(ABE1) Human Optimality}: For $i \in N_H$, the fixed-point property gives $\sigma_i^* \in BR_i^H(\sigma^*)$, so $\sigma_i^*$ maximizes $U_i^H$ given $\sigma_{-i}^*$ and the induced beliefs.

\emph{(ABE2) AI Optimality}: For $j \in N_A$, we have $\sigma_j^* \in BR_j^A(\sigma^*)$, so $\sigma_j^*$ maximizes $U_j^A$ given $\sigma_{-j}^*$.

\emph{(ABE3) Genuine Belief Consistency}: Define equilibrium beliefs by:
\begin{align*}
    h_i^{*(1,k)} &:= h_i^{(1,k)}(\sigma^*) = \sigma_k^* \\
    h_i^{*(2,k)} &:= h_i^{(2,k)}(\sigma^*) = \sigma_i^*
\end{align*}
First-order consistency holds: $h_i^{*(1,k)} = \sigma_k^*$ (beliefs equal actual strategies). For second-order consistency, we need $h_i^{*(2,k)} = h_k^{*(1,i)}$. By construction:
\[
    h_i^{*(2,k)} = \sigma_i^* = h_k^{(1,i)}(\sigma^*) = h_k^{*(1,i)}
\]
Thus $i$'s belief about what $k$ expected from $i$ equals what $k$ actually expected from $i$.

\emph{(ABE4) Attribution Consistency}: Define $\tilde{h}_i^{*(2,j)} := \phi_i(\theta_j, x_j, \omega_i)$. This is exactly the attribution consistency condition: attributed beliefs are determined by the attribution function.

All four ABE conditions are satisfied. Therefore, $(\sigma^*, h^*, \tilde{h}^*)$ is an Attributed Belief Equilibrium.
\end{proof}

%==============================================================================
% REMARKS
%==============================================================================

\begin{remark}[Behavioral Attribution: Future Extension]
\label{rem:behavioral}
Case (iii)---behavioral attribution---creates a feedback loop: $\tilde{h}_i^{(2,j)}(\sigma) = \phi_i^{beh}(\theta_j, x_j, \sigma_j, \omega_i)$ depends on AI's equilibrium strategy $\sigma_j$, which itself depends on other players' strategies.

Extending Theorem~\ref{thm:existence} to behavioral attribution requires additional analysis. The key complication: when $\tilde{h}_i^{(2,j)}$ depends on $\sigma_j$, the objective function $U_i^H(s, h_i(\sigma), \tilde{h}_i(\sigma))$ is no longer separable between genuine and attributed beliefs. Verifying convexity of $BR_i^H(\sigma)$ requires additional assumptions ensuring that the composite $\phi_i^{beh}(\theta_j, x_j, \sigma_j, \omega_i)$ depends only on $\sigma_j$ (not on the full profile $\sigma$) and that linearity is preserved.

We conjecture that existence holds under the additional assumption:
\begin{quote}
    \textbf{(A2')} \emph{Behavioral attribution continuity}: $\phi_i^{beh}$ is continuous in $\sigma_j \in \Delta(S_j)$ for fixed $(\theta_j, x_j, \omega_i)$, and depends only on $\sigma_j$, not on other players' strategies.
\end{quote}
A complete treatment of behavioral attribution is deferred to future work.
\end{remark}

\begin{remark}[Role of Assumptions]
\label{rem:assumptions}
The proof uses the following assumptions:
\begin{itemize}
    \item A1a (finite strategy spaces): for compactness of $\Sigma$ and well-defined best responses;
    \item A1b (compact type spaces): for uniform boundedness in Lemma~\ref{lem:bounded-psi};
    \item A1c (continuous payoffs): for Berge's Maximum Theorem and upper hemicontinuity of $BR$;
    \item A2 (attribution continuity and validity): for attributed beliefs to be valid probability distributions in $\Delta(S_i)$ and continuous in parameters.
\end{itemize}
Lemma~\ref{lem:bounded-psi} shows that A3 (bounded psychological payoffs) is implied by A1--A2, hence redundant as a separate assumption.
\end{remark}

\begin{remark}[Comparison with Standard PGT]
\label{rem:comparison}
The ABE existence argument is structurally simpler than existence proofs for standard psychological game equilibria \citep{battigalli2009dynamic}. Attributed beliefs are exogenously determined---constants, not equilibrium variables. The fixed-point problem is confined to the finite-dimensional space $\Sigma$, not the infinite-dimensional belief space. This simplification arises because AI agents lack genuine mental states, so there is no belief consistency requirement linking attributed beliefs to AI ``expectations.''
\end{remark}

\begin{remark}[Topology and Generalizations]
\label{rem:topology}
Our restriction to finite strategy spaces ensures that the standard Euclidean topology on simplices coincides with the weak* topology on probability measures. This equivalence, which fails in infinite dimensions, allows us to invoke Kakutani's fixed-point theorem in its classical finite-dimensional form. Extensions to infinite strategy spaces would require the weak* topology to preserve compactness via the Banach-Alaoglu theorem; see \citet{aliprantis2006infinite} for a comprehensive treatment.
\end{remark}
