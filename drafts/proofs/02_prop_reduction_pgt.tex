% Proof of Proposition: Reduction to Standard Psychological Game Theory
% For inclusion in ABE manuscript appendix
% When N_A = empty, ABE reduces to Psychological Nash Equilibrium

\noindent\textit{Scope clarification}: \citet{battigalli2009dynamic} define SPE for extensive-form psychological games. For static (normal-form) games---the setting of our ABE definition---SPE coincides with PNE of \citet{geanakoplos1989psychological}. We prove that ABE reduces to this static-game equilibrium concept when $N_A = \emptyset$.

\paragraph{PNE Definition.} For a static psychological game with player set $N$, a Psychological Nash Equilibrium consists of $(s^*, h^*)$ satisfying:
\begin{enumerate}[label=\textbf{(PNE\arabic*)}]
    \item \textbf{Optimality}: For all $i \in N$,
    \[
        s_i^* \in \arg\max_{s_i \in S_i} U_i(s_i, s_{-i}^*, h_i^*)
    \]
    where $U_i = \pi_i(s) + \psi_i(s, h_i^{(2)})$ depends on second-order beliefs.

    \item \textbf{Belief Consistency}: For all $i, k \in N$ with $k \neq i$,
    \begin{align*}
        h_i^{*(1,k)} &= s_k^* \quad \text{(first-order consistency)} \\
        h_i^{*(2,k)} &= h_k^{*(1,i)} \quad \text{(second-order consistency)}
    \end{align*}
\end{enumerate}

\begin{proof}
Suppose $N_A = \emptyset$, so $N = N_H$ (all players are human). We show that the ABE conditions reduce exactly to the PNE conditions.

\medskip
\textbf{Step 1: Vacuous Conditions (ABE2 and ABE4).}

Condition ABE2 requires AI optimality ``for all $j \in N_A$.'' Since $N_A = \emptyset$, this is a universal quantification over an empty set, which is vacuously true. No constraint is imposed.

Similarly, condition ABE4 requires attribution consistency ``for all $i \in N_H$ and $j \in N_A$.'' The inner quantification over $j \in N_A = \emptyset$ makes this vacuously true for every $i$. No constraint is imposed.

\medskip
\textbf{Step 2: Empty Attributed Belief System.}

The attributed belief system is $\tilde{h}^* = \{\tilde{h}_i^{*(2,j)}\}_{i \in N_H, j \in N_A}$. When $N_A = \emptyset$, this collection is empty: $\tilde{h}^* = \emptyset$.

For each human $i$, the collection of attributed second-order beliefs is:
\[
    \tilde{h}_i^{*(2)} = \{\tilde{h}_i^{*(2,j)}\}_{j \in N_A} = \emptyset.
\]

\medskip
\textbf{Step 3: Simplification of Psychological Payoffs.}

The psychological payoff functions in ABE take an additive form that separates contributions from genuine beliefs about humans and attributed beliefs about AI (see Definitions~\ref{def:indignation} and \ref{def:guilt}):
\[
    \psi_i(s, h_i^{(2)}, \tilde{h}_i^{(2)}) = \underbrace{\psi_i^H(s, \{h_i^{(2,k)}\}_{k \in N_H \setminus \{i\}})}_{\text{genuine belief terms}} + \underbrace{\psi_i^A(s, \{\tilde{h}_i^{(2,j)}\}_{j \in N_A})}_{\text{attributed belief terms}}
\]
When $N_A = \emptyset$, the attributed belief terms vanish (sum over empty set), leaving:
\[
    \psi_i(s, h_i^{(2)}, \emptyset) = \psi_i^H(s, \{h_i^{(2,k)}\}_{k \in N_H \setminus \{i\}}) =: \hat{\psi}_i(s, h_i^{(2)})
\]
where $\hat{\psi}_i$ depends only on genuine second-order beliefs, exactly as in standard PGT.

\medskip
\textbf{Step 4: Reduction of Human Utility.}

Human $i$'s utility in ABE is:
\[
    U_i^H(s, h_i, \tilde{h}_i) = \pi_i(s) + \psi_i(s, h_i^{(2)}, \tilde{h}_i^{(2)}).
\]

By Step 3, when $N_A = \emptyset$:
\[
    U_i^H(s, h_i, \emptyset) = \pi_i(s) + \hat{\psi}_i(s, h_i^{(2)}) =: U_i(s, h_i).
\]

This is exactly the utility function in standard PGT.

\medskip
\textbf{Step 5: Reduction of ABE1 to PNE1.}

The ABE human optimality condition (ABE1) states:
\[
    s_i^* \in \arg\max_{s_i \in S_i} U_i^H(s_i, s_{-i}^*, h_i^*, \tilde{h}_i^*).
\]

By Step 4, when $N_A = \emptyset$:
\[
    s_i^* \in \arg\max_{s_i \in S_i} U_i(s_i, s_{-i}^*, h_i^*).
\]

Since $N = N_H$ when $N_A = \emptyset$, this condition applies to all $i \in N$, which is exactly PNE1.

\medskip
\textbf{Step 6: Reduction of ABE3 to PNE2.}

The ABE genuine belief consistency condition (ABE3) states: for all $i, k \in N_H$ with $k \neq i$,
\begin{align*}
    h_i^{*(1,k)} &= s_k^* \\
    h_i^{*(2,k)} &= h_k^{*(1,i)}.
\end{align*}

When $N_A = \emptyset$, we have $N_H = N$. Thus, ABE3 becomes: for all $i, k \in N$ with $k \neq i$,
\begin{align*}
    h_i^{*(1,k)} &= s_k^* \\
    h_i^{*(2,k)} &= h_k^{*(1,i)}.
\end{align*}

This is exactly PNE2.

\medskip
\textbf{Step 7: Conclusion.}

When $N_A = \emptyset$:
\begin{itemize}
    \item ABE2 and ABE4 are vacuously satisfied (Step 1)
    \item The attributed belief system is empty (Step 2)
    \item Psychological payoffs depend only on genuine beliefs (Step 3)
    \item Human utility reduces to standard PGT utility (Step 4)
    \item ABE1 reduces to PNE1 (Step 5)
    \item ABE3 reduces to PNE2 (Step 6)
\end{itemize}

Therefore, $(s^*, h^*, \tilde{h}^*)$ is an ABE if and only if $(s^*, h^*)$ is a PNE. Since PNE for static games coincides with SPE restricted to static games \citep{battigalli2009dynamic}, the result follows.
\end{proof}

\begin{remark}[Converse Direction]
The proof establishes a bijection between ABE and PNE when $N_A = \emptyset$. Given any PNE $(s^*, h^*)$, we can construct an ABE $(s^*, h^*, \emptyset)$ by taking the attributed belief system to be empty. Conversely, any ABE $(s^*, h^*, \tilde{h}^*)$ with $N_A = \emptyset$ necessarily has $\tilde{h}^* = \emptyset$, and $(s^*, h^*)$ is a PNE.
\end{remark}
