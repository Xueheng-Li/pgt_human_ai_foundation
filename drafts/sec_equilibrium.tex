% Section 3: Equilibrium
\section{Attributed Belief Equilibrium}
\label{sec:equilibrium}

This section defines Attributed Belief Equilibrium (ABE) and establishes its existence under the regularity conditions introduced above.

\subsection{Definition}

\begin{defn}[Attributed Belief Equilibrium]
\label{def:ABE}
A strategy profile $s^* = (s_H^*, s_A^*)$, genuine belief system $h^*$, and attributed belief system $\tilde{h}^*$ constitute an \textbf{Attributed Belief Equilibrium} if:
\begin{enumerate}
    \item \textbf{Human Optimality}: For all $i \in N_H$,
    \begin{equation}
        s_i^* \in \arg\max_{s_i \in S_i} U_i^H(s_i, s_{-i}^*, h_i^*, \tilde{h}_i^*)
    \end{equation}

    \item \textbf{AI Optimality}: For all $j \in N_A$,
    \begin{equation}
        s_j^* \in \arg\max_{s_j \in S_j} U_j^A(s_j, s_{-j}^*; \theta_j)
    \end{equation}

    \item \textbf{Genuine Belief Consistency}: For all $i, k \in N_H$,
    \begin{align}
        h_i^{*(1,k)} &= s_k^* \quad \text{(first-order consistency)} \\
        h_i^{*(2,k)} &= h_k^{*(1,i)} \quad \text{(second-order consistency)}
    \end{align}

    \item \textbf{Attribution Consistency}: For all $i \in N_H$ and $j \in N_A$,
    \begin{equation}
        \tilde{h}_i^{*(2,j)} = \phi_i(\theta_j, x_j, \omega_i)
    \end{equation}
\end{enumerate}
\end{defn}

The definition captures two distinct consistency requirements. Genuine beliefs about other humans satisfy standard Bayesian consistency as in \citet{battigalli2009dynamic}. Attributed beliefs about AI satisfy attribution consistency: they are determined by the attribution function given AI characteristics and the human's anthropomorphism tendency.

\begin{remark}[Mixed Strategies]
The definition extends naturally to mixed strategies by replacing pure strategies $s_i$ with mixed strategies $\sigma_i \in \Delta(S_i)$ and taking expectations over strategy profiles.
\end{remark}

Three key features distinguish ABE from standard psychological game equilibria:
\begin{enumerate}
    \item \textbf{Dual consistency}: Genuine beliefs (H-H) satisfy Bayesian consistency; attributed beliefs (H-A) satisfy attribution consistency.
    \item \textbf{Asymmetric rationality}: Humans are psychologically rational; AI are design-rational.
    \item \textbf{No cognizability for attributed beliefs}: Unlike \citet{battigalli2009dynamic}, attributed beliefs need not be ``cognizable'' since AI lacks genuine beliefs.
\end{enumerate}

\subsection{Existence}

\begin{theorem}[Existence of ABE]
\label{thm:existence}
Under Assumptions A1--A3, an Attributed Belief Equilibrium exists.
\end{theorem}

\begin{proof}[Proof Sketch]
The proof extends the fixed-point approach of \citet{battigalli2009dynamic} to dual belief structures.

\textbf{Step 1: Construct the relevant spaces.} Define the space of strategy profiles $\Sigma = \prod_{i \in N} \Delta(S_i)$, which is non-empty, compact, and convex by A1. For each human $i$, the attributed belief is pinned down by $\tilde{h}_i^{(2,j)} = \phi_i(\theta_j, x_j, \omega_i)$, which is well-defined and continuous by A2.

\textbf{Step 2: Define the best-response correspondence.} For humans, given beliefs $(h_i, \tilde{h}_i)$, define
\[
    BR_i^H(h_i, \tilde{h}_i) = \arg\max_{\sigma_i \in \Delta(S_i)} \mathbb{E}[U_i^H(\sigma_i, \sigma_{-i}, h_i, \tilde{h}_i)].
\]
For AI, define
\[
    BR_j^A(\sigma_{-j}) = \arg\max_{\sigma_j \in \Delta(S_j)} \mathbb{E}[U_j^A(\sigma_j, \sigma_{-j}; \theta_j)].
\]
By A1 (compactness and continuity), these correspondences are non-empty, convex-valued, and upper hemicontinuous.

\textbf{Step 3: Construct the fixed-point mapping.} Define $\Phi: \Sigma \to \Sigma$ as follows. Given strategy profile $\sigma$:
\begin{enumerate}
    \item Compute attributed beliefs: $\tilde{h}_i^{(2,j)} = \phi_i(\theta_j, x_j, \omega_i)$
    \item Compute genuine belief consistency: $h_i^{(1,k)} = \sigma_k$ and $h_i^{(2,k)} = h_k^{(1,i)}$
    \item Apply best responses: $\Phi(\sigma) = BR^H(h, \tilde{h}) \times BR^A(\sigma)$
\end{enumerate}
By Kakutani's fixed-point theorem, $\Phi$ has a fixed point $\sigma^*$.

\textbf{Step 4: Verify equilibrium conditions.} At the fixed point, human optimality holds by construction of $BR_i^H$, AI optimality holds by construction of $BR_j^A$, genuine belief consistency holds by the construction in Step 3, and attribution consistency holds by definition of $\tilde{h}_i^{(2,j)}$.
\end{proof}

\begin{remark}[Role of Bounded Psychological Payoffs]
Assumption A3 ensures that the best-response correspondences are well-behaved. Without boundedness, psychological payoffs could dominate material payoffs arbitrarily, potentially leading to non-existence.
\end{remark}

\subsection{Special Cases and Nesting}

\begin{proposition}[Reduction to Standard PGT]
\label{prop:nesting}
When $N_A = \emptyset$ (no AI agents), ABE reduces to Psychological Nash Equilibrium (PNE) as defined in \citet{geanakoplos1989psychological}, which coincides with Sequential Psychological Equilibrium (SPE) of \citet{battigalli2009dynamic} for static games.
\end{proposition}

The proof shows that when $N_A = \emptyset$, conditions ABE2 and ABE4 are vacuously satisfied, psychological payoffs simplify to depend only on genuine beliefs, and ABE1/ABE3 reduce exactly to PNE optimality and belief consistency. See Appendix~\ref{app:proofs}.

\begin{proposition}[Reduction to Nash]
\label{prop:nash}
When $\psi_i \equiv 0$ for all $i \in N_H$ (no psychological payoffs), ABE strategy profiles coincide with Nash equilibria of the material game $\Gamma^M = (N, \{S_i\}, \{\pi_i\}_{i \in N_H}, \{U_j^A(\cdot; \theta_j)\}_{j \in N_A})$.
\end{proposition}

The proof establishes a bijection: when $\psi_i \equiv 0$, human utility reduces to $U_i^H = \pi_i$, so beliefs become ``strategically irrelevant'' and the optimality conditions reduce to Nash best responses. See Appendix~\ref{app:proofs}.

\begin{proposition}[Rational Attribution and Bayesian Game Equivalence]
\label{prop:rational-attribution}
A strategy profile $\sigma^*$ together with attribution function $\phi$ satisfy \textbf{rational attribution} if: (i) $\phi_i(\theta_j, x_j, \omega_i) = \sigma_i^*$ for all $i \in N_H, j \in N_A$ (attribution projects equilibrium), and (ii) $\sigma^*$ satisfies ABE optimality given $\phi$ (strategies are optimal). Under rational attribution, ABE strategy profiles correspond bijectively to Bayes-Nash equilibria of an equivalent Bayesian game $\Gamma^B$ with type uncertainty about AI design parameters.
\end{proposition}

Rational attribution is a fixed-point requirement: $\phi$ projects equilibrium play, and $\sigma^*$ is an equilibrium given $\phi$. Existence is not guaranteed for general games but holds in games with unique Nash equilibria. The proposition identifies when ABE reduces to standard Bayesian game theory---precisely when humans form ``correct'' beliefs about AI expectations. See Appendix~\ref{app:proofs} for the complete proof and construction of the equivalent Bayesian game $\Gamma^B$.

\begin{defn}[Rational Attribution Equilibrium]
\label{def:RAE}
Fix a psychological game $\Gamma$. A strategy profile $\sigma^* \in \prod_{k \in N} \Delta(S_k)$ together with attribution function $\phi$ constitute a \textbf{Rational Attribution Equilibrium (RAE)} if:
\begin{enumerate}
    \item[(i)] $\sigma^*$ is an Attributed Belief Equilibrium under $\phi$;
    \item[(ii)] For all $i \in N_H$ and $j \in N_A$:
    \begin{equation}
        \tilde{h}_i^{*(2,j)} = \phi_i(\theta_j, x_j, \omega_i) = \sigma_i^*
        \label{eq:RAE-fixed-point}
    \end{equation}
\end{enumerate}
Condition (ii) is a fixed-point requirement: the attributed belief about ``what AI $j$ expects human $i$ to do'' equals human $i$'s actual equilibrium strategy.
\end{defn}

RAE captures the case where humans correctly anticipate equilibrium behavior. The attributed belief coincides with actual play---no systematic bias in belief formation. This is analogous to rational expectations in macroeconomics: beliefs are model-consistent.

\begin{remark}[RAE vs. Zero-Anthropomorphism]
\label{rem:RAE-vs-ZA}
These concepts are distinct. Zero-anthropomorphism (Definition~\ref{def:zero-anthro}) characterizes attributed beliefs when $\omega_i = 0$---a limiting case independent of equilibrium. RAE (Definition~\ref{def:RAE}) requires attributed beliefs to equal equilibrium strategies---a fixed-point condition that may hold at any $\omega_i \in [0,1]$. Neither implies the other: zero-anthropomorphism typically yields $\tilde{h}_i^{(2,j),ZA} \neq \sigma_i^*$, while RAE can hold with positive anthropomorphism.
\end{remark}

\begin{proposition}[RAE Existence]
\label{prop:RAE-existence}
Consider a psychological game $\Gamma$ satisfying Assumptions A1--A3. Rational Attribution Equilibrium exists under either of the following conditions:
\begin{enumerate}
    \item[(i)] \textbf{Unique Nash benchmark}: The material game $\Gamma^M$ has a unique Nash equilibrium $\sigma^{NE}$, and $\phi_i(\theta_j, x_j, \omega_i) = \sigma_i^{NE}$ for all $i \in N_H$, $j \in N_A$.
    \item[(ii)] \textbf{Equilibrium-projecting attribution}: The attribution function satisfies
    \begin{equation}
        \phi_i(\theta_j, x_j, \omega_i) = \arg\max_{\sigma_i} \mathbb{E}[U_i^H(\sigma_i, \sigma_{-i}^*, h_i^*, \tilde{h}_i^*)]
    \end{equation}
    for some ABE $(\sigma^*, h^*, \tilde{h}^*)$.
\end{enumerate}
\end{proposition}

\begin{proof}[Proof Sketch]
(i) Under uniqueness, equilibrium strategies are pinned down by material payoffs. If attribution projects this unique equilibrium, the fixed-point condition (\ref{eq:RAE-fixed-point}) holds by construction.

(ii) This condition directly imposes the fixed-point: attributed beliefs equal optimal strategies. Existence follows from Theorem~\ref{thm:existence} applied to the restricted class of attribution functions satisfying this property. \qed
\end{proof}

\begin{remark}[Non-Existence of RAE]
\label{rem:RAE-nonexistence}
RAE need not exist in general. Consider a game where $\sigma_i^* \in (0,1)$ is the unique equilibrium mixing probability, but the attribution function satisfies $\phi_i(\theta_j, x_j, \omega_i) \in \{0, 1\}$ (pure-strategy attribution). The fixed-point condition (\ref{eq:RAE-fixed-point}) cannot be satisfied. RAE is a special case of ABE, not the generic outcome.
\end{remark}

\begin{corollary}[Complete Information Reduction]
\label{cor:complete-info}
When types are common knowledge (degenerate prior $p$), rational attribution ABE reduces to Nash equilibrium of a game with type-dependent payoffs.
\end{corollary}

\begin{remark}[Economic Interpretation]
Proposition~\ref{prop:rational-attribution} identifies when ABE reduces to Bayesian game theory: precisely when attribution is rational. ABE departs from Bayesian predictions when $\phi_i(\theta_j, x_j, \omega_i) \neq \sigma_i^*$---due to anthropomorphic bias, signal effects, or systematic misattribution. This creates design implications: interfaces inducing rational attribution yield Bayesian outcomes; strategic manipulation shifts outcomes away from this benchmark.
\end{remark}

\begin{remark}[Equilibrium Multiplicity under Rational Attribution]
If the underlying game has multiple equilibria, the rational attribution condition can hold for at most one equilibrium per attribution function $\phi$. This is because $\phi_i(\theta_j, x_j, \omega_i)$ is a deterministic function of its arguments---it cannot output different values for different equilibria. The proposition establishes correspondence for each equilibrium satisfying the fixed-point condition separately.
\end{remark}

\subsection{Relationship Between Equilibrium Concepts}
\label{sec:concept-relationships}

The framework yields a nested hierarchy of equilibrium concepts. We clarify terminology and explain when each applies.

\subsubsection{Conceptual Hierarchy}

Three equilibrium concepts have been introduced:
\begin{enumerate}
    \item \textbf{Attributed Belief Equilibrium (ABE)}: The general concept (Definition~\ref{def:ABE}). Attributed beliefs are determined by the attribution function $\phi_i$, which depends on AI characteristics $(\theta_j, x_j)$ and human anthropomorphism $\omega_i$. No restriction on the relationship between attributed beliefs and equilibrium strategies.

    \item \textbf{Rational Attribution Equilibrium (RAE)}: A refinement of ABE (Definition~\ref{def:RAE}). Attributed beliefs equal equilibrium strategies: $\tilde{h}_i^{*(2,j)} = \sigma_i^*$. This is a fixed-point requirement analogous to rational expectations.

    \item \textbf{Zero-Anthropomorphism Benchmark}: A limiting case (Definition~\ref{def:zero-anthro}). Attributed beliefs when $\omega_i = 0$. This is a descriptive benchmark, not an equilibrium concept.
\end{enumerate}

The nesting is:
\begin{equation}
    \text{ABE} \supseteq \text{RAE} \supseteq \{\text{Zero-anthropomorphism ABE}\}
\end{equation}
where the final inclusion holds only when zero-anthropomorphism happens to yield RAE (typically not the case).

% Terminological notes: "Rational" refers to the fixed-point property in RAE (model-consistency, not normative judgment). "Zero-anthropomorphism" is the limiting case $\omega_i = 0$ (descriptive baseline). "Elevated anthropomorphism" means $\omega_i > 0$ leads to attributed beliefs exceeding zero-anthropomorphism benchmark.

\subsubsection{When Each Concept Applies}

ABE is the general concept, accommodating heterogeneous anthropomorphism and heuristic attribution. RAE applies when attributed beliefs match equilibrium play---under learning, mechanism design, or unique salient equilibria. The zero-anthropomorphism benchmark enables comparative statics by isolating anthropomorphism effects ($\omega_i = 0$ vs.\ $\omega_i > 0$).

\subsubsection{Relationship to Standard Game Theory}

Proposition~\ref{prop:rational-attribution} establishes that RAE corresponds bijectively to Bayes-Nash equilibria of an equivalent Bayesian game. The ABE framework differs non-trivially from standard game theory precisely when attribution is \emph{not} rational---when $\tilde{h}_i^{*(2,j)} \neq \sigma_i^*$. This departure arises from:
\begin{itemize}
    \item \textbf{Anthropomorphic bias}: $\omega_i > 0$ inflates attributed beliefs beyond what equilibrium play warrants
    \item \textbf{Signal effects}: Interface design ($x_j$) shapes attribution independent of equilibrium
    \item \textbf{Systematic misattribution}: Humans project their own cooperative tendencies onto AI
\end{itemize}

The practical import: ABE captures departures from Bayesian rationality that are empirically documented in human-AI interaction. RAE is the benchmark where no such departures occur.

\subsection{Multiplicity}

Different attribution functions can sustain different equilibria in the same material game.

\begin{proposition}[Attribution-Dependent Multiplicity]
\label{prop:multiplicity}
Consider a psychological game $\Gamma$ with at least one human and one AI. Suppose for some human $i \in N_H$ and AI $j \in N_A$:
\begin{enumerate}
    \item[(i)] \textbf{Belief-dependent payoffs}: $\partial \psi_i / \partial \tilde{h}_i^{(2,j)} \neq 0$ for some strategy profile.
    \item[(ii)] \textbf{Distinct attributions}: $\phi_i(\theta_j, x_j, \omega_i) \neq \phi'_i(\theta_j, x_j, \omega_i)$ for some configuration.
    \item[(iii)] \textbf{Best-response separation}: The change in attributed beliefs shifts equilibrium strategies.
\end{enumerate}
Then $\Gamma$ admits ABE under $\phi$ and $\phi'$ with distinct equilibrium strategy profiles: $s^*(\phi) \neq s^*(\phi')$.
\end{proposition}

\begin{remark}[Comparison with Standard PGT Multiplicity]
In standard psychological game theory, multiplicity arises from feedback between equilibrium strategies and belief consistency. In ABE, a distinct source emerges: attributed beliefs are exogenously fixed by $\phi$, so changing $\phi$ directly changes utility and thereby equilibrium. This feed-forward structure---$\phi \to \tilde{h}^{(2)} \to U^H \to s^*$---makes ABE multiplicity conceptually cleaner.
\end{remark}

\begin{remark}[Genericity]
Conditions (i)--(iii) are generic. Condition (i) fails only when $\psi_i$ is independent of attributed beliefs. Condition (ii) fails only when all attribution functions coincide. Condition (iii) fails only when payoff changes leave best responses unchanged. Attribution-dependent multiplicity is the typical outcome, not an exceptional one.
\end{remark}

This multiplicity has important design implications. Interface design, framing, and behavioural presentation affect attribution patterns and thereby equilibrium selection---even with fixed material payoffs. A prosocial AI signalling expectations through humanlike cues induces more cooperation than an equally prosocial AI with a mechanical interface.

\subsubsection*{Illustrative Examples}

We provide three numerical examples demonstrating the proposition.

\paragraph{Example 1: Trust Game.}
AI trustor ($E = 10$, $\rho_A = 0.3$), human trustee ($\gamma_H = 2$, $\lambda_H^{GUILT} = 0.5$, $\omega_H = 0.8$). Under anthropomorphic interface: $\tilde{h}_H^{(2,A)} = 15.2$, return $y^* = 15.2$. Under mechanical interface: $\tilde{h}_H^{(2,A)} = 7.2$, return $y^* = 7.2$. The interface doubles AI payoff through elevated attributed expectations.

\paragraph{Example 2: Public Goods.}
One human, one AI ($E = 10$, $m = 1.5$, binary contributions). AI contributes $c_A = 10$. Human: $\beta_H = 8$, $\lambda_H^{IND} = 0.5$. Material gain from defection: 2.5. Under high attribution ($\tilde{h}_H^{(2,A)}(C) = 0.64$): indignation cost $= 2.56 > 2.5$, human cooperates. Under low attribution ($\tilde{h}_H^{(2,A)}(C) = 0.16$): indignation cost $= 0.64 < 2.5$, human defects.

\paragraph{Example 3: Coordination.}
One human, one AI choosing technology $\{A, B\}$. Coordination: 2; miscoordination: 0. AI plays $A$; human has $\beta_H = 3$. Under $\phi$ (AI expects $A$ with probability 0.9): human plays $A$, ABE is $(A, A)$. Under $\phi'$ (AI expects $B$ with probability 0.9): human plays $B$, ABE is $(B, A)$ with payoffs $(0, 0)$. Attribution serves as equilibrium selection device.
