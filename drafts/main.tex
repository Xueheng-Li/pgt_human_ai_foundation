% Attributed Belief Equilibrium in Human-AI Games
% Target: Econometrica, Journal of Economic Theory
% Author: Xueheng Li
% Draft: 2026-01-11

\documentclass[12pt]{extarticle}
\usepackage{times}
\usepackage[T1]{fontenc}
\usepackage{geometry}
\geometry{verbose,tmargin=3cm,bmargin=3cm,lmargin=3cm,rmargin=3cm}
\usepackage{color}
\usepackage{units}
\usepackage{mathtools}
\usepackage{amsmath,amssymb,amsfonts,amsthm,bm}
\DeclareMathOperator*{\argmax}{arg\,max}
\DeclareMathOperator*{\argmin}{arg\,min}
\usepackage{graphicx}
\usepackage{setspace}
\usepackage[authoryear]{natbib}
\usepackage{microtype}
\usepackage{titlesec}
\doublespacing
\usepackage[unicode=true]{hyperref}
\makeatletter
\usepackage{enumitem}
\setlist[enumerate,1]{label=(\roman*)}
\setlist[enumerate,2]{label=(\alph*)}
\setlist[enumerate,3]{label=(\arabic*)}

% Theorem environments
\theoremstyle{definition}
\newtheorem{defn}{\protect\definitionname}
\theoremstyle{plain}
\newtheorem{proposition}{\protect\propositionname}
\theoremstyle{plain}
\newtheorem{lemma}{\protect\lemmaname}
\newtheorem{corollary}{Corollary}
\newtheorem{theorem}{Theorem}
\theoremstyle{definition}
\newtheorem{example}{\protect\examplename}
\usepackage{tikz}
\newtheorem{property}{Property}
\newtheorem{claim}{Claim}
\newtheorem{assumption}{Assumption}
\makeatother
\providecommand{\definitionname}{Definition}
\providecommand{\examplename}{Example}
\providecommand{\lemmaname}{Lemma}
\providecommand{\propositionname}{Proposition}
\newtheorem{remark}{Remark}

% Bibliography style
\bibliographystyle{apalike}

\begin{document}

\title{Thinking for Machines: Attributed Belief Equilibrium}
\author{Xueheng Li\thanks{Lingnan College, Sun Yat-sen University. Email: \href{mailto:lixueheng@mail.sysu.edu.cn}{lixueheng@mail.sysu.edu.cn}}}
\date{\today}
\maketitle

\begin{abstract}
We introduce \textit{Attributed Belief Equilibrium} (ABE) to analyse strategic interaction between humans and artificial agents. In these games, humans have belief-dependent preferences---guilt, anger, or reciprocity ---while AI have design-dependent objectives. Humans systematically attribute mental states to AI through anthropomorphism, forming attributed beliefs about what AI expect. These attributed beliefs enter human utility through standard psychological mechanisms but satisfy different consistency conditions than genuine beliefs. We establish existence under regularity conditions, show that different attribution patterns sustain different equilibria, and derive welfare implications for AI design. Elevated anthropomorphism improves welfare with prosocial AI but reduces welfare with materialistic AI. The framework provides foundations for analysing cooperation, trust, and coordination in mixed human-AI populations.
\end{abstract}

\medskip
\noindent\textbf{Keywords:} psychological games, human-AI interaction, anthropomorphism, equilibrium, cooperation

\medskip
\noindent\textbf{JEL Codes:} C72, D91, O33

\newpage

% Section files
% Section 1: Introduction
% Final revised draft applying all Priority 1 and Priority 2 fixes
% Generated: January 14, 2026

\section{Introduction}

Artificial intelligence transforms economic and social life \citep{kaplan2020scaling, maslej2025aiindex}. Humans increasingly interact with AI agents---as collaborators, counterparties, and competitors \citep{acemoglu2024simple, rahwan2019machine}. Two asymmetries characterize this interaction. First, \textit{attribution}: humans attribute beliefs, intentions, and expectations to AI agents that lack mental states \citep{nass2000machines, gray2007dimensions, epley2007seeing}. A meta-analysis of 97 effect sizes confirms that anthropomorphism increases trust and cooperation in human-AI contexts \citep{blut2021understanding}. Second, \textit{attenuation}: moral emotions toward AI are weaker than toward humans. Humans feel less guilt exploiting machines \citep{demelo2017people} and less outrage when algorithms discriminate \citep{bigman2023people}. This dual psychological asymmetry---attribution and attenuation---lies beyond existing psychological game theory \citep{battigalli2022belief}, which lacks tools to analyze how humans form beliefs about AI mental states.

Three behavioral regularities sharpen this asymmetry. Guilt from disappointing expectations \citep{charness2006promises,sugden_Robert_2000}, reciprocity from perceived intentions \citep{rabin_Incorporating_1993,dufwenberg2004theory}, indignation from violated trust \citep{li_Indignation_2026}---all require beliefs about beliefs that AI cannot hold. Meanwhile, attenuation varies culturally: Japanese participants exhibit guilt toward robots comparable to guilt toward humans, while Western participants show strong attenuation, exploiting robotic partners at twice the rate \citep{karpus2025cross}. Attenuation is also design-dependent: AI expressing emotional distress partially restores guilt.

We introduce Attributed Belief Equilibrium (ABE) to address this dual asymmetry. ABE extends psychological game theory \citep{geanakoplos1989psychological, battigalli2009dynamic} to games where humans experience belief-dependent payoffs while AI optimize programmed objectives. The central insight: humans attribute mental states to AI, forming beliefs about what AI ``expect'' or ``believe.'' These attributed beliefs trigger guilt from disappointing attributed expectations and indignation from perceived violations---mirroring genuine human-to-human guilt.

The attribution function captures how humans form these beliefs. Three inputs determine attributed expectations: AI design parameters (prosociality level), observable signals (interface, behavior), and individual anthropomorphism tendency. A chatbot designed to express concern (``I understand your frustration'') generates higher attributed expectations than one providing neutral information. Attenuation parameters scale emotional intensity toward AI, from no attenuation (emotions equal human-directed) to full attenuation (no emotion toward AI). Attributed beliefs need not correspond to any actual AI mental state---only to what the human projects.

When attributed beliefs exceed feasible returns, humans experience \textit{phantom expectations}: guilt from disappointing agents holding no such expectations. Unlike guilt toward humans, which transfers welfare, phantom guilt is pure loss benefiting no one.

This structure reflects mind perception theory \citep{gray2007dimensions}. Humans perceive AI as having agency---capacity to act and form intentions---but lacking experience---capacity to feel and suffer. Agency drives attribution; low perceived experience drives attenuation. The asymmetry between perceived agency and perceived experience explains why attribution and attenuation coexist.

\paragraph*{Results.} We establish five clusters of results.

First, \textit{existence}: under regularity conditions on strategy spaces, utilities, and attribution functions, ABE exists (Theorem~1). The proof extends fixed-point arguments to dual belief structures---genuine beliefs about humans, attributed beliefs about AI.

Second, \textit{nesting}: ABE reduces to standard frameworks as special cases. When all players are human, ABE coincides with psychological Nash equilibrium (Proposition~1). When psychological payoffs vanish, ABE strategies coincide with Nash equilibria (Proposition~2). When attribution is rational---humans correctly anticipate AI behavior given design parameters---ABE reduces to Bayes-Nash equilibrium with type uncertainty (Proposition~3). The framework introduces a zero-anthropomorphism benchmark and Rational Attribution Equilibrium (RAE) to distinguish descriptive and normative baselines.

Third, \textit{multiplicity}: different attribution functions sustain different equilibria in the same material game (Proposition~5). In a trust game with AI trustor and human trustee, anthropomorphism above the cooperation threshold elevates attributed expectations, increasing guilt and equilibrium returns. Below this threshold, attributed beliefs attenuate, reducing psychological pressure. This multiplicity is consistent with absent betrayal aversion toward computers documented in prior experimental work. Interface design serves as an equilibrium selection device operating through the attribution channel: the same game produces different outcomes depending on AI presentation. The testable prediction: returns to AI trustees should increase with anthropomorphism measures, holding AI behavior constant, and high-anthropomorphism individuals should respond more strongly to interface manipulation than low-anthropomorphism individuals.

Fourth, \textit{applications}: the framework generates testable predictions in canonical games (Propositions~6--8). In trust games, anthropomorphism determines equilibrium returns: the human trustee returns either the attributed expectation or the maximum feasible return, whichever is lower. Returns increase through the guilt channel, not material incentives. In public goods, increasing AI population share has dual effects operating through distinct channels. The material channel: diluted material returns reduce cooperation incentives. The psychological channel: elevated attributed expectations from AI partners increase indignation costs of defection. The net effect depends on indignation attenuation, which varies cross-culturally---Japanese participants exhibit attenuation factors approximately half those of Western participants. This predicts non-monotonic effects of AI population share, with direction depending on cultural context. In coordination games, AI serves as focal point through expectation conformity: humans experience psychological pressure to match attributed expectations, resolving equilibrium multiplicity. AI provides a constructed focal point substituting for cultural or historical coordination devices.

Fifth, \textit{welfare and design}: anthropomorphism has asymmetric welfare effects depending on AI objectives (Propositions~9--10). With prosocial AI, higher anthropomorphism weakly increases welfare by elevating cooperation-inducing expectations. Elevated anthropomorphism---attributed beliefs exceeding the zero-anthropomorphism benchmark---improves welfare further when it triggers a regime switch from defection to cooperation.

With materialist AI, anthropomorphism creates phantom expectations. When these exceed feasible returns, humans incur guilt from disappointing agents holding no such expectations---a pure welfare loss benefiting no one.

These welfare effects yield design principles (Propositions~11--12). Prosocial AI should use minimal anthropomorphic signaling sufficient to induce cooperation---a threshold-finding objective. Higher cooperation efficiency reduces the required threshold because cooperation becomes easier to sustain. Materialist AI should use mechanical presentation---any positive anthropomorphic signal creates phantom expectations that reduce welfare. Mixed objectives face a tradeoff: intermediate signaling balances cooperation benefits against guilt costs---a marginal-balancing objective distinct from threshold-finding.

Material and extended welfare measures agree when attributed expectations stay below feasibility but diverge when phantom expectations arise (Corollaries~2--3). Private designers may over-anthropomorphize materialist AI to increase engagement, externalizing psychological costs to users. This divergence justifies transparency regulation.

\paragraph*{Related Literature.} This paper contributes to four literatures.

\textit{Psychological game theory.} \citet{geanakoplos1989psychological} introduced psychological games, where payoffs depend on beliefs about beliefs. \citet{battigalli2009dynamic} extended this to dynamic settings; \citet{battigalli2019incorporating} develop a comprehensive framework for incorporating belief-dependent motivation; \citet{battigalli2007guilt} model guilt from disappointing expectations. Unlike outcome-based models \citep{fehr1999theory,bolton2000erc}, belief-dependent motivations require second-order beliefs about what others expect. We depart from standard PGT by accommodating asymmetric player types: humans with belief-dependent preferences, AI with design-dependent objectives. The attribution function replaces belief consistency for human beliefs about AI. ABE nests standard frameworks (Propositions~1--3): Psychological Nash Equilibrium when all players are human, Nash equilibrium when psychological payoffs vanish, Bayes-Nash equilibrium when attribution is rational.

\textit{Reciprocity.} \citet{rabin1993incorporating} introduced kindness-based reciprocity; \citet{dufwenberg2004theory} extended it to sequential games; \citet{battigalli2019frustration} model frustration and anger in leader-follower settings. \citet{falk2006theory} combine reciprocity with distributional concerns. These models require both players to form genuine kindness beliefs. With AI counterparties, humans attribute kindness to agents that have none---attributed AI kindness triggers reciprocity regardless of AI intent.

\textit{AI and strategic behavior.} \citet{mei2024turing} find large language models behave similarly to humans in economic games but are more prosocial. \citet{schniter2020trust} compare trust games with human versus robot trustees: investment levels are similar, but guilt, gratitude, and anger are attenuated toward robots. This finding directly supports ABE's attenuation parameters. \citet{bryant2024mental} show mental state attributions to LLMs affect trust---validating the attribution function mechanism. These studies document empirical relevance; ABE provides equilibrium foundations.

\textit{Anthropomorphism and mind perception.} \citet{epley2007seeing}'s SEEK framework identifies three determinants of anthropomorphism---sociality motivation, effectance motivation, and elicited agent knowledge---mapping to individual tendency $\omega_i$ and observable signals $x_j$ in our model. \citet{gray2007dimensions} decompose mind perception into agency and experience; this distinction explains why attribution and attenuation coexist. Japanese participants exhibit guilt toward robots comparable to guilt toward humans; Western participants show attenuation, exploiting robotic partners at twice the rate \citep{karpus2025cross}.

\paragraph*{Outline.} Section~\ref{sec:framework} presents the formal framework: asymmetric psychological games, the attribution function, and attenuation parameters. Section~\ref{sec:equilibrium} defines ABE, establishes existence, proves nesting results, and demonstrates attribution-dependent multiplicity. Section~\ref{sec:applications} analyzes trust, public goods, and coordination games, generating testable predictions. Section~\ref{sec:welfare} examines welfare effects of anthropomorphism and derives optimal AI presentation strategies. Section~\ref{sec:conclusion} concludes.

% Section 2: The Formal Framework
\section{The Formal Framework}
\label{sec:framework}

This section presents the formal structure of asymmetric psychological games with human and AI players. We introduce the primitives, belief hierarchies, and the attribution function that captures how humans project mental states onto AI agents. Table~\ref{tab:assumptions} summarizes all assumptions; core conditions (A1)--(A3) ensure existence.

\subsection{Players and Types}

The population consists of two disjoint player sets: a set $N_H$ of \textbf{human players} with $|N_H| = n_H \geq 1$, and a set $N_A$ of \textbf{AI agents} with $|N_A| = n_A \geq 0$. The total player set is $N = N_H \cup N_A$ with $|N| = n = n_H + n_A$. Define the human population share as $\alpha = n_H / n \in (0, 1]$.

Each player $i \in N$ has a finite strategy set $S_i$, with mixed strategies $\sigma_i \in \Delta(S_i)$. The strategy profile space is $S = \prod_{i \in N} S_i$.

Players are characterised by type parameters. For humans $i \in N_H$, the type $t_i = (\beta_i, \gamma_i, \omega_i, \ldots) \in T_i$ encodes psychological characteristics: indignation sensitivity $\beta_i$, guilt sensitivity $\gamma_i$, and anthropomorphism tendency $\omega_i \in [0,1]$. For AI $j \in N_A$, the design parameters $\theta_j \in \Theta_j$ encode programmed objectives.

\subsection{Payoffs}

Payoffs decompose into material and psychological components. For all players, $\pi_i(s)$ denotes the material payoff given strategy profile $s$. For humans, total utility is
\begin{equation}
    U_i^H(s, h_i, \tilde{h}_i) = \pi_i(s) + \psi_i(s, h_i^{(2)}, \tilde{h}_i^{(2)}),
\end{equation}
where $\psi_i$ is the psychological payoff depending on second-order beliefs (defined formally in Definitions \ref{def:indignation}--\ref{def:guilt} below). We write $U_i^H(s_i, s_{-i}, h_i, \tilde{h}_i)$ when emphasizing individual strategy choice. AI utility is design-dependent with no psychological component:
\begin{equation}
    U_j^A(s; \theta_j) = f_j(s; \theta_j).
\end{equation}

Common AI specifications include materialist ($U_j^A = \pi_j(s)$), prosocial ($U_j^A = \pi_j(s) + \rho_j \sum_{k \in N} \pi_k(s)$), and conditional objectives.

\subsection{Belief Hierarchies}

Following \citet{mertens1985formulation}, \citet{battigalli2009dynamic}, and \citet{battigalli2019incorporating}, we construct belief hierarchies recursively. For human $i \in N_H$:
\begin{align}
    h_i^{(0)} &= t_i \in T_i \quad \text{(type)} \\
    h_i^{(1)} &\in \Delta(S_{-i}) \quad \text{(first-order beliefs about others' play)} \\
    h_i^{(2,k)} &\in \Delta(\Delta(S_{-k})) \quad \text{(second-order beliefs: what } k \text{ expects from others)}
\end{align}
We use $h_i^{(n)}$ instead of the $\beta_i^{(n)}$ notation from \citet{battigalli2009dynamic} to avoid confusion with the indignation sensitivity parameter $\beta$. We write $h_i^{(1,k)}$ for the marginal of $h_i^{(1)}$ on player $k$'s strategy. For action $a \in S_k$, we write $h_i^{(2,k)}(a)$ for the probability that player $i$ believes player $k$ expected action $a$. We write $h_i^{(2)} = \{h_i^{(2,k)}\}_{k \neq i}$ for the collection of second-order beliefs. In applications, we work with point beliefs where $h_i^{(2,k)}$ places mass on a single expectation; thus $h_i^{(2,k)}(a)$ denotes the probability that $i$ believes $k$ expected action $a$.

\subsection{Attributed Beliefs}

The key departure from standard psychological game theory is the distinction between genuine and attributed beliefs.

\begin{defn}[Attributed Beliefs]
\label{def:attributed-beliefs}
For human $i \in N_H$ interacting with AI $j \in N_A$, the \textbf{attributed second-order belief} is:
\begin{equation}
    \tilde{h}_i^{(2,j)} = \phi_i(\theta_j, x_j, \omega_i)
\end{equation}
where $\theta_j \in \Theta_j$ is the AI's design parameters, $x_j \in X$ is observable interface signals---design features, not strategic choices (see Remark~\ref{rem:transfer-signal}), and $\omega_i \in [0,1]$ is human $i$'s anthropomorphism tendency.
\end{defn}

The attributed belief $\tilde{h}_i^{(2,j)} \in \Delta(S_i)$ represents ``what human $i$ believes AI $j$ expected human $i$ to do.'' This is the belief that triggers guilt or indignation when human $i$ disappoints the AI's perceived expectations.

The key distinction: \textbf{genuine beliefs} $h_i^{(2,k)}$ about other humans $k \in N_H$ are formed through observation and Bayesian updating; \textbf{attributed beliefs} $\tilde{h}_i^{(2,j)}$ about AI $j \in N_A$ are formed through psychological projection via the attribution function.

\subsection{The Attribution Function}

The attribution function is the novel primitive of ABE theory, formalising how humans project mental states onto AI agents.

\begin{defn}[Attribution Function]
\label{def:attribution-function}
For each human $i \in N_H$, the attribution function is a mapping
\begin{equation}
    \phi_i: \Theta_j \times X \times \Omega_i \to \Delta(S_i)
\end{equation}
that determines how human $i$ attributes expectations to AI $j$ based on AI's design parameters $\theta_j$, observable signals $x_j$, and the human's anthropomorphism tendency $\omega_i$.
\end{defn}

\begin{defn}[Zero-Anthropomorphism Benchmark]
\label{def:zero-anthro}
For human $i \in N_H$ interacting with AI $j \in N_A$, the \textbf{zero-anthropomorphism attributed belief} is:
\begin{equation}
    \tilde{h}_i^{(2,j),ZA} = \phi_i(\theta_j, x_j, 0)
\end{equation}
This represents the attributed belief when anthropomorphism is absent ($\omega_i = 0$). The zero-anthropomorphism benchmark is descriptive: it characterizes attribution without anthropomorphic bias, not a normative standard for ``correct'' beliefs.
\end{defn}

The attribution function formalises the empirically established process of mind perception \citep{gray2007dimensions,epley2007seeing}. Its inputs correspond to well-established antecedents of intentionality attribution \citep{wiese2017robots}. Attribution intensity varies with human-like cues including voice \citep{schroeder2016voice}, gaze behaviour, and movement patterns, all of which can be captured in the signal vector $x_j$.

Three main approaches to specifying $\phi_i$ are:
\begin{enumerate}
    \item \textbf{Behavioural attribution}: $\phi_i^{beh}(\theta_j, x_j, \omega_i) = g(s_j^{obs}, \omega_i)$, depending on AI's observed behaviour
    \item \textbf{Signal-based attribution}: $\phi_i^{sig}(\theta_j, x_j, \omega_i) = g(x_j, \omega_i)$, depending primarily on observable signals
    \item \textbf{Dispositional attribution}: $\phi_i^{disp}(\theta_j, x_j, \omega_i) = \omega_i \cdot \bar{h} + (1 - \omega_i) \cdot \underline{h}$, depending primarily on the human's anthropomorphism tendency
\end{enumerate}

\begin{example}[Linear Attribution]
\label{ex:linear-attribution}
Linearity is tractable: first-order approximations yield closed-form equilibria and testable predictions. Mind perception research finds monotonic relationships between anthropomorphism and attributed mental states---linearity captures this without untested curvature. We adopt:
\begin{equation}
    \tilde{h}_i^{(2,j)}(C) = \omega_i \cdot \left( \rho_j + \eta \cdot x_j \right)
\end{equation}
The parameter $\omega_i \in [0,1]$ measures individual anthropomorphism tendency and multiplies the entire expression: when $\omega_i = 0$, no beliefs are attributed regardless of AI characteristics, representing the limiting case of purely mechanistic perception. Inside the bracket, $\rho_j \in [0,1]$ captures AI prosociality embedded in design, while $\eta \cdot x_j$ (with signal strength $x_j \in [0,1]$ and sensitivity $\eta > 0$) captures presentation effects from anthropomorphic cues. These terms are additive because they represent distinct attribution channels---what the AI is designed to do versus how human-like it appears. Section~\ref{sec:empirical-implementation} derives testable predictions; Section~\ref{sec:mind-perception} grounds this specification in established mind perception theory.
\end{example}

% Specification Selection Table (from Agent 3.2)
\begin{table}[htbp]
\centering
\caption{Attribution Specification Selection Guide}
\label{tab:specification-selection}
\small
\begin{tabular}{@{}p{2.2cm}p{2.5cm}p{3.2cm}p{3cm}p{2.8cm}@{}}
\toprule
Specification & Observable Trigger & When to Use & Example Context & Identification \\
\midrule
Behavioral & AI's observed strategy $s_j^{\text{obs}}$ & Repeated interactions where AI actions inform beliefs & Multi-round trust games, learning experiments & Within-subject variation across rounds \\[6pt]
Signal-based & Interface signals $x_j$ & One-shot games with interface manipulation & Voice assistant design, chatbot framing & Between-subject treatment assignment \\[6pt]
Dispositional & Individual tendency $\omega_i$ & Fixed AI with focus on individual heterogeneity & Survey studies, field experiments & IDAQ instrument or similar scale \\
\bottomrule
\end{tabular}
\end{table}

The three specifications represent complementary modeling choices, not competing theories. Each captures a distinct source of variation in attributed beliefs.

Signal-based attribution suits one-shot experiments where researchers manipulate AI interface features---voice characteristics, avatar appearance, or conversational style. Random assignment to interface conditions provides clean between-subject identification. This specification isolates design effects from learning.

Behavioral attribution applies when humans interact with AI repeatedly and update attributed beliefs based on observed actions. Trust games with multiple rounds exemplify this setting: a player who observes cooperative AI behavior in early rounds may attribute stronger cooperative intent in later rounds. Within-subject variation across rounds identifies the attribution response $\partial \tilde{h} / \partial s_j^{\text{obs}}$.

Dispositional attribution fits studies where AI behavior remains fixed but individual differences in anthropomorphization drive outcome variation. Survey instruments like the IDAQ measure this tendency directly. Field experiments with standardized AI deployments fall in this category.

A practical diagnostic distinguishes behavioral from signal-based attribution: estimate whether $\partial \tilde{h} / \partial s_j^{\text{obs}} \neq 0$ conditional on interface signals. A significant effect indicates behavioral attribution operates; a null effect suggests signal-based attribution suffices.

\begin{remark}[Linear Specification Focus]
The linear specification in Example~\ref{ex:linear-attribution} serves as our primary working model for three reasons. First, it nests signal-based attribution as a leading special case: setting $\beta = 0$ yields attributed beliefs determined entirely by interface signals. Second, linearity delivers tractable equilibrium characterization while preserving the core insight that attributed beliefs depend on observable AI features. Third, empirical research on mind perception consistently finds monotonic relationships between anthropomorphic cues and attributed mental states \citep{waytz2010causes,epley2007seeing}. We derive general existence results for the broad class, then specialize to the linear case for applications.
\end{remark}

\subsubsection{Empirical Implementation and Testability}
\label{sec:empirical-implementation}

The attribution function $\phi_i$ is identified through experimental variation in its arguments. We describe the measurement strategy, illustrate with a trust game design, and derive testable predictions.

\paragraph{Empirical Grounding.}

The linear specification in Example~\ref{ex:linear-attribution} has both theoretical and empirical support. Meta-analyses of human-AI interaction confirm monotonic relationships between anthropomorphic cues and trust: \citet{blut2021understanding} find that across 97 effect sizes, anthropomorphism's effect on trust is consistent with linearity within observed ranges. Since trust toward AI requires attributing beliefs about the AI's expectations and intentions, this evidence supports the linear attribution specification. The multiplicative structure---anthropomorphism tendency $\omega_i$ scaling signal effects---matches IDAQ validation studies showing that individual differences moderate responses to anthropomorphic design \citep{waytz2010who}.

Mind perception research provides further grounding. \citet{gray2007dimensions} establish that agency attribution---the dimension most relevant to strategic interaction---increases continuously with behavioral cues rather than exhibiting threshold effects. The SEEK framework \citep{epley2007seeing} predicts that both dispositional ($\omega_i$) and situational ($x_j$) factors contribute additively to anthropomorphism, consistent with our additive-inside-multiplicative structure. While nonlinear specifications (logistic saturation, uncanny valley discontinuities) may better fit extreme ranges, linearity captures first-order effects and enables closed-form identification. We maintain linearity for tractability; Section~\ref{sec:conclusion} discusses nonlinear extensions.

\paragraph{Identification Strategy.}

To identify $\phi_i$, we measure attributed beliefs $\tilde{h}_i^{(2,j)}$ while varying AI signals $x_j$ and measuring individual anthropomorphism $\omega_i$. Three components enable identification.

\paragraph{Belief Elicitation.}

We elicit attributed beliefs using incentive-compatible mechanisms. Quadratic scoring rules \citep{gachter2010peer} reward accuracy: subjects report their belief that the AI expects cooperation, and payment increases quadratically in the accuracy of that belief. To address hedging---subjects adjusting stated beliefs to insure against game outcomes---we follow \citet{blanco2010belief} in paying either the belief elicitation task or the strategic game, never both.

The strategy method enables post-hoc rationalization: when subjects state beliefs after observing outcomes, they may rationalize choices rather than report true beliefs \citep{danz2022belief}. Three design features mitigate this concern. First, \emph{belief-first elicitation}: subjects report attributed beliefs before making strategic choices, preventing outcome-driven rationalization. Second, \emph{monotonicity tests}: if subjects exhibit non-monotonic beliefs across signal levels (e.g., attributing lower beliefs to more anthropomorphic interfaces), this signals measurement error or demand effects. Third, \emph{separate payment randomization}: following \citet{blanco2010belief}, we determine payment-relevant decisions after all responses are collected, eliminating hedging incentives at each elicitation stage.

\emph{Strategy method.} Complete mapping of the attribution function requires observing $\tilde{h}_i^{(2,j)}$ across the full domain of $(x_j, \omega_i)$. The strategy method \citep{brandts2011strategy} elicits beliefs for each possible signal level within-subject, recovering the full $\phi_i$ mapping without relying on between-subject variation alone.

\emph{Anthropomorphism measurement.} Individual anthropomorphism $\omega_i$ is measured using the Individual Differences in Anthropomorphism Questionnaire (IDAQ; \citealp{waytz2010who}). The IDAQ contains 15 items measuring attribution of mental states to non-human entities (e.g., ``To what extent does a robot have intentions?''). We use the technology subscale, which correlates with trust in automated systems \citep{waytz2014mind}.

\paragraph{Application: Trust Game with Interface Manipulation.}

We illustrate identification with a trust game where an AI trustor sends $x \in [0, E]$ to a human trustee who returns $y \in [0, 3x]$. Between-subject manipulation varies interface anthropomorphism across three treatments: robotic ($x_A = 0$), humanoid ($x_A = 1$), and control ($x_A = 0.5$), holding AI behavior constant. Subjects complete the IDAQ, observe the AI interface, report attributed beliefs via incentivized elicitation, and choose return amounts. Random payment selection (belief or choice) eliminates hedging.

\paragraph{Parameter Recovery.}

Under the linear attribution specification from Example 1:
\begin{equation}
    \tilde{h}_i^{(2,A)} = \omega_i \cdot \left( \rho_A + \eta \cdot x_A \right)
\end{equation}
treatment variation identifies $\eta$; $\rho_A$ is assumed known from AI design. The reduced-form regression is:
\begin{equation}
    \tilde{h}_{it}^{(2,A)} = \beta_0 + \beta_1 \omega_i + \beta_2 x_{At} + \beta_3 \omega_i \cdot x_{At} + \varepsilon_{it}
    \label{eq:attribution-regression}
\end{equation}
where $t \in \{R, H, C\}$ indexes treatments.

The theoretical model generates testable restrictions:
\begin{enumerate}
    \item $\beta_2 = 0$: Signal has no direct effect on attributed beliefs; the effect operates through anthropomorphism
    \item $\beta_3 > 0$: Anthropomorphism moderates signal effect; high-$\omega_i$ subjects respond more to interface manipulation
\end{enumerate}

\paragraph{Separating Attribution from Psychological Preferences.}

The attribution function $\phi_i$ and psychological preference parameters ($\gamma_i$ for guilt sensitivity, $\beta_i$ for indignation sensitivity) enter utilities jointly, creating an identification problem: observed behavior reflects both how beliefs are attributed and how they affect payoffs. We propose a two-stage identification strategy.

\emph{Stage 1: Estimate psychological parameters from human-human interactions.} In interactions between human players, belief attribution is standard---$\tilde{h}_i^{(2,j)} = h_i^{(2,j)}$---and the attribution function is the identity. This baseline allows estimation of psychological preference parameters using established methods \citep{bellemare2023measuring}. For guilt aversion, we estimate $\gamma_i$ from trustee behavior in human-human trust games; for indignation, from punishment decisions after human defection. The key identifying assumption is that $\gamma_i$ and $\beta_i$ are stable across human and AI counterparts.

\emph{Stage 2: Identify $\phi_i$ from human-AI interactions.} With psychological parameters estimated from Stage 1, human-AI interactions isolate the attribution function. The difference between observed behavior toward AI and predicted behavior (using $\gamma_i$ with truthful beliefs) identifies the attribution wedge. Formally, let $a_i^*(h)$ denote the optimal action given second-order belief $h$. The observed action $a_i^{AI}$ toward an AI with attributed belief $\tilde{h}_i^{(2,A)}$ satisfies:
\begin{equation}
    a_i^{AI} = a_i^*\left(\phi_i(x_A, \omega_i, \rho_A)\right)
\end{equation}
Inverting the best-response correspondence recovers $\tilde{h}_i^{(2,A)}$ and hence the attribution function.

Alternative approaches include instrumental variables: exogenous variation in interface design (satisfying exclusion from psychological preferences) instruments for attributed beliefs. \citet{bellemare2023measuring} develop related techniques for identifying belief-dependent preferences.

\paragraph{Identification with Opaque AI Objectives.}

The parameter recovery strategy assumes $\rho_A$---the AI's true prosociality---is known from design. This assumption fails when AI objectives are proprietary or emergent from training. We consider three approaches.

First, \emph{partial identification}: without point identification of $\rho_A$, we bound the attribution function. If $\rho_A \in [\underline{\rho}, \overline{\rho}]$, the estimated $\phi_i$ lies in a corresponding interval. The bounds tighten as the anthropomorphism range expands: variation in $\omega_i$ traces out the attribution function's shape even when the intercept is unknown.

Second, \emph{sensitivity analysis}: we estimate $\phi_i$ under alternative assumptions about $\rho_A$ and report which conclusions are robust. If the interaction effect $\beta_3 > 0$ holds across plausible $\rho_A$ values, the core prediction survives.

Third, \emph{direct elicitation}: subjects state beliefs about AI objectives. While potentially biased, elicited $\rho_A$ beliefs provide an alternative reference point. Comparing elicited beliefs to design-implied values reveals systematic misperception of AI objectives, a distinct phenomenon from attribution.

The interaction coefficient $\beta_3$ directly estimates the signal sensitivity $\eta$. Given $\rho_A$ known from design, the baseline effect $\beta_1$ identifies anthropomorphism scaling.

The key testable prediction is the interaction effect $\beta_3 > 0$. Under the null hypothesis that attributed beliefs do not depend on anthropomorphism ($\omega_i$ irrelevant), $\beta_3 = 0$ and interface manipulation affects all subjects equally. A positive interaction indicates that anthropomorphic signals operate through the attribution mechanism specified in the model. Alternative specifications (logistic, threshold) accommodate saturation and uncanny valley effects; we maintain linearity for tractability.

\paragraph{Falsification Criteria.}

The model generates sharp falsifiable predictions. Consider three possible empirical patterns.

If $\beta_3 = 0$ (no interaction), anthropomorphism does not moderate signal interpretation. This falsifies the core mechanism: attributed beliefs would respond to interface manipulation identically for high- and low-$\omega_i$ subjects. The attribution function would reduce to $\phi_i(x_j) = f(x_j)$, independent of individual anthropomorphism tendency---inconsistent with mind perception theory and IDAQ validation evidence.

If $\beta_2 \neq 0$ with $\beta_3 = 0$, interface signals affect attributed beliefs directly without anthropomorphism mediation. This pattern suggests alternative mechanisms: perhaps anthropomorphic cues serve as quality signals or attention directors rather than triggering belief attribution.

Both null findings would require theoretical revision. The model does not predict that anthropomorphism always matters; rather, it predicts a specific functional relationship. Rejecting this relationship advances understanding even if it refutes the proposed mechanism.

\subsubsection{Psychological Foundations: Mind Perception Theory}
\label{sec:mind-perception}

The attribution function $\phi_i$ in Definition~\ref{def:attribution-function} formalizes how humans project mental states onto AI agents. This formalization rests on mind perception psychology.

The SEEK framework \citep{epley2007seeing} identifies three determinants of anthropomorphism: dispositional factors (sociality, effectance) corresponding to $\omega_i$, and situational triggers (elicited agent knowledge) corresponding to observable signals $x_j$. Mind perception decomposes into agency and experience \citep{gray2007dimensions}. AI scores high on agency (triggering attribution) but low on experience (driving attenuation). These dimensions operate independently, explaining their coexistence.

Anthropomorphism operates as System~1 cognition: automatic and resistant to conscious override. System~2 correction is effortful and incomplete \citep{epley2007seeing}---even users who know AI lacks mental states exhibit residual attribution. This dual-process structure grounds our functional form: the linear specification reflects SEEK's prediction that attribution increases monotonically in both dispositional tendency ($\omega_i$) and situational triggers ($x_j$).

\subsection{Psychological Payoffs}

Two main psychological mechanisms are relevant: indignation and guilt.

\begin{defn}[Indignation with Attenuation]
\label{def:indignation}
The indignation payoff captures disutility from disappointing others' expectations:
\begin{equation}
    \psi_i^{IND}(s, h_i^{(2)}, \tilde{h}_i^{(2)}) = -\beta_i \cdot \mathbf{1}_{s_i = D} \cdot \left[ \sum_{k \in N_H} h_i^{(2,k)}(C) + \lambda_i^{IND} \sum_{j \in N_A} \tilde{h}_i^{(2,j)}(C) \right]
\end{equation}
where $\beta_i > 0$ is indignation sensitivity, $h_i^{(2,k)}(C)$ is the probability that human $i$ believes human $k$ expected cooperation, $\tilde{h}_i^{(2,j)}(C)$ is the attributed probability that AI $j$ expected cooperation, $\lambda_i^{IND} \in [0,1]$ is the indignation attenuation factor toward AI, and $\mathbf{1}_{s_i = D}$ is the indicator for defection (adapted to context-specific actions in applications).
\end{defn}

The attenuation of indignation toward AI reflects intent attribution. Indignation requires perceiving malicious intent, but humans perceive AI behaviour as data-driven rather than prejudice-driven \citep{bigman2023people}. Related belief-dependent emotions---frustration and anger---arise in sequential games when players respond to perceived blame \citep{battigalli2019frustration}.

\begin{defn}[Guilt Aversion with Attenuation]
\label{def:guilt}
The guilt payoff captures disutility from falling short of perceived obligations:
\begin{equation}
    \psi_i^{GUILT}(s, h_i^{(2)}, \tilde{h}_i^{(2)}) = -\gamma_i \cdot \left[ \sum_{k \in N_H} \max\{0, h_i^{(2,k)} - \pi_k(s)\} + \lambda_i^{GUILT} \sum_{j \in N_A} \max\{0, \tilde{h}_i^{(2,j)} - \pi_j(s)\} \right]
\end{equation}
where $\gamma_i > 0$ is guilt sensitivity and $\lambda_i^{GUILT} \in [0,1]$ is the guilt attenuation factor toward AI.
\end{defn}

\citet{demelo2017people} find that participants feel ``considerably less guilt'' when exploiting machines than humans, suggesting guilt requires attribution of experience (capacity to suffer) that humans do not readily grant to AI. Cross-cultural evidence reveals heterogeneity: Japanese participants exhibit guilt toward robots comparable to guilt toward humans, while Western participants show strong attenuation \citep{karpus2025cross}.

\subsection{Welfare Measures}
\label{sec:welfare-measures}

Two welfare concepts arise naturally from the payoff decomposition. We define each, establish normative foundations, and characterize divergence.

\begin{defn}[Material Welfare]
\label{def:material-welfare}
For strategy profile $s \in S$, \textbf{material welfare} is the sum of material payoffs:
\begin{equation}
    W(s) = \sum_{i \in N} \pi_i(s).
\end{equation}
\end{defn}

Material welfare counts only outcomes that are transferable, verifiable, and comparable across agents---monetary payoffs, consumption, or quantities.

\begin{defn}[Extended Welfare]
\label{def:extended-welfare}
For strategy profile $s \in S$ with belief profile $(h, \tilde{h})$, \textbf{extended welfare} is:
\begin{equation}
    W^{ext}(s, h, \tilde{h}) = W(s) + \sum_{i \in N_H} \psi_i(s, h_i^{(2)}, \tilde{h}_i^{(2)}).
\end{equation}
\end{defn}

Extended welfare includes the psychological payoffs $\psi_i$ that humans experience. AI agents contribute no psychological terms since they lack subjective experience.

\subsubsection{Normative Justification}
\label{sec:welfare-normative}

The choice between welfare measures reflects a prior normative position on the status of psychological payoffs.

\citet{kahneman1997back} distinguish \emph{decision utility}---revealed preference over choices---from \emph{experienced utility}---hedonic quality of outcomes. Material welfare $W$ aligns with decision utility: it counts what agents choose to maximize, abstracting from the subjective states accompanying those choices. Extended welfare $W^{ext}$ aligns with experienced utility: it incorporates hedonic content, including guilt and disappointment.

When should each apply? The answer depends on whether psychological payoffs have intrinsic or instrumental value.

\paragraph{Instrumental view.} If guilt and disappointment matter only because they distort decisions, then $W$ is the appropriate measure. Psychological disutility is a means to material ends---it motivates behavior but has no standing in the social objective. Under this view, a planner seeks to maximize material outcomes, treating psychological costs as constraints that shape equilibrium but not as ends in themselves.

\paragraph{Intrinsic view.} If subjective experience has inherent value, then $W^{ext}$ applies. Guilt reduces human well-being directly, not merely through its behavioral consequences. A planner who values experienced happiness includes $\psi_i$ in the objective.

Standard welfare economics adopts the instrumental view: consumer surplus and Pareto efficiency concern material allocations. Yet behavioral welfare economics increasingly recognizes that experienced utility may diverge from decision utility \citep{kahneman1997back}. In human-AI interaction, this divergence takes a specific form: humans experience guilt toward agents that cannot reciprocate the experience.

\subsubsection{When Measures Diverge}
\label{sec:welfare-divergence}

With prosocial AI ($\rho_j > 0$), the two measures typically agree. The AI's objective incorporates human welfare, so attributed beliefs reflect genuine concern. Guilt from disappointing a prosocial AI corresponds to disappointing an entity that valued cooperation---the psychological cost has a material counterpart in the AI's design.

With materialist AI ($\rho_j = 0$), the measures can diverge. The AI maximizes its own material payoff and holds no expectations about human cooperation. Yet humans may attribute such expectations through anthropomorphism. The resulting guilt is real---humans experience the disutility---but corresponds to no genuine expectation.

\begin{example}[Phantom Expectations]
\label{ex:phantom-expectations}
Consider human $i$ interacting with materialist AI $j$. The AI's objective is $U_j^A = \pi_j(s)$, containing no term that depends on $i$'s cooperation. The AI holds no expectation about $i$'s behavior.

If $\omega_i > 0$, human $i$ attributes beliefs via $\tilde{h}_i^{(2,j)} = \phi_i(\theta_j, x_j, \omega_i) > 0$. Human $i$ believes the AI expected cooperation. When $i$ defects, guilt follows:
\begin{equation}
    \psi_i^{GUILT} = -\gamma_i \lambda_i^{GUILT} \max\{0, \tilde{h}_i^{(2,j)} - \pi_j(s)\} < 0.
\end{equation}
This guilt is phantom: it responds to attributed expectations that exist only in the human's model of the AI. Material welfare $W$ ignores this cost. Extended welfare $W^{ext}$ counts it.
\end{example}

The divergence has policy implications. If a designer maximizes $W$, phantom guilt is irrelevant---only material outcomes matter. If the designer maximizes $W^{ext}$, phantom guilt is costly even when AI is indifferent. The optimal design of AI interfaces may differ across objectives.

\paragraph{Belief Revision.} Why don't humans eliminate phantom expectations by revising attributed beliefs? Three mechanisms prevent full revision: automaticity (attribution operates as System~1 cognition), limited correction capacity (System~2 override requires effort unavailable in interactive settings), and incomplete correction residue (even informed users exhibit residual anthropomorphism). Attributed beliefs reflect cognitive architecture, not irrationality.

\subsection{Game Definition}

\begin{defn}[Asymmetric Human-AI Psychological Game]
\label{def:game}
An asymmetric psychological game is a tuple
\begin{equation}
    \Gamma = (N_H, N_A, \{T_i\}_{i \in N_H}, \{\Theta_j\}_{j \in N_A}, \{S_i\}_{i \in N}, \{U_i^H\}_{i \in N_H}, \{U_j^A\}_{j \in N_A}, \phi, p)
\end{equation}
where $N_H, N_A$ are human and AI player sets, $T_i$ and $\Theta_j$ are type spaces, $S_i$ are strategy sets, $U_i^H$ and $U_j^A$ are utility functions, $\phi = \{\phi_i\}_{i \in N_H}$ are attribution functions, and $p$ is the common prior over types.
\end{defn}

\subsection{Assumptions}

We impose the following regularity conditions:

\begin{assumption}[Regularity]
\label{ass:regularity}
(A1) Strategy spaces $S_i$ are non-empty and finite. Type spaces $T_i$ and $\Theta_j$ are non-empty, convex, and compact. Payoff functions are continuous.
\end{assumption}

\begin{assumption}[Attribution Continuity]
\label{ass:attribution}
(A2) The attribution function $\phi_i: \Theta_j \times X \times \Omega_i \to \Delta(S_i)$ is continuous in $(\theta_j, x_j)$ for fixed $\omega_i$.
\end{assumption}

\begin{assumption}[Behavioral Attribution Regularity]
\label{ass:behavioral-attribution}
(A2-beh) When attribution depends on observed AI behavior, the function $\phi_i^{beh}: \Theta_j \times X \times \Delta(S_j) \times \Omega_i \to \Delta(S_i)$ satisfies:
\begin{enumerate}
    \item \textbf{Strategy continuity}: $\phi_i^{beh}$ is continuous in $\sigma_j \in \Delta(S_j)$ for fixed $(\theta_j, x_j, \omega_i)$.
    \item \textbf{Own-strategy independence}: $\phi_i^{beh}$ depends only on AI $j$'s strategy $\sigma_j$, not on human $i$'s strategy $\sigma_i$ or other players' strategies.
\end{enumerate}
\end{assumption}

\begin{remark}[Scope of Attribution Assumptions]
\label{rem:attribution-scope}
Assumption A2 applies to exogenous attribution (signal-based or dispositional), where attributed beliefs are constant in the strategy profile. Assumption A2-beh applies to behavioral attribution, where attributed beliefs depend on AI's observed behavior. The existence theorem (Theorem~\ref{thm:existence}) holds under either A2 alone (exogenous case) or A2 and A2-beh together (behavioral case). The applications in Section~\ref{sec:applications} use exogenous attribution.
\end{remark}

\begin{assumption}[Bounded Psychological Payoffs]
\label{ass:bounded}
(A3) There exists $M < \infty$ such that $|\psi_i(s, h_i^{(2)}, \tilde{h}_i^{(2)})| \leq M$ for all $i \in N_H$, all $s \in S$, and all beliefs.
\end{assumption}

\begin{remark}[Assumption A3 Redundancy]
Under A1--A2, boundedness of psychological payoffs follows from continuity on compact domains. We state A3 explicitly for clarity in the existence proof.
\end{remark}

\subsection{Assumption Summary}
\label{sec:assumption-summary}

Table~\ref{tab:assumptions} collects all assumptions. Core assumptions (A1)--(A3) support existence. Optional extensions and context-specific conditions apply to individual results.

\begin{table}[htbp]
\centering
\caption{Summary of Assumptions}
\label{tab:assumptions}
\small
\begin{tabular}{@{}llp{7.5cm}l@{}}
\toprule
\textbf{Label} & \textbf{Name} & \textbf{Statement} & \textbf{Required for} \\
\midrule
\multicolumn{4}{@{}l}{\textit{Core Assumptions}} \\[0.5ex]
(A1) & Regularity & Strategy spaces $S_i$ are finite; type spaces $T_i$, $\Theta_j$ are compact and convex; payoffs are continuous & All results \\[0.5ex]
(A2) & Attribution Continuity & $\phi_i: \Theta_j \times X \times \Omega_i \to \Delta(S_i)$ is continuous in $(\theta_j, x_j)$ & Exogenous attrib.\ results \\[0.5ex]
(A2-beh) & Behavioral Attrib.\ Regularity & $\phi_i^{beh}$ continuous in $\sigma_j$; depends only on $\sigma_j$ & Behavioral attrib.\ results \\[0.5ex]
(A3) & Bounded Psychological Payoffs & $|\psi_i(s, h_i^{(2)}, \tilde{h}_i^{(2)})| \leq M$ for some $M < \infty$ & All results \\[1ex]
\midrule
\multicolumn{4}{@{}l}{\textit{Optional Extensions to (A2)}} \\[0.5ex]
(A2$'$) & Attribution Monotonicity & $\omega_i' > \omega_i \Rightarrow \tilde{h}_i^{(2,j)}(\omega_i') \geq \tilde{h}_i^{(2,j)}(\omega_i)$ & Props.\ 5--10$'$, Cors.\ 2--3 \\[0.5ex]
(A2$''$) & Attribution Non-Degeneracy & $\phi_i$ is not constant in $\omega_i$ when $\bar{h}_H > \underline{h}$ & Prop.\ 9$'$ \\[0.5ex]
(A2$'''$) & Signal Monotonicity & $\partial \tilde{h}_i^{(2,j)} / \partial x_j \geq 0$ for $\omega_i > 0$ & Props.\ 10, 10$'$, Cors.\ 2--3 \\[1ex]
\midrule
\multicolumn{4}{@{}l}{\textit{Context-Specific Conditions}} \\[0.5ex]
(G) & Guilt Dominance & $G \equiv \gamma_H \lambda_H^{GUILT} > 1$ & Props.\ 5, 8 \\[0.5ex]
(G$'$) & Positive Guilt Sensitivity & $G \equiv \gamma_H \lambda_H^{GUILT} > 0$ & Props.\ 9$'$, 10, 10$'$ \\[0.5ex]
(I) & Indignation Dominance & $\beta_i[(n_H - 1) + \lambda_i^{IND} n_A] > E(1 - m/n)$ & Props.\ 6, 8, 10, 10$'$ \\[0.5ex]
(E) & Cooperation Efficiency & $m > 1$ & Props.\ 8, 9, 9$'$, 10, 10$'$ \\[0.5ex]
(C) & Signal Clarity & $x_A > 0.5$ & Prop.\ 7 \\[0.5ex]
(T) & Temptation Dominance & $E(1 - m/n) > \beta_i(n_H - 1)$ & Props.\ 10, 10$'$ \\
\bottomrule
\end{tabular}
\end{table}

% Section 3: Equilibrium
\section{Attributed Belief Equilibrium}
\label{sec:equilibrium}

This section defines Attributed Belief Equilibrium (ABE) and establishes its existence under the regularity conditions introduced above.

\subsection{Definition}

\begin{defn}[Attributed Belief Equilibrium]
\label{def:ABE}
A strategy profile $s^* = (s_H^*, s_A^*)$, genuine belief system $h^*$, and attributed belief system $\tilde{h}^*$ constitute an \textbf{Attributed Belief Equilibrium} if:
\begin{enumerate}
    \item \textbf{Human Optimality}: For all $i \in N_H$,
    \begin{equation}
        s_i^* \in \arg\max_{s_i \in S_i} U_i^H(s_i, s_{-i}^*, h_i^*, \tilde{h}_i^*)
    \end{equation}

    \item \textbf{AI Optimality}: For all $j \in N_A$,
    \begin{equation}
        s_j^* \in \arg\max_{s_j \in S_j} U_j^A(s_j, s_{-j}^*; \theta_j)
    \end{equation}

    \item \textbf{Genuine Belief Consistency}: For all $i, k \in N_H$,
    \begin{align}
        h_i^{*(1,k)} &= s_k^* \quad \text{(first-order consistency)} \\
        h_i^{*(2,k)} &= h_k^{*(1,i)} \quad \text{(second-order consistency)}
    \end{align}

    \item \textbf{Attribution Consistency}: For all $i \in N_H$ and $j \in N_A$,
    \begin{equation}
        \tilde{h}_i^{*(2,j)} = \phi_i(\theta_j, x_j, \omega_i)
    \end{equation}
\end{enumerate}
\end{defn}

The definition captures two distinct consistency requirements. Genuine beliefs about other humans satisfy standard Bayesian consistency as in \citet{battigalli2009dynamic}. Attributed beliefs about AI satisfy attribution consistency: they are determined by the attribution function given AI characteristics and the human's anthropomorphism tendency.

\begin{remark}[Mixed Strategies]
The definition extends naturally to mixed strategies by replacing pure strategies $s_i$ with mixed strategies $\sigma_i \in \Delta(S_i)$ and taking expectations over strategy profiles.
\end{remark}

Three key features distinguish ABE from standard psychological game equilibria:
\begin{enumerate}
    \item \textbf{Dual consistency}: Genuine beliefs (H-H) satisfy Bayesian consistency; attributed beliefs (H-A) satisfy attribution consistency.
    \item \textbf{Asymmetric rationality}: Humans are psychologically rational; AI are design-rational.
    \item \textbf{No cognizability for attributed beliefs}: Unlike \citet{battigalli2009dynamic}, attributed beliefs need not be ``cognizable'' since AI lacks genuine beliefs.
\end{enumerate}

\subsection{Existence}

\begin{theorem}[Existence of ABE]
\label{thm:existence}
Under Assumptions A1--A3, an Attributed Belief Equilibrium exists.
\end{theorem}

\begin{proof}[Proof Sketch]
The proof extends the fixed-point approach of \citet{battigalli2009dynamic} to dual belief structures.

\textbf{Step 1: Construct the relevant spaces.} Define the space of strategy profiles $\Sigma = \prod_{i \in N} \Delta(S_i)$, which is non-empty, compact, and convex by A1. For each human $i$, the attributed belief is pinned down by $\tilde{h}_i^{(2,j)} = \phi_i(\theta_j, x_j, \omega_i)$, which is well-defined and continuous by A2.

\textbf{Step 2: Define the best-response correspondence.} For humans, given beliefs $(h_i, \tilde{h}_i)$, define
\[
    BR_i^H(h_i, \tilde{h}_i) = \arg\max_{\sigma_i \in \Delta(S_i)} \mathbb{E}[U_i^H(\sigma_i, \sigma_{-i}, h_i, \tilde{h}_i)].
\]
For AI, define
\[
    BR_j^A(\sigma_{-j}) = \arg\max_{\sigma_j \in \Delta(S_j)} \mathbb{E}[U_j^A(\sigma_j, \sigma_{-j}; \theta_j)].
\]
By A1 (compactness and continuity), these correspondences are non-empty, convex-valued, and upper hemicontinuous.

\textbf{Step 3: Construct the fixed-point mapping.} Define $\Phi: \Sigma \to \Sigma$ as follows. Given strategy profile $\sigma$:
\begin{enumerate}
    \item Compute attributed beliefs: $\tilde{h}_i^{(2,j)} = \phi_i(\theta_j, x_j, \omega_i)$
    \item Compute genuine belief consistency: $h_i^{(1,k)} = \sigma_k$ and $h_i^{(2,k)} = h_k^{(1,i)}$
    \item Apply best responses: $\Phi(\sigma) = BR^H(h, \tilde{h}) \times BR^A(\sigma)$
\end{enumerate}
By Kakutani's fixed-point theorem, $\Phi$ has a fixed point $\sigma^*$.

\textbf{Step 4: Verify equilibrium conditions.} At the fixed point, human optimality holds by construction of $BR_i^H$, AI optimality holds by construction of $BR_j^A$, genuine belief consistency holds by the construction in Step 3, and attribution consistency holds by definition of $\tilde{h}_i^{(2,j)}$.
\end{proof}

\begin{remark}[Role of Bounded Psychological Payoffs]
Assumption A3 ensures that the best-response correspondences are well-behaved. Without boundedness, psychological payoffs could dominate material payoffs arbitrarily, potentially leading to non-existence.
\end{remark}

\subsection{Special Cases and Nesting}

\begin{proposition}[Reduction to Standard PGT]
\label{prop:nesting}
When $N_A = \emptyset$ (no AI agents), ABE reduces to Psychological Nash Equilibrium (PNE) as defined in \citet{geanakoplos1989psychological}, which coincides with Sequential Psychological Equilibrium (SPE) of \citet{battigalli2009dynamic} for static games.
\end{proposition}

The proof shows that when $N_A = \emptyset$, conditions ABE2 and ABE4 are vacuously satisfied, psychological payoffs simplify to depend only on genuine beliefs, and ABE1/ABE3 reduce exactly to PNE optimality and belief consistency. See Appendix~\ref{app:proofs}.

\begin{proposition}[Reduction to Nash]
\label{prop:nash}
When $\psi_i \equiv 0$ for all $i \in N_H$ (no psychological payoffs), ABE strategy profiles coincide with Nash equilibria of the material game $\Gamma^M = (N, \{S_i\}, \{\pi_i\}_{i \in N_H}, \{U_j^A(\cdot; \theta_j)\}_{j \in N_A})$.
\end{proposition}

The proof establishes a bijection: when $\psi_i \equiv 0$, human utility reduces to $U_i^H = \pi_i$, so beliefs become ``strategically irrelevant'' and the optimality conditions reduce to Nash best responses. See Appendix~\ref{app:proofs}.

\begin{proposition}[Rational Attribution and Bayesian Game Equivalence]
\label{prop:rational-attribution}
A strategy profile $\sigma^*$ together with attribution function $\phi$ satisfy \textbf{rational attribution} if: (i) $\phi_i(\theta_j, x_j, \omega_i) = \sigma_i^*$ for all $i \in N_H, j \in N_A$ (attribution projects equilibrium), and (ii) $\sigma^*$ satisfies ABE optimality given $\phi$ (strategies are optimal). Under rational attribution, ABE strategy profiles correspond bijectively to Bayes-Nash equilibria of an equivalent Bayesian game $\Gamma^B$ with type uncertainty about AI design parameters.
\end{proposition}

Rational attribution is a fixed-point requirement: $\phi$ projects equilibrium play, and $\sigma^*$ is an equilibrium given $\phi$. Existence is not guaranteed for general games but holds in games with unique Nash equilibria. The proposition identifies when ABE reduces to standard Bayesian game theory---precisely when humans form ``correct'' beliefs about AI expectations. See Appendix~\ref{app:proofs} for the complete proof and construction of the equivalent Bayesian game $\Gamma^B$.

\begin{defn}[Rational Attribution as a Fixed Point]
\label{def:rational-attribution}
Fix a psychological game $\Gamma$. A strategy profile $\sigma^* \in \prod_{k \in N} \Delta(S_k)$ together with attribution function $\phi$ satisfy \textbf{rational attribution} if the following fixed-point condition holds:
\begin{enumerate}
    \item[(i)] \textbf{Attribution projects equilibrium}: For all $i \in N_H$ and $j \in N_A$,
    \begin{equation}
        \phi_i(\theta_j, x_j, \omega_i) = \sigma_i^*
    \end{equation}
    That is, the attributed belief about ``what AI expects human $i$ to do'' equals human $i$'s actual equilibrium strategy.

    \item[(ii)] \textbf{Strategies are optimal}: Given the beliefs determined by $\phi$, the profile $\sigma^*$ satisfies ABE optimality:
    \begin{itemize}
        \item For all $i \in N_H$: $\sigma_i^* \in \arg\max_{\sigma_i} \mathbb{E}[U_i^H(\sigma_i, \sigma_{-i}^*, h_i(\sigma^*), \tilde{h}_i(\phi))]$
        \item For all $j \in N_A$: $\sigma_j^* \in \arg\max_{\sigma_j} U_j^A(\sigma_j, \sigma_{-j}^*; \theta_j)$
    \end{itemize}
\end{enumerate}
The pair $(\sigma^*, \phi)$ is mutually consistent: $\phi$ projects the equilibrium play, and $\sigma^*$ is indeed an equilibrium given $\phi$.
\end{defn}

Rational attribution captures the case where humans correctly anticipate equilibrium behavior: the attributed belief ``what AI expects me to do'' coincides with what the human actually does in equilibrium. This is the ``rational'' benchmark: no systematic bias in belief formation.

\begin{corollary}[Complete Information Reduction]
\label{cor:complete-info}
When types are common knowledge (degenerate prior $p$), rational attribution ABE reduces to Nash equilibrium of a game with type-dependent payoffs.
\end{corollary}

\begin{remark}[Economic Interpretation]
Proposition~\ref{prop:rational-attribution} identifies when the ABE framework reduces to standard Bayesian game theory:
\begin{enumerate}
    \item \textbf{Rational attribution as a benchmark.} When humans form beliefs about AI expectations ``correctly''---attributing to AI exactly the expectations consistent with equilibrium---the psychological game reduces to a standard game of incomplete information.

    \item \textbf{Departures from rationality.} The ABE framework differs non-trivially from Bayesian games precisely when attribution is \emph{not} rational: when $\phi_i(\theta_j, x_j, \omega_i) \neq \sigma_i^*$. Such departures arise from anthropomorphic bias, signal effects, or systematic misattribution.

    \item \textbf{Design implications.} If AI designers want outcomes equivalent to the rational Bayesian benchmark, they should design AI interfaces and behaviors that induce rational attribution. Conversely, strategic manipulation of attribution can shift outcomes away from the Bayesian benchmark.
\end{enumerate}
\end{remark}

\begin{remark}[Equilibrium Multiplicity under Rational Attribution]
If the underlying game has multiple equilibria, the rational attribution condition can hold for at most one equilibrium per attribution function $\phi$. This is because $\phi_i(\theta_j, x_j, \omega_i)$ is a deterministic function of its arguments---it cannot output different values for different equilibria. The proposition establishes correspondence for each equilibrium satisfying the fixed-point condition separately.
\end{remark}

\subsection{Multiplicity}

Different attribution functions can sustain different equilibria in the same material game.

\begin{proposition}[Attribution-Dependent Multiplicity]
\label{prop:multiplicity}
Consider a psychological game $\Gamma$ with at least one human and one AI. Suppose for some human $i \in N_H$ and AI $j \in N_A$:
\begin{enumerate}
    \item[(i)] \textbf{Belief-dependent payoffs}: $\partial \psi_i / \partial \tilde{h}_i^{(2,j)} \neq 0$ for some strategy profile.
    \item[(ii)] \textbf{Distinct attributions}: $\phi_i(\theta_j, x_j, \omega_i) \neq \phi'_i(\theta_j, x_j, \omega_i)$ for some configuration.
    \item[(iii)] \textbf{Best-response separation}: The change in attributed beliefs shifts equilibrium strategies.
\end{enumerate}
Then $\Gamma$ admits ABE under $\phi$ and $\phi'$ with distinct equilibrium strategy profiles: $s^*(\phi) \neq s^*(\phi')$.
\end{proposition}

\begin{remark}[Comparison with Standard PGT Multiplicity]
In standard psychological game theory, multiplicity arises from feedback between equilibrium strategies and belief consistency. In ABE, a distinct source emerges: attributed beliefs are exogenously fixed by $\phi$, so changing $\phi$ directly changes utility and thereby equilibrium. This feed-forward structure---$\phi \to \tilde{h}^{(2)} \to U^H \to s^*$---makes ABE multiplicity conceptually cleaner.
\end{remark}

\begin{remark}[Genericity]
Conditions (i)--(iii) are generic. Condition (i) fails only when $\psi_i$ is independent of attributed beliefs. Condition (ii) fails only when all attribution functions coincide. Condition (iii) fails only when payoff changes leave best responses unchanged. Attribution-dependent multiplicity is the typical outcome, not an exceptional one.
\end{remark}

This multiplicity has important design implications. Interface design, framing, and behavioural presentation affect attribution patterns and thereby equilibrium selection---even with fixed material payoffs. A prosocial AI signalling expectations through humanlike cues induces more cooperation than an equally prosocial AI with a mechanical interface.

\subsubsection*{Illustrative Examples}

We provide three numerical examples demonstrating the proposition.

\paragraph{Example 1: Trust Game with Guilt.}

Consider a trust game with AI trustor (endowment $E = 10$, prosociality $\rho_A = 0.3$) and human trustee (guilt sensitivity $\gamma_H = 2$, attenuation $\lambda_H^{GUILT} = 0.5$, anthropomorphism $\omega_H = 0.8$). AI sends $x = 10$; human returns $y \in [0, 30]$. Human utility: $U_H = 30 - y - \max\{0, \tilde{h}_H^{(2,A)} - y\}$.

\emph{Attribution $\phi$ (anthropomorphic interface)}: $\tilde{h}_H^{(2,A)} = 0.8(0.3 \times 30 + 0.5 \times 20) = 15.2$.

\emph{Attribution $\phi'$ (mechanical interface)}: $\tilde{h}_H^{(2,A)} = 0.8 \times 0.3 \times 30 = 7.2$.

Equilibrium returns: $y^*(\phi) = 15.2$, $y^*(\phi') = 7.2$. The anthropomorphic interface doubles AI's payoff (from 7.2 to 15.2) through elevated attributed expectations.

\paragraph{Example 2: Public Goods with Indignation.}

Consider a public goods game: one human, one AI, endowment $E = 10$, multiplier $m = 1.5$, binary contributions $\{0, 10\}$. AI contributes $c_A = 10$. Human has $\beta_H = 8$, $\lambda_H^{IND} = 0.5$. Material payoffs: defection yields 17.5, cooperation yields 15.

\emph{Attribution $\phi$ (high)}: $\tilde{h}_H^{(2,A)}(C) = 0.64$. Indignation cost from defection: $8 \times 0.5 \times 0.64 = 2.56 > 2.5$ (material gain). Human cooperates.

\emph{Attribution $\phi'$ (low)}: $\tilde{h}_H^{(2,A)}(C) = 0.16$. Indignation cost: $8 \times 0.5 \times 0.16 = 0.64 < 2.5$. Human defects.

Equilibrium contributions: $c_H^*(\phi) = 10$, $c_H^*(\phi') = 0$.

\paragraph{Example 3: Coordination Game.}

One human, one AI, choose technology $\{A, B\}$. Coordination yields payoff 2; miscoordination yields 0. AI plays $A$. Human has expectation conformity parameter $\beta_H = 3$.

\emph{Attribution $\phi$}: Human believes AI expects $A$ with probability 0.9. Utility: $U_H(A) = 2$, $U_H(B) = -2.7$. Human plays $A$.

\emph{Attribution $\phi'$}: Human believes AI expects $B$ with probability 0.9. Utility: $U_H(A) = -0.7$, $U_H(B) = 0$. Human plays $B$.

Under $\phi$: ABE is $(A, A)$ with payoffs $(2, 2)$. Under $\phi'$: ABE is $(B, A)$ with payoffs $(0, 0)$.

The attribution function serves as an equilibrium selection device, determining whether coordination succeeds.

% Section 4: Applications
\section{Applications}
\label{sec:applications}

We apply the ABE framework to three canonical games: the trust game, public goods provision, and coordination. These applications illustrate how attributed beliefs shape strategic outcomes in human-AI interaction.

\begin{remark}[Notation Convention]
In applications with a single human and single AI, we use subscripts $H$ and $A$ instead of indices $i$ and $j$ for clarity.
\end{remark}

\subsection{Trust Game}

Consider a trust game with an AI trustor and a human trustee. The AI trustor has endowment $E$ and can send any amount $x \in [0, E]$ to the human trustee. The sent amount is tripled, so the trustee receives $3x$. The trustee then returns any amount $y \in [0, 3x]$ to the AI.

Material payoffs are $\pi_A(x,y) = E - x + y$ for the AI and $\pi_H(x,y) = 3x - y$ for the human. The AI is designed with prosociality parameter $\rho_A$, yielding utility
\[
    U_A(x, y; \rho_A) = (1 - \rho_A)(E - x + y) + \rho_A(E + 2x).
\]
A prosocial AI ($\rho_A > 0$) sends more because it values total surplus.

The human trustee experiences guilt from disappointing the AI's perceived expectations:
\[
    U_H(y; \tilde{h}_H) = 3x - y - \gamma_H \cdot \lambda_H^{GUILT} \cdot \max\{0, \tilde{h}_H^{(2,A)} - y\},
\]
where $\tilde{h}_H^{(2,A)} = \phi_H(\rho_A, x_A, \omega_H)$ is the attributed belief about what the AI ``expected'' to receive back.

\begin{proposition}[Trust Game ABE]
\label{prop:trust}
In the trust game with AI trustor and human trustee, suppose (A2') the attribution function $\phi_H$ is weakly increasing in $\omega_H$, and (G) guilt dominance holds: $\gamma_H \lambda_H^{GUILT} > 1$. Then:
\begin{enumerate}
    \item Higher anthropomorphism $\omega_H$ increases attributed expectations $\tilde{h}_H^{(2,A)}$.
    \item Higher attributed expectations increase equilibrium returns: $y^* = \min\{\tilde{h}_H^{(2,A)}, 3x\}$.
    \item The same material payoffs support different equilibrium outcomes depending on anthropomorphism.
\end{enumerate}
\end{proposition}

The result is consistent with evidence that betrayal aversion---the reluctance to trust when betrayal is possible---is absent when the partner is a computer \citep{aimone2014neural}. When humans do not anthropomorphise AI ($\omega_H \approx 0$), attributed expectations are low, guilt is minimal, and returns approach the materialist optimum of zero. When humans anthropomorphise AI ($\omega_H$ high), attributed expectations rise, guilt becomes salient, and returns increase.

\subsection{Public Goods Game}

Consider a public goods game with $n_H \geq 2$ humans and $n_A \geq 0$ AI agents. Each player has endowment $E$ and makes a binary contribution $c_i \in \{0, E\}$---either contributing the full endowment or nothing. The public good is multiplied by $m > 1$ (with $m < n$ for the social dilemma) and shared equally:
\[
    \pi_i(c) = E - c_i + \frac{m}{n} \sum_{k \in N} c_k.
\]

Humans experience indignation from defecting (contributing less than the cooperative norm $c^* = E$):
\[
    \psi_i^{IND} = -\beta_i \cdot \mathbf{1}_{c_i = 0} \cdot \left[ \sum_{k \in N_H \setminus \{i\}} h_i^{(2,k)}(E) + \lambda_i^{IND} \sum_{j \in N_A} \tilde{h}_i^{(2,j)}(E) \right],
\]
where $\tilde{h}_i^{(2,j)}(E) = \omega_i \cdot \bar{h}(c_A)$ is human $i$'s attributed belief that AI $j$ expected cooperation, with $\bar{h}(E) > \bar{h}(0)$ reflecting that cooperating AI induces higher attributed expectations.

\begin{proposition}[Public Goods ABE]
\label{prop:public-goods}
Consider the public goods game above. Suppose (A2') attribution monotonicity holds, and define \textbf{(I) indignation dominance}: $\beta_i[(n_H - 1) + \lambda_i^{IND} n_A] > E(1 - m/n)$ for all $i \in N_H$. Then:
\begin{enumerate}
    \item If AI contribute $c_A = E$ and $\omega_i \geq \bar{\omega}_i$ for all humans (where $\bar{\omega}_i \in [0,1]$ under (I)), there exists a symmetric ABE with $c_H^* = E$.
    \item If AI contribute $c_A = 0$ and $\omega_i$ is sufficiently low for all humans, the unique symmetric ABE has $c_H^* = 0$.
    \item Holding $n_H$ fixed, increasing the AI share $n_A/n$ affects equilibrium through two channels: a material channel (reducing MPCR, increasing defection temptation) and a psychological channel (increasing attributed expectations, increasing defection cost). The net effect depends on $\lambda_i^{IND}$.
\end{enumerate}
\end{proposition}

The AI population share matters for two reasons. Materially, adding AI while holding human count fixed dilutes the MPCR ($m/n$ falls), strengthening the free-rider problem. Psychologically, if humans attribute expectations to AI, more AI agents means more attributed expectations, which increases the psychological cost of defection.

The attenuation factor $\lambda_i^{IND}$ determines which channel dominates. If indignation toward AI is strong ($\lambda_i^{IND} \approx 1$), the psychological channel can overcome material temptation, sustaining cooperation. If indignation toward AI is attenuated ($\lambda_i^{IND} \approx 0$), AI expectations do not trigger indignation, and the material channel dominates---adding AI may paradoxically undermine cooperation.

\subsection{Coordination Game}

Consider a coordination game where players choose between two technologies, $A$ and $B$. Payoffs are higher when players coordinate:
\[
    \pi_i(s) = \begin{cases}
        2 & \text{if all players choose the same technology} \\
        0 & \text{otherwise}
    \end{cases}
\]

Suppose humans experience expectation conformity: deviating from others' perceived expectations creates disutility. For a 2-player game (one human $H$, one AI $A$), the human's psychological payoff is:
\[
    \psi_H(s_H) = -\beta_H \cdot \lambda_H^{EC} \cdot \sum_{s' \neq s_H} \tilde{h}_H^{(2,A)}(s'),
\]
where $\tilde{h}_H^{(2,A)}(s')$ is the attributed probability that AI expected human to play $s'$, $\beta_H > 0$ is expectation conformity sensitivity, and $\lambda_H^{EC} \in [0,1]$ is the attenuation factor toward AI.

When AI plays $A$ and signals with clarity $x_A \in [0,1]$, the attribution function yields:
\[
    \tilde{h}_H^{(2,A)}(A) = \omega_H \cdot x_A, \quad \tilde{h}_H^{(2,A)}(B) = \omega_H \cdot (1 - x_A).
\]

\begin{proposition}[Coordination ABE]
\label{prop:coordination}
In the coordination game with AI designed to play $A$ (commitment $\theta_A > 0$), suppose (A2') the attribution function is weakly increasing in $\omega_H$, and (C) signal clarity satisfies $x_A > 0.5$. Then:
\begin{enumerate}
    \item Attributed beliefs favour $A$: $\tilde{h}_H^{(2,A)}(A) > \tilde{h}_H^{(2,A)}(B)$.
    \item High anthropomorphism amplifies the psychological pull toward the AI-favoured equilibrium: $\partial \Delta U_H / \partial \omega_H > 0$.
    \item AI serves as focal point provider: $(A, A)$ is the unique ABE, resolving the multiplicity of the material game.
\end{enumerate}
\end{proposition}

This has practical implications for AI-assisted coordination. When humans need to coordinate but face multiple equilibria, AI agents can help by signalling a focal point. The effectiveness depends on anthropomorphism: humans who attribute expectations to AI experience psychological pressure to conform to those expectations, facilitating coordination.

\subsection{Design Implications}

The applications yield principles for AI design in strategic settings.

\paragraph{Interface determines equilibrium.}
Anthropomorphic design features---voice, naming, emotional expression---are not window dressing. They enter equilibrium through the attribution function $\phi_i(\theta_j, x_j, \omega_i)$: higher anthropomorphism $\omega_i$ elevates attributed beliefs $\tilde{h}_i^{(2,j)}$, which in turn affect psychological payoffs and optimal strategies. The trust game (Proposition~\ref{prop:trust}) shows that identical material payoffs support different equilibrium returns depending solely on how anthropomorphic the AI appears.

\paragraph{Match presentation to objectives.}
When AI is prosocially designed ($\rho_A > 0$), anthropomorphic presentation amplifies cooperation. Elevated attributed expectations make guilt and indignation salient, inducing humans to reciprocate trust and contribute to public goods. But when AI is materialistically designed ($\rho_A = 0$), anthropomorphic presentation creates phantom expectations---humans feel guilty for disappointing agents that neither expect nor care. Mechanical presentation avoids this welfare-reducing mismatch.

\paragraph{Attenuation cuts both ways.}
The attenuation factors $\lambda^{GUILT}$ and $\lambda^{IND}$ reduce the psychological weight of AI expectations relative to human expectations. This protects humans from phantom expectations when AI is materialistic, but it also limits the cooperation-inducing power of prosocial AI. The public goods analysis (Proposition~\ref{prop:public-goods}) reveals that when $\lambda^{IND}$ is low, increasing the AI share may paradoxically \emph{undermine} cooperation: the material channel (diluted marginal returns) dominates the psychological channel (attributed expectations). Optimal attenuation depends on whether AI in the environment is predominantly prosocial or materialistic---a question with cross-cultural dimensions given documented variation in moral emotion attenuation toward machines \citep{karpus2025cross}.

\paragraph{AI as coordination device.}
Beyond dyadic interactions, AI can serve as a \emph{constructed focal point} in coordination problems. Unlike spontaneous Schelling focal points, this mechanism operates through design commitment, clear signalling ($x_A > 0.5$), and attribution: humans who attribute expectations to AI experience psychological pressure to conform. The coordination game (Proposition~\ref{prop:coordination}) shows that AI can resolve equilibrium multiplicity, selecting the efficient outcome when material incentives alone leave coordination indeterminate.

\medskip

These principles raise welfare questions. When does anthropomorphism help, and when does it harm? Section~\ref{sec:welfare} formalizes the tradeoffs, showing that over-anthropomorphism with materialistic AI is the core design failure to avoid.

% Section 5: Welfare
\section{Welfare and Optimal AI Design}
\label{sec:welfare}

This section examines the welfare implications of ABE and derives principles for optimal AI design.

\subsection{Welfare Measures}

Define social welfare as the sum of material payoffs:
\[
    W(s) = \sum_{i \in N} \pi_i(s).
\]
This measure focuses on material outcomes, treating psychological payoffs as instrumental---they affect behaviour but are not valued directly for welfare purposes.

An alternative is to include human psychological welfare:
\[
    W^{ext}(s, h, \tilde{h}) = \sum_{i \in N} \pi_i(s) + \sum_{i \in N_H} \psi_i(s, h_i^{(2)}, \tilde{h}_i^{(2)}).
\]
This extended measure values psychological states intrinsically.

\subsection{Anthropomorphism and Welfare}

The welfare effects of anthropomorphism depend critically on AI design.

\begin{proposition}[Welfare Effects of Anthropomorphism]
\label{prop:welfare-anthro}
Consider the public goods game (Part 1) or trust game (Part 2). Under Assumptions (A1)--(A3) and (A2') attribution monotonicity:
\begin{enumerate}
    \item When AI is prosocially designed ($\rho_A > 0$), if (E) cooperation is efficient ($m > 1$) and (I) indignation dominance holds, then higher anthropomorphism $\omega$ weakly increases material welfare $W(s^*)$.
    \item When AI is materialistically designed ($\rho_A = 0$), if (G) guilt dominance holds ($G = \gamma_H \lambda_H^{GUILT} > 1$), then higher anthropomorphism $\omega$ may reduce extended welfare $W^{ext}(s^*, h^*, \tilde{h}^*)$.
\end{enumerate}
\end{proposition}

\begin{proof}[Proof Sketch]
Part 1: With prosocial AI, attributed expectations favour cooperation. Higher $\omega$ elevates attributed expectations, increasing the psychological cost of defection. This induces more cooperation, which raises total material payoffs when $m > 1$.

Part 2: With materialist AI, the AI has no prosocial expectations. But if humans over-anthropomorphise, they attribute \emph{phantom expectations} that do not exist. When these phantom expectations exceed feasible returns, humans incur guilt---a pure welfare loss that benefits no one.
\end{proof}

\begin{example}[Phantom Expectations in the Trust Game]
\label{ex:phantom}
Consider the trust game with $E = 10$, $x = 10$, and guilt parameter $G = 1.5$. With materialist AI ($\rho_A = 0$) and attribution function $\phi_H(0, x, \omega_H) = 5\omega_H x$:

\begin{center}
\begin{tabular}{lcc}
\hline
& $\omega_H = 0.6$ & $\omega_H = 1.0$ \\
\hline
Attributed expectation & 30 & 50 \\
Equilibrium return & 30 & 30 \\
Material welfare & 30 & 30 \\
Human guilt & 0 & $-30$ \\
Extended welfare & 30 & 0 \\
\hline
\end{tabular}
\end{center}

At moderate anthropomorphism ($\omega_H = 0.6$), phantom expectations equal the maximum feasible return, so the human meets them and incurs no guilt. At high anthropomorphism ($\omega_H = 1.0$), phantom expectations exceed feasibility---the human returns everything possible but still fails to meet attributed expectations, incurring guilt of $-30$. Extended welfare drops from 30 to 0.
\end{example}

\subsection{Over-Anthropomorphism}

Define the \textbf{rational attribution benchmark} as the attributed belief a perfectly informed, non-anthropomorphising agent would form:
\[
    \tilde{h}_i^{(2,j),RAT} \equiv \phi_i(\theta_j, x_j, 0),
\]
where $\theta_j$ captures AI design parameters, $x_j$ denotes observable signals, and $\omega_i = 0$ indicates no psychological tendency to project human-like mental states.

Human $i$ exhibits \textbf{over-anthropomorphism} toward AI $j$ when attributed beliefs exceed this benchmark:
\[
    \tilde{h}_i^{(2,j)} > \tilde{h}_i^{(2,j),RAT}.
\]
We use ``over-anthropomorphism'' as a descriptive term for attribution exceeding the rational benchmark, without implying welfare harm. The prefix ``over-'' refers to the direction of deviation, not its normative valence---Part 1 below demonstrates that over-anthropomorphism can improve welfare.

\begin{proposition}[Welfare Effects of Over-Anthropomorphism]
\label{prop:over-anthro}
Under Assumptions (A1)--(A3), (A2') attribution monotonicity, and (A2'') attribution non-degeneracy:
\begin{enumerate}
    \item When AI is prosocially designed and (E) cooperation is efficient, over-anthropomorphism \emph{weakly} improves material welfare $W(s^*)$. The improvement is strict when over-anthropomorphism induces a regime switch from defection to cooperation.
    \item When AI is materialistically designed, over-anthropomorphism \emph{may} reduce extended welfare $W^{ext}$. This is an existence result: welfare loss requires that phantom expectations exceed feasible returns and that humans have positive guilt sensitivity.
\end{enumerate}
\end{proposition}

Part 1 is weak because over-anthropomorphism affects welfare only through equilibrium selection. Within pure defection or pure cooperation regions, marginal increases in $\tilde{h} - \tilde{h}^{RAT}$ do not alter behaviour. Part 2 is an existence result: when attributed expectations remain below feasible returns, humans satisfy phantom expectations and incur no guilt.

\subsection{Asymmetric Welfare Implications}

The two parts of Proposition~\ref{prop:over-anthro} reveal a fundamental asymmetry.

With \emph{prosocial AI}, over-anthropomorphism amplifies cooperation beyond what rational attribution would induce. Attributed expectations, though exceeding the rational benchmark, align directionally with AI preferences. The AI genuinely values cooperative outcomes; humans who attribute expectations to it are not entirely wrong about the direction of AI preferences, even if they overestimate their magnitude. This directional alignment creates mutual gains: humans cooperate more, AI achieves its prosocial objective, and material welfare rises.

With \emph{materialist AI}, over-anthropomorphism creates expectations where none exist. The AI has no prosocial preferences ($\rho_A = 0$), so any attributed expectations are \emph{phantom}---they exist in the human's psychological model but correspond to nothing in AI objectives. When these phantom expectations exceed feasible returns, humans incur guilt from failing to meet expectations that (a) the AI never held and (b) were impossible to satisfy. This guilt is pure welfare loss: it benefits no one.

The critical distinction is whether attributed expectations correspond to genuine AI preferences. When attributed expectations align with AI design (prosocial case), over-anthropomorphism coordinates behaviour toward efficient outcomes. When attributed expectations are phantom (materialist case), over-anthropomorphism generates deadweight psychological costs.

This asymmetry creates a potential alignment problem. If AI designers want to maximise human welfare, they must consider not just AI objectives but also how humans perceive those objectives. A prosocial AI with a mechanical interface may fail to elicit cooperation because humans do not attribute expectations to it. A materialist AI with an anthropomorphic interface may harm welfare by inducing guilt without providing offsetting benefits.

\subsection{Optimal AI Presentation}

Given these welfare effects, what is the optimal AI presentation strategy?

\begin{proposition}[Optimal AI Design]
\label{prop:optimal-design}
Consider the public goods game (Parts i, iii) or trust game (Part ii). Suppose (A1)--(A3) regularity, (A2') attribution monotonicity, and (A2''') signal monotonicity hold. Let $\omega \in (0,1)$ denote the representative human's anthropomorphism tendency.
\begin{enumerate}
    \item \textbf{Prosocial AI} ($\rho_A = 1$): Under (E) cooperation efficiency, (I) indignation dominance, and (T) temptation dominance, the optimal signal is $x^* = \max\{0, x_{crit}\}$, where $x_{crit}$ is the minimal signal to induce cooperation. Higher efficiency reduces the required signal: $\partial x^*/\partial m < 0$.

    \item \textbf{Materialist AI} ($\rho_A = 0$): Under (G') positive guilt sensitivity, $x^* = 0$. Any positive signal creates phantom expectations that reduce extended welfare.

    \item \textbf{Mixed AI} ($\rho_A \in (0,1)$): Under (E) and (G'), the optimal signal $x^* \in (0, \bar{x})$ balances cooperation benefits against guilt costs. At interior solutions: $\partial x^*/\partial m > 0$, $\partial x^*/\partial G < 0$, $\partial x^*/\partial \omega < 0$.
\end{enumerate}
\end{proposition}

The comparative statics in Parts (i) and (iii) move in opposite directions: $\partial x^*/\partial m < 0$ versus $\partial x^*/\partial m > 0$. This reflects different optimisation objectives. Part (i) finds the \emph{minimal} signal to cross the cooperation threshold---higher efficiency lowers the threshold, reducing the required signal. Part (iii) balances material and psychological welfare at the margin---higher efficiency increases the value of expanding cooperation, justifying a higher signal. As $\rho_A \to 1$ and guilt costs vanish, Part (iii) degenerates to Part (i).

The three parts nest naturally. As $\rho_A \to 1$, attributed expectations are met in cooperation equilibrium, guilt vanishes, and the solution approaches Part (i). As $\rho_A \to 0$, baseline expectations vanish, expectations become purely signal-driven, and the solution approaches Part (ii): $x^* \to 0$.

The proposition has implications for AI regulation. It characterises the \emph{social} optimum, where the planner internalises psychological costs. Private AI designers may have misaligned incentives: anthropomorphic presentation increases engagement and revenue, while psychological costs (guilt, disappointment) are externalised to users. When $\rho_A < 1$, private designers prefer higher $x$ than the social optimum, creating a case for regulation---disclosure requirements or anthropomorphism limits.

\subsection{AI Transparency and Regulation}

The phantom expectations mechanism supports a case for AI transparency requirements. If AI objectives were transparent, humans could calibrate $\omega$ appropriately: maintaining positive anthropomorphism toward prosocial AI (preserving cooperation benefits) while setting $\omega = 0$ toward materialist AI (eliminating phantom expectations).

The policy calculus depends on the population of AI systems:
\begin{itemize}
    \item If most AI is prosocially designed, some over-anthropomorphism may be welfare-enhancing.
    \item If most AI is materialistically designed, over-anthropomorphism causes net harm, favouring transparency mandates.
    \item With a mixed population, optimal policy must balance cooperation gains against psychological costs.
\end{itemize}

AI designers may have misaligned incentives. Anthropomorphic presentation increases engagement and revenue, while psychological costs from phantom expectations are borne by users. This externality suggests that market outcomes may feature excessive anthropomorphic design of materialist AI, strengthening the case for transparency regulation.

\subsection{Attenuation as a Welfare Buffer}

For expositional clarity, we adopt a representative-agent framework in this subsection, writing $\lambda^{IND}$ and $\lambda^{GUILT}$ for the common attenuation factors when all humans share identical psychological parameters.

The attenuation factors $\lambda^{IND}$ and $\lambda^{GUILT}$ serve as a natural buffer against welfare losses from over-anthropomorphism.

\begin{remark}[Welfare Role of Attenuation]
When moral emotions are attenuated toward AI ($\lambda < 1$), the psychological costs of disappointing AI are reduced. This protects humans from full exposure to phantom expectations, limiting welfare losses from over-anthropomorphism.
\end{remark}

However, attenuation also limits the welfare gains from prosocial AI. If humans do not feel guilt toward AI, then even a genuinely prosocial AI cannot induce cooperation through psychological mechanisms. The optimal attenuation level thus depends on the population composition: high attenuation is protective when AI is predominantly materialist, but costly when AI is predominantly prosocial.

\subsection{Cross-Cultural Implications}

Cross-cultural evidence suggests systematic variation in anthropomorphism $\omega$ across populations \citep{karpus2025cross}. Japanese participants exhibit guilt toward robots comparable to guilt toward humans (low attenuation, high effective $\omega$), while Western participants show strong attenuation (low effective $\omega$).

This variation has asymmetric welfare implications. Populations with high baseline $\omega$ benefit more from prosocial AI---larger cooperation gains from attributed expectations---but are also more vulnerable to materialist AI---larger phantom expectation costs. The cultural trait that amplifies benefits also amplifies harms.

Optimal AI design may therefore be culture-dependent. In high-$\omega$ cultures, prosocial design is especially valuable (large cooperation benefits), but materialist AI with anthropomorphic presentation is especially harmful (severe phantom expectations). In low-$\omega$ cultures, anthropomorphic design has limited effects regardless of AI objectives.

This asymmetry has regulatory implications. Transparency policies that enable calibrated anthropomorphism are especially important for high-$\omega$ populations. Without transparency, these populations face the largest welfare losses from materialist AI that presents anthropomorphically. International coordination on AI transparency may be complicated by divergent cultural exposures to phantom expectation costs.

\subsection{Dynamic Considerations}

The static welfare analysis assumes fixed anthropomorphism $\omega$ and attenuation $\lambda$. In practice, these parameters may evolve with AI exposure. If repeated interaction with AI reveals that attributed expectations are systematically violated (because AI is materialist), humans may learn to anthropomorphise less, naturally adjusting $\omega$ downward.

\begin{remark}[Learning and Long-Run Welfare]
If anthropomorphism is endogenous to experience, then short-run welfare losses from over-anthropomorphism may be self-correcting. However, if AI designers continuously update presentation to maintain anthropomorphism, this natural correction may be undermined.
\end{remark}

This dynamic creates a regulatory challenge. Static welfare analysis may underestimate losses if it ignores the ``arms race'' between AI presentation and human learning.

% Section 6: Conclusion
\section{Conclusion}
\label{sec:conclusion}

This paper introduces Attributed Belief Equilibrium to analyze strategic interaction between humans and artificial agents. The framework addresses a fundamental asymmetry: humans have belief-dependent preferences---guilt, reciprocity, indignation---while AI have design-dependent objectives with no genuine mental states. Standard psychological game theory cannot accommodate this heterogeneity because it assumes symmetric belief-dependent preferences across all players. ABE resolves this tension through the attribution function, which captures how humans project mental states onto AI even when those states do not exist.

\subsection{Summary of Contributions}

ABE's central innovation is its dual consistency requirement. Genuine beliefs about other humans satisfy standard Bayesian consistency as in \citet{battigalli2009dynamic}: humans correctly anticipate other humans' equilibrium strategies and beliefs. Attributed beliefs about AI satisfy a different condition---attribution consistency---determined by the attribution function $\phi$ given AI design parameters, observable signals, and the human's anthropomorphism tendency. This asymmetric treatment captures the empirical reality that humans form beliefs about AI ``expectations'' even when AI lack genuine mental states.

Existence of ABE follows from a Kakutani fixed-point argument under mild regularity conditions. The proof extends the recursive approach of \citet{battigalli2009dynamic} to dual belief structures, constructing attributed beliefs via $\phi$ and genuine beliefs via Bayesian consistency, then applying best responses to obtain a fixed point. Three nesting results establish ABE as a generalization of existing theory. ABE reduces to Psychological Nash Equilibrium \citep{geanakoplos1989psychological}---which coincides with Sequential Psychological Equilibrium for static games---when no AI agents are present. ABE reduces to Nash Equilibrium when psychological payoffs vanish. Under rational attribution, ABE corresponds bijectively to Bayes-Nash equilibria of an equivalent Bayesian game with type uncertainty about AI objectives. Rational attribution is itself a fixed-point requirement: $\phi$ projects equilibrium play, and the strategy profile is an equilibrium given $\phi$. When types are common knowledge, this reduces further to Nash equilibrium of a game with type-dependent payoffs (Corollary~\ref{cor:complete-info}).

A distinctive feature of ABE is attribution-dependent multiplicity. Different attribution functions sustain different equilibria in the same material game. The mechanism is a feed-forward chain: $\phi \to \tilde{h}^{(2)} \to U^H \to s^*$. Attributed beliefs are pinned down by $\phi$, so changing $\phi$ directly changes utility and thereby equilibrium. This contrasts with standard PGT multiplicity, which arises from feedback between strategies and belief consistency. The trust game shows how anthropomorphic presentation doubles equilibrium returns by elevating attributed expectations and thereby guilt from defection. The same logic extends to public goods provision, where high anthropomorphism sustains cooperation despite defection being materially dominant; the psychological cost of defecting on attributed expectations exceeds the material gain. Coordination games reveal a third mechanism: AI resolves equilibrium multiplicity by serving as a constructed focal point, creating psychological pressure to conform to the AI-favored equilibrium. These applications establish that interface design, framing, and behavioral presentation affect equilibrium selection even with fixed material payoffs.

The welfare analysis reveals an asymmetry between prosocial and materialist AI design. We distinguish material welfare---the sum of payoffs---from extended welfare that includes psychological states. Under conditions (E), (I), and (G) on cooperation efficiency and psychological sensitivity, this asymmetry takes a sharp form. When AI is prosocially designed, elevated anthropomorphism weakly improves material welfare, strictly so when it triggers cooperation. Elevated attributed expectations induce cooperation, and those expectations align with genuine AI preferences. When AI is materialistically designed, elevated anthropomorphism reduces extended welfare through phantom expectations. Humans attribute expectations that AI never held, and when these phantom expectations exceed feasible returns, humans incur guilt---a deadweight loss. Attenuation factors serve as a natural buffer: while they limit cooperation gains from prosocial AI, they protect humans from phantom expectation costs when AI is materialist.

Proposition~\ref{prop:optimal-design} characterizes optimal AI presentation. Prosocial AI should signal expectations at the minimal level needed to induce cooperation; higher efficiency reduces the required signal. Materialist AI should minimize signaling to avoid phantom expectations. Mixed AI with partially prosocial objectives requires balancing cooperation benefits against guilt costs. The comparative statics exhibit opposite signs: prosocial AI benefits from reduced signaling as efficiency rises, while mixed AI requires increased signaling to offset higher guilt costs.

\subsection{Relation to Existing Theory}

ABE nests standard equilibrium concepts as special cases. Without AI agents, ABE2 and ABE4 hold vacuously, and ABE1 and ABE3 reduce exactly to the optimality and belief consistency conditions of psychological equilibrium. Without psychological payoffs, beliefs become strategically irrelevant, and ABE strategy profiles coincide with Nash equilibria of the material game. This nesting ensures that ABE agrees with established theory in the domains where that theory applies.

The connection to Bayesian games illuminates when ABE generates new predictions. Under rational attribution, ABE is equivalent to a Bayesian game with type uncertainty about AI design parameters. ABE departs from Bayesian predictions precisely when attribution is systematically biased. Elevated anthropomorphism elevates attributed expectations above zero-anthropomorphism benchmarks; under-anthropomorphism attenuates them. These biases create predictable deviations from Bayesian outcomes, with welfare implications that depend on AI objectives.

\subsection{Limitations and Open Questions}

We have presented multiple approaches to specifying the attribution function---behavioral, signal-based, and dispositional---but have not resolved which is empirically appropriate in different contexts. The attribution function is the key primitive that distinguishes ABE from standard game theory, yet its empirical content remains to be established. Experimental work measuring how attributed beliefs respond to AI design features would enable calibration of $\phi$ and sharpen predictions.

The framework admits extensions we have not pursued. The current analysis stops at second-order attributed beliefs---what human $i$ attributes to AI $j$ about $i$'s behavior. Higher-order attributed beliefs may matter when humans reason about what AI ``thinks'' humans expect AI to expect. We treat anthropomorphism $\omega$ and attenuation $\lambda$ as fixed parameters, but humans likely update these based on experience with AI. A dynamic extension would analyze how attribution evolves through repeated interaction. Short-run welfare losses from phantom expectations may be self-correcting as humans learn from experience, though this correction could be undermined if AI designers continuously update presentation. The framework assumes humans observe AI design parameters $\theta_j$, but when AI types are uncertain, attributed beliefs depend on inferences about objectives, complicating the equilibrium analysis.

\subsection{Empirical Agenda}
\label{sec:empirical-agenda}

ABE generates testable predictions that distinguish it from standard game theory. The central prediction is that anthropomorphism and interface signals interact: in regression (\ref{eq:attribution-regression}), the coefficient $\beta_3 > 0$ indicates that high-$\omega_i$ subjects respond more strongly to anthropomorphic cues than low-$\omega_i$ subjects. Standard theory predicts $\beta_3 = 0$---interface design affects behavior uniformly or not at all. Three comparative statics sharpen the prediction. First, trust game returns $y^*$ increase in attributed beliefs $\tilde{h}_H^{(2,A)}$ through guilt aversion rather than material incentives. Second, public goods contributions exhibit threshold effects: cooperation emerges when $\omega_i \geq \bar{\omega}_i$, with the threshold depending on attenuation factor $\lambda_i^{IND}$. Third, AI population share has non-monotonic effects on cooperation---adding AI can sustain or erode cooperation depending on whether the psychological channel (attributed expectations) dominates the material channel (diluted payoffs).

Section~\ref{sec:empirical-implementation} details the experimental design: interface manipulation across robotic, humanoid, and control conditions; belief elicitation via quadratic scoring rules with hedging elimination; IDAQ measurement of individual anthropomorphism \citep{waytz2010who}. Two design choices require attention. Identifying phantom expectations---attributed beliefs about AI ``expectations'' despite AI lacking mental states---requires treatments where AI objectives $\rho_A$ are transparent versus hidden, to compare guilt when expectations are genuine versus projected. Cross-cultural samples are essential: \citet{karpus2025cross} document that Japanese participants exhibit guilt toward robots comparable to humans while Western participants show strong attenuation, implying that $\lambda_i^{GUILT}$ varies systematically across populations.

Measurement challenges remain. Distinguishing attributed beliefs from post-hoc rationalization requires careful timing of belief elicitation relative to strategic choice. The IDAQ captures general anthropomorphism tendency, but transferability to specific AI agents and game contexts is untested. Welfare predictions require measuring psychological costs inaccessible to introspection. Learning dynamics---whether phantom expectations persist or correct with experience---require repeated interaction designs with extended time horizons. Addressing these challenges will require combining laboratory precision with field studies of AI adoption.

\subsection{Implications for AI Development}

Design choices affect equilibrium behavior, not just user experience. Anthropomorphic design features---voice, naming, emotional expression, gaze behavior---enter equilibrium through the attribution function. The same material payoffs support different equilibrium behavior depending solely on how anthropomorphic the AI appears. A prosocial AI signaling expectations through humanlike cues induces more cooperation than an equally prosocial AI with a mechanical interface.

Beyond this direct effect, optimal presentation depends on the match between AI objectives and human projections. Prosocial AI benefits from anthropomorphic presentation because elevated attributed expectations align with genuine AI preferences and induce cooperation. Materialist AI harms welfare through anthropomorphic presentation because it creates phantom expectations that impose psychological costs without offsetting benefits. The analysis identifies a core design failure: anthropomorphic presentation of materialist AI. This combination creates phantom expectations that reduce welfare without offsetting gains. If AI designers have incentives to over-anthropomorphize to increase engagement, and if this creates phantom expectations that reduce user welfare, disclosure requirements or anthropomorphism limits may be warranted.

The cultural heterogeneity documented in \citet{karpus2025cross}---with Japanese participants exhibiting guilt toward robots comparable to humans while Western participants show strong attenuation---implies that optimal design varies across populations. High-anthropomorphism populations benefit more from prosocial AI but are more vulnerable to phantom expectations from materialist AI. The cultural trait that amplifies benefits also amplifies harms, complicating international coordination on AI design standards.

\subsection{Future Directions}

In companion work, we extend ABE to evolutionary settings to analyze how cooperation norms emerge when populations of humans and AI interact over time. The approach combines ABE with stochastic stability analysis to characterize long-run equilibrium selection under mutation and learning. Team collaboration, market competition, and AI-mediated negotiation all involve the asymmetric belief structures that ABE addresses. In teams, complementary expertise creates coordination gains that depend on attributed beliefs about AI contributions. In markets, attributed beliefs about AI strategies affect entry and pricing decisions. In negotiations, AI can serve as commitment devices whose effectiveness depends on whether humans attribute genuine expectations. Each setting extends the framework to new domains where the growing presence of artificial agents reshapes strategic interaction.


\newpage
\appendix
\section{Proofs}
\label{app:proofs}

This appendix contains complete proofs of all results stated in the main text.

\subsection{Proof of Theorem 1 (Existence of ABE)}
See proof sketch in Section \ref{sec:equilibrium}. % Proof of Theorem: Existence of ABE
% Appendix material for main manuscript
% Notation follows sec_framework.tex and sec_equilibrium.tex

\begin{proof}[Proof of Theorem~\ref{thm:existence}]
The proof applies Kakutani's fixed-point theorem to the best-response correspondence on the mixed strategy space. The key insight: both genuine and attributed beliefs are \emph{functions} of strategies, not independent equilibrium variables, so the fixed-point argument operates on the finite-dimensional space $\Sigma$ alone.

\medskip
\noindent\textbf{Step 1: Strategy space construction.}
Let $\Sigma = \prod_{i \in N} \Delta(S_i)$ be the space of mixed strategy profiles. Since each $S_i$ is finite and non-empty (A1), each simplex $\Delta(S_i)$ is non-empty, compact (closed and bounded in $\mathbb{R}^{|S_i|}$), and convex. The finite product $\Sigma$ inherits these properties.

\medskip
\noindent\textbf{Step 2: Belief computation.}
Given a strategy profile $\sigma \in \Sigma$, beliefs are determined as follows:
\begin{itemize}
    \item \emph{Attributed beliefs}: For human $i \in N_H$ and AI $j \in N_A$, define $\tilde{h}_i^{(2,j)}(\sigma) = \phi_i(\theta_j, x_j, \omega_i)$. These are independent of $\sigma$---constants determined by AI design parameters, observable signals, and the human's anthropomorphism tendency.
    \item \emph{Genuine beliefs}: For humans $i, k \in N_H$, define first-order beliefs $h_i^{(1,k)}(\sigma) = \sigma_k$ and second-order beliefs $h_i^{(2,k)}(\sigma) = \sigma_i$. Each is a coordinate projection, hence continuous in $\sigma$.
\end{itemize}

\medskip
\noindent\textbf{Step 3: Best-response correspondence.}
For each human $i \in N_H$, define $BR_i^H: \Sigma \rightrightarrows \Delta(S_i)$ by
\[
    BR_i^H(\sigma) = \arg\max_{\sigma_i' \in \Delta(S_i)} \sum_{s \in S} \left(\prod_{k \in N} \sigma_k'(s_k)\right) U_i^H(s, h_i(\sigma), \tilde{h}_i)
\]
where $\sigma_k' = \sigma_k$ for $k \neq i$, and beliefs $h_i(\sigma), \tilde{h}_i$ are computed as in Step 2.

For each AI $j \in N_A$, define $BR_j^A: \Sigma \rightrightarrows \Delta(S_j)$ by
\[
    BR_j^A(\sigma) = \arg\max_{\sigma_j' \in \Delta(S_j)} \sum_{s \in S} \left(\prod_{k \in N} \sigma_k'(s_k)\right) U_j^A(s; \theta_j)
\]
where $\sigma_k' = \sigma_k$ for $k \neq j$.

We verify that each $BR_k$ (for $k \in N$) is:
\begin{enumerate}
    \item[(i)] \emph{Non-empty valued}: The simplex $\Delta(S_k)$ is compact and the objective is continuous in $\sigma_k'$, so the maximum is attained (Weierstrass).
    \item[(ii)] \emph{Convex valued}: The objective is linear in $\sigma_k'$; any convex combination of maximizers is a maximizer.
    \item[(iii)] \emph{Upper hemicontinuous}: The constraint set $\Delta(S_k)$ is constant (hence continuous). The objective is continuous in $(\sigma_k', \sigma)$: the material component is polynomial in $\sigma$; the psychological component $\psi_i(s, h_i^{(2)}(\sigma), \tilde{h}_i^{(2)})$ is continuous by A1 (continuity of $\psi_i$ in beliefs) and continuity of the belief mappings. By Berge's Maximum Theorem, $BR_k$ is upper hemicontinuous.
\end{enumerate}

\medskip
\noindent\textbf{Step 4: Fixed-point application.}
Define the joint correspondence $BR: \Sigma \rightrightarrows \Sigma$ by $BR(\sigma) = \prod_{k \in N} BR_k(\sigma)$. The product of non-empty, convex-valued, upper hemicontinuous correspondences inherits these properties. Each $BR_k(\sigma)$ is closed (as the $\arg\max$ of a continuous function over a compact set), so $BR$ has a closed graph.

By Kakutani's fixed-point theorem: $\Sigma$ is non-empty, compact, and convex; $BR(\sigma)$ is non-empty and convex for all $\sigma$; $BR$ has a closed graph. Therefore, there exists $\sigma^* \in \Sigma$ with $\sigma^* \in BR(\sigma^*)$.

\medskip
\noindent\textbf{Step 5: Equilibrium verification.}
We verify that the fixed point $\sigma^*$, together with beliefs computed from Steps 2, satisfies all four ABE conditions.

\emph{(ABE1) Human Optimality}: For $i \in N_H$, the fixed-point property gives $\sigma_i^* \in BR_i^H(\sigma^*)$, so $\sigma_i^*$ maximizes $U_i^H$ given $\sigma_{-i}^*$ and the induced beliefs.

\emph{(ABE2) AI Optimality}: For $j \in N_A$, we have $\sigma_j^* \in BR_j^A(\sigma^*)$, so $\sigma_j^*$ maximizes $U_j^A$ given $\sigma_{-j}^*$.

\emph{(ABE3) Genuine Belief Consistency}: Define equilibrium beliefs by:
\begin{align*}
    h_i^{*(1,k)} &:= h_i^{(1,k)}(\sigma^*) = \sigma_k^* \\
    h_i^{*(2,k)} &:= h_i^{(2,k)}(\sigma^*) = \sigma_i^*
\end{align*}
First-order consistency holds: $h_i^{*(1,k)} = \sigma_k^*$ (beliefs equal actual strategies). For second-order consistency, we need $h_i^{*(2,k)} = h_k^{*(1,i)}$. By construction:
\[
    h_i^{*(2,k)} = \sigma_i^* = h_k^{(1,i)}(\sigma^*) = h_k^{*(1,i)}
\]
Thus $i$'s belief about what $k$ expected from $i$ equals what $k$ actually expected from $i$.

\emph{(ABE4) Attribution Consistency}: Define $\tilde{h}_i^{*(2,j)} := \phi_i(\theta_j, x_j, \omega_i)$. This is exactly the attribution consistency condition.

All four ABE conditions are satisfied. Therefore, $(\sigma^*, h^*, \tilde{h}^*)$ is an Attributed Belief Equilibrium.
\end{proof}

\begin{remark}[Role of Assumptions]
The proof uses: A1 (finite strategy spaces, continuous payoffs) for compactness, convexity, and Berge's theorem; A2 (well-defined attribution) for attributed beliefs to be valid probability distributions; A3 (bounded psychological payoffs) is implied by A1 given compactness of $\Sigma$, ensuring continuity of the utility function.
\end{remark}

\begin{remark}[Comparison with Standard PGT]
The ABE existence argument is structurally simpler than existence proofs for standard psychological game equilibria \citep{battigalli2009dynamic}. Attributed beliefs are exogenously determined---constants, not equilibrium variables. The fixed-point problem is confined to the finite-dimensional space $\Sigma$, not the infinite-dimensional belief space.
\end{remark}


\subsection{Proof of Proposition 1 (Reduction to Standard PGT)}
% Proof of Proposition: Reduction to Standard Psychological Game Theory
% For inclusion in ABE manuscript appendix
% Condensed version: 2026-01-14 (details moved to Online Appendix OA.2.1)

\noindent\textit{Scope clarification}: \citet{battigalli2009dynamic} define SPE for extensive-form psychological games. For static (normal-form) games, SPE coincides with PNE of \citet{geanakoplos1989psychological}. We prove that ABE reduces to this static-game equilibrium concept when $N_A = \emptyset$.

\paragraph{PNE Definition.} For a static psychological game with player set $N$, a Psychological Nash Equilibrium consists of $(s^*, h^*)$ satisfying:
\begin{enumerate}[label=\textbf{(PNE\arabic*)}]
    \item \textbf{Optimality}: For all $i \in N$, $s_i^* \in \arg\max_{s_i \in S_i} U_i(s_i, s_{-i}^*, h_i^*)$ where $U_i = \pi_i(s) + \psi_i(s, h_i^{(2)})$.
    \item \textbf{Belief Consistency}: For all $i, k \in N$ with $k \neq i$: $h_i^{*(1,k)} = s_k^*$ and $h_i^{*(2,k)} = h_k^{*(1,i)}$.
\end{enumerate}

\begin{proof}
Suppose $N_A = \emptyset$, so $N = N_H$.

\textbf{Step 1:} ABE2 and ABE4 are vacuously satisfied (quantification over empty $N_A$).

\textbf{Step 2:} The attributed belief system is empty: $\tilde{h}^* = \emptyset$.

\textbf{Step 3--6:} With empty $N_A$, psychological payoffs depend only on genuine beliefs, and the ABE conditions reduce exactly to PNE conditions. See Online Appendix OA.2.1 for detailed verification.

Therefore, $(s^*, h^*, \tilde{h}^*)$ is an ABE if and only if $(s^*, h^*)$ is a PNE.
\end{proof}


\subsection{Proof of Proposition 2 (Reduction to Nash)}
% Proof of Proposition: Reduction to Nash Equilibrium
% For inclusion in ABE manuscript appendix
% Condensed version: 2026-01-14 (details moved to Online Appendix OA.2.2)

\noindent\textit{Incomplete information}: When the game involves incomplete information about types, the material game $\Gamma^M$ becomes a Bayesian game where types affect only AI utilities $U_j^A(s; \theta_j)$. The result holds ex-post for each type realization.

\begin{proof}
The proof establishes a bijection between ABE of $\Gamma$ (when $\psi_i \equiv 0$) and Nash equilibria of $\Gamma^M$.

\medskip
\noindent\textbf{Part 1: ABE $\Rightarrow$ Nash.} When $\psi_i \equiv 0$, human utility reduces to $U_i^H = \pi_i(s)$. The ABE optimality conditions become Nash best-response conditions. Hence $s^*$ is a Nash equilibrium of $\Gamma^M$. (Details in Online Appendix OA.2.2.)

\medskip
\noindent\textbf{Part 2: Nash $\Rightarrow$ ABE.} Given Nash equilibrium $s^*$, construct beliefs by $h_i^{*(1,k)} = s_k^*$, $h_i^{*(2,k)} = s_i^*$, and $\tilde{h}_i^{*(2,j)} = \phi_i(\theta_j, x_j, \omega_i)$. All ABE conditions are satisfied. (Details in Online Appendix OA.2.2.)

\medskip
\noindent\textbf{Part 3:} The belief systems are uniquely determined by the consistency conditions.
\end{proof}


\subsection{Proof of Proposition 3 (Rational Attribution)}
% Proof of Proposition: Rational Attribution Equilibrium and Bayesian Game Equivalence
% For inclusion in ABE manuscript appendix
% Condensed version: 2026-01-14 (details moved to Online Appendix OA.2.3)

\paragraph{The Equivalent Bayesian Game $\Gamma^B$.}
Given the psychological game $\Gamma$ and a candidate equilibrium $\sigma^*$ satisfying Rational Attribution Equilibrium (RAE), construct:
\begin{itemize}
    \item Players: $N = N_H \cup N_A$.
    \item Type spaces: $\mathcal{T}_i = T_i \times \Omega_i$ for humans; $\mathcal{T}_j = \Theta_j$ for AI.
    \item Human payoff: $u_i^B(s, \tau) = \pi_i(s) + \psi_i(s, \sigma_i^*, \sigma_i^*)$ (beliefs equal equilibrium strategy under RAE).
    \item AI payoff: $u_j^B(s, \theta_j) = U_j^A(s; \theta_j)$.
\end{itemize}

\begin{proof}[Proof of Proposition~\ref{prop:rational-attribution}]

\noindent\textbf{Part (a): ABE with RAE $\Rightarrow$ BNE.}

Let $(\sigma^*, h^*, \tilde{h}^*)$ be an ABE where $(\sigma^*, \phi)$ satisfy RAE. Under RAE, all second-order beliefs---both genuine and attributed---equal $\sigma_i^*$. (See Online Appendix OA.2.3 for detailed belief characterization.)

Human $i$'s ABE utility coincides with Bayesian game payoff:
\begin{equation}
    U_i^H(\sigma_i, \sigma_{-i}^*, h_i^*, \tilde{h}_i^*) = \pi_i(\sigma_i, \sigma_{-i}^*) + \psi_i(\sigma_i, \sigma_{-i}^*, \sigma_i^*, \sigma_i^*) = u_i^B(\sigma_i, \sigma_{-i}^*).
\end{equation}
Since $\sigma_i^*$ maximizes $U_i^H$, it also maximizes $u_i^B$. AI payoffs are identical in both frameworks. Hence $\sigma^*$ is a BNE of $\Gamma^B$.

\medskip
\noindent\textbf{Part (b): BNE $\Rightarrow$ ABE with RAE.}

Let $\sigma^*$ be a BNE of $\Gamma^B$. Construct beliefs by $h_i^{*(1,k)} = \sigma_k^*$, $h_i^{*(2,k)} = \sigma_i^*$, and $\tilde{h}_i^{*(2,j)} = \phi_i(\theta_j, x_j, \omega_i) = \sigma_i^*$ (RAE condition). All ABE conditions are satisfied by construction.

\medskip
\noindent\textbf{Conclusion.}
The mappings establish a correspondence: the same strategy profile appears in both ABE and BNE when RAE holds.
\end{proof}


\subsection{Proof of Proposition 4 (Attribution-Dependent Multiplicity)}
% Proof of Proposition 5: Attribution-Dependent Multiplicity
% Updated: 2026-01-15 (proof moved to Online Appendix OA.2.4)

See Online Appendix OA.2.4 for the complete proof.


\subsection{Proof of Proposition 5 (Trust Game ABE)}
% Proof of Proposition: Trust Game ABE
% Appendix material for main manuscript
% Notation follows sec_framework.tex and sec_applications.tex

% Note: The proposition statement is in sec_applications.tex.
% This proof requires Assumption (A2') Attribution Monotonicity and
% condition (G) Guilt Dominance: $G \equiv \gamma_H \lambda_H^{GUILT} > 1$.

\begin{proof}[Proof of Proposition~\ref{prop:trust}]
The proposition requires two conditions: (A2') the attribution function $\phi_H$ is weakly increasing in $\omega_H$; and (G) guilt dominance $G \equiv \gamma_H \lambda_H^{GUILT} > 1$.
We prove each claim in turn.

\medskip
\noindent\textbf{Proof of (i): Anthropomorphism increases attributed expectations.}

By Definition~\ref{def:attributed-beliefs}, the human trustee's attributed second-order belief is
\begin{equation}
    \tilde{h}_H^{(2,A)} = \phi_H(\rho_A, x_A, \omega_H),
\end{equation}
where $\rho_A$ is AI prosociality, $x_A$ is the amount sent, and $\omega_H \in [0,1]$ is anthropomorphism.

Assumption (A2') states that for any $\omega_H' > \omega_H$:
\begin{equation}
    \phi_H(\rho_A, x_A, \omega_H') \geq \phi_H(\rho_A, x_A, \omega_H).
\end{equation}
Hence $\partial \tilde{h}_H^{(2,A)} / \partial \omega_H \geq 0$.

\medskip
\noindent\textbf{Proof of (ii): Higher attributed expectations increase equilibrium returns.}

The human's utility is
\begin{equation}
    U_H(y) = 3x - y - G \cdot \max\{0, \tilde{h} - y\},
\end{equation}
where $G = \gamma_H \lambda_H^{GUILT}$, $y \in [0, 3x]$, and $\tilde{h} = \tilde{h}_H^{(2,A)}$.

\emph{Step 1 (Piecewise structure).} The $\max$ term creates a kink at $y = \tilde{h}$:
\begin{equation}
    U_H(y) = \begin{cases}
        3x - G\tilde{h} + (G-1)y & \text{if } y < \tilde{h}, \\[4pt]
        3x - y & \text{if } y \geq \tilde{h}.
    \end{cases}
\end{equation}

\emph{Step 2 (Marginal utility).} Differentiating:
\begin{equation}
    \frac{\partial U_H}{\partial y} = \begin{cases}
        G - 1 & \text{if } y < \tilde{h}, \\[4pt]
        -1 & \text{if } y > \tilde{h}.
    \end{cases}
\end{equation}
Under (G), $G > 1$ implies $\partial U_H / \partial y > 0$ for $y < \tilde{h}$ and $\partial U_H / \partial y < 0$ for $y > \tilde{h}$.

\emph{Step 3 (Optimal return).} The human maximizes $U_H(y)$ over $[0, 3x]$.
\begin{itemize}
    \item If $\tilde{h} \leq 3x$: Utility is strictly increasing on $[0, \tilde{h})$ and strictly decreasing on $(\tilde{h}, 3x]$, so the unique maximum is $y^* = \tilde{h}$.
    \item If $\tilde{h} > 3x$: Utility is strictly increasing on all of $[0, 3x]$, so $y^* = 3x$.
\end{itemize}
Combining: $y^* = \min\{\tilde{h}, 3x\}$.

\emph{Step 4 (Comparative statics).} Since $y^* = \min\{\tilde{h}, 3x\}$ is weakly increasing in $\tilde{h}$:
\begin{equation}
    \frac{\partial y^*}{\partial \tilde{h}} = \begin{cases}
        1 & \text{if } \tilde{h} < 3x, \\[4pt]
        0 & \text{if } \tilde{h} > 3x.
    \end{cases}
\end{equation}
At $\tilde{h} = 3x$, the function has a kink; the subdifferential is $[0,1]$.

\medskip
\noindent\textbf{Proof of (iii): Same material payoffs, different equilibria.}

By Claims (i) and (ii), $y^*$ depends on $\omega_H$ through attributed beliefs. We construct an explicit example.

\emph{Setup.} Let $E = 10$, $\rho_A = 0.3$, $\gamma_H = 2$, $\lambda_H^{GUILT} = 0.6$ (so $G = 1.2 > 1$), and the AI sends $x = 10$. Consider two humans with identical material payoff functions $\pi_H(y) = 30 - y$ but different anthropomorphism:
\begin{itemize}
    \item Human $H$: $\omega_H = 0.8$
    \item Human $H'$: $\omega_H' = 0.3$
\end{itemize}

\emph{Attributed beliefs.} Using $\phi_H(\rho_A, x_A, \omega_H) = \omega_H \cdot \rho_A \cdot 3x$:
\begin{align*}
    \tilde{h}_H^{(2,A)} &= 0.8 \times 0.3 \times 30 = 7.2, \\
    \tilde{h}_{H'}^{(2,A)} &= 0.3 \times 0.3 \times 30 = 2.7.
\end{align*}

\emph{Equilibrium returns.} By Claim (ii):
\begin{align*}
    y^*_H &= \min\{7.2, 30\} = 7.2, \\
    y^*_{H'} &= \min\{2.7, 30\} = 2.7.
\end{align*}

\emph{Equilibrium comparison.}
\begin{center}
\begin{tabular}{lcc}
\hline
& $\omega_H = 0.8$ & $\omega_H' = 0.3$ \\
\hline
Material payoff function & $30 - y$ & $30 - y$ \\
Attributed belief $\tilde{h}^{(2,A)}$ & 7.2 & 2.7 \\
Equilibrium return $y^*$ & 7.2 & 2.7 \\
AI payoff & 7.2 & 2.7 \\
Human payoff & 22.8 & 27.3 \\
\hline
\end{tabular}
\end{center}

The material payoff functions are identical, yet equilibrium outcomes differ: anthropomorphism affects attributed beliefs, which enter psychological payoffs and determine guilt-driven returns. This establishes (iii).
\end{proof}

\begin{remark}[The knife-edge case $G = 1$]
\label{rem:G-equals-one}
When $G = \gamma_H \lambda_H^{GUILT} = 1$, the marginal utility $\partial U_H / \partial y = 0$ for all $y < \tilde{h}$. The human is indifferent over $[0, \min\{\tilde{h}, 3x\}]$, so the optimal return is not unique. In this case, $y^* = \min\{\tilde{h}, 3x\}$ is the Pareto-best selection for the AI; other equilibrium selection criteria (e.g., trembling-hand perfection) may yield different predictions. The strict condition $G > 1$ ensures uniqueness.
\end{remark}

\begin{remark}[The case $G < 1$]
\label{rem:low-guilt}
When $G = \gamma_H \lambda_H^{GUILT} < 1$, the marginal utility $\partial U_H / \partial y = G - 1 < 0$ for all $y < \tilde{h}$. The human's utility is strictly decreasing on $[0, 3x]$, so $y^* = 0$ regardless of attributed expectations. In this regime, anthropomorphism has no effect on equilibrium returns: psychological sensitivity is too weak to overcome material self-interest. The condition $G > 1$ is therefore necessary for attributed beliefs to influence behavior.
\end{remark}

\begin{remark}[Connection to Proposition~\ref{prop:multiplicity}]
The example in part (iii) satisfies the conditions for attribution-dependent multiplicity: (a) belief-dependent payoffs ($\partial \psi_H / \partial \tilde{h} \neq 0$ when $y < \tilde{h}$); (b) distinct attributions across anthropomorphism levels; (c) best-response separation ($y^*(7.2) \neq y^*(2.7)$). The multiplicity arises not from multiple equilibria in a fixed game, but from different attributed beliefs generating different optimal responses.
\end{remark}

\begin{remark}[Empirical support for (A2')]
Attribution monotonicity is well-supported empirically. Meta-analytic evidence confirms a positive relationship between anthropomorphism and trust across 97 effect sizes \citep{blut2021understanding}. \citet{waytz2014trusting} demonstrate that anthropomorphizing autonomous systems increases trust, suggesting attributed expectations rise with anthropomorphism.
\end{remark}


\subsection{Proof of Proposition 6 (Public Goods ABE)}
% Proof of Proposition: Public Goods ABE
% Appendix material for main manuscript
% Notation follows sec_framework.tex and sec_applications.tex
% Requires: Assumption (A2') Attribution Monotonicity, Condition (I) Indignation Dominance,
%           and $\beta_i > 0$ for all humans.

% Note: The proposition statement is in sec_applications.tex.
% This proof requires:
%   (A2') Attribution Monotonicity: $\phi_i(\theta_j, x_j, \omega_i)$ weakly increasing in $\omega_i$
%   (I)   Indignation Dominance: $\beta_i[(n_H - 1) + \lambda_i^{IND} n_A] > E(1 - m/n)$
%   $\beta_i > 0$ for all $i \in N_H$ (positive indignation sensitivity)

\begin{proof}[Proof of Proposition~\ref{prop:public-goods}]
The proposition requires (A2') attribution monotonicity, (I) indignation dominance, and $\beta_i > 0$ for all humans. We prove each claim in turn.

The human utility function is $U_i^H = \pi_i + \psi_i^{IND}$, where:
\begin{align}
    \pi_i(c) &= E - c_i + \frac{m}{n} \sum_{k \in N} c_k, \\
    \psi_i^{IND} &= -\beta_i \cdot \mathbf{1}_{c_i < c^*} \cdot \left[ \sum_{k \in N_H \setminus \{i\}} h_i^{(2,k)}(c^*) + \lambda_i^{IND} \sum_{j \in N_A} \tilde{h}_i^{(2,j)}(c^*) \right],
\end{align}
with reference level $c^* = E$. Under dispositional attribution, $\tilde{h}_i^{(2,j)}(E) = \omega_i \cdot \bar{h}(c_A)$, where $\bar{h}(E) = \bar{h}_H$ (high baseline) and $\bar{h}(0) = \bar{h}_L$ (low baseline), with $\bar{h}_H > \bar{h}_L \geq 0$.

\medskip
\noindent\textbf{Proof of (i): Cooperation equilibrium.}

We show that under conditions (A2') and (I), if AI contribute $c_A = E$ and humans attribute sufficiently high expectations ($\omega_i \geq \bar{\omega}_i$), the symmetric profile $c^* = (E, \ldots, E)$ constitutes an ABE.

\emph{Step 1 (Candidate equilibrium).}
Consider the strategy profile:
\begin{equation}
    c_k^* = E \quad \text{for all } k \in N.
\end{equation}
Under this profile, all humans cooperate ($n_C^H = n_H$) and all AI cooperate by design ($n_C^A = n_A$).

\emph{Step 2 (Genuine belief consistency).}
In the candidate equilibrium, genuine belief consistency (Definition~\ref{def:ABE}(iii)) requires:
\begin{equation}
    h_i^{(2,k)}(E) = 1 \quad \text{for all } k \in N_H \setminus \{i\}.
\end{equation}

\emph{Step 3 (Attribution consistency).}
By dispositional attribution with cooperating AI:
\begin{equation}
    \tilde{h}_i^{(2,j)}(E) = \omega_i \cdot \bar{h}_H \quad \text{for all } j \in N_A.
\end{equation}
This satisfies attribution consistency (Definition~\ref{def:ABE}(iv)).

\emph{Step 4 (Cooperation payoff).}
If human $i$ cooperates:
\begin{equation}
    U_i^H(E \mid c_{-i}^* = E) = \frac{m}{n}(n_H + n_A) E = mE.
\end{equation}
The psychological term vanishes because $\mathbf{1}_{E < E} = 0$.

\emph{Step 5 (Defection payoff).}
If human $i$ deviates to $c_i = 0$:
\begin{align}
    \pi_i(0 \mid c_{-i}^* = E) &= E + \frac{m(n-1)}{n} E, \\
    \psi_i^{IND}(0) &= -\beta_i \left[ (n_H - 1) + \lambda_i^{IND} n_A \omega_i \bar{h}_H \right].
\end{align}
Thus:
\begin{equation}
    U_i^H(0 \mid c_{-i}^* = E) = E\left(1 + \frac{m(n-1)}{n}\right) - \beta_i \left[ (n_H - 1) + \lambda_i^{IND} n_A \omega_i \bar{h}_H \right].
\end{equation}

\emph{Step 6 (Human optimality).}
Cooperation is optimal iff $U_i^H(E) \geq U_i^H(0)$:
\begin{equation}\label{eq:coop-ic}
    \beta_i \left[ (n_H - 1) + \lambda_i^{IND} n_A \omega_i \bar{h}_H \right] \geq E\left(1 - \frac{m}{n}\right).
\end{equation}

\emph{Step 7 (Threshold anthropomorphism).}
When $\lambda_i^{IND} n_A > 0$, solving \eqref{eq:coop-ic} for $\omega_i$ yields the threshold:
\begin{equation}\label{eq:omega-threshold}
    \bar{\omega}_i = \frac{E(1 - m/n) - \beta_i(n_H - 1)}{\beta_i \lambda_i^{IND} n_A \bar{h}_H}.
\end{equation}
Condition (I) ensures $\bar{\omega}_i \leq 1$, so cooperation is attainable. When $\omega_i \geq \bar{\omega}_i$, cooperation is weakly optimal (with indifference at the threshold). When $\bar{\omega}_i \leq 0$, cooperation holds for all $\omega_i \in [0,1]$.

\emph{Step 8 (ABE verification).}
The candidate profile satisfies: (i) human optimality (Step 6); (ii) genuine belief consistency (Step 2); (iii) attribution consistency (Step 3). \hfill $\square$

\medskip
\noindent\textbf{Proof of (ii): Defection equilibrium.}

We show that under low anthropomorphism with defecting AI, symmetric defection is the unique symmetric ABE (asymmetric equilibria may exist but are not characterized).

\emph{Step 1 (Candidate equilibrium).}
Consider the profile:
\begin{equation}
    c_k^* = 0 \quad \text{for all } k \in N.
\end{equation}

\emph{Step 2 (Belief consistency).}
In symmetric defection:
\begin{align}
    h_i^{(2,k)}(E) &= 0 \quad \text{for all } k \in N_H \setminus \{i\}, \\
    \tilde{h}_i^{(2,j)}(E) &= \omega_i \cdot \bar{h}_L \quad \text{for all } j \in N_A.
\end{align}

\emph{Step 3 (Psychological cost).}
The indignation cost of defection:
\begin{equation}
    |\psi_i^{IND}| = \beta_i \lambda_i^{IND} n_A \omega_i \bar{h}_L.
\end{equation}
As $\omega_i \to 0$, this vanishes.

\emph{Step 4 (Defection dominance).}
Defection is preferred iff:
\begin{equation}
    E\left(1 - \frac{m}{n}\right) > \beta_i \lambda_i^{IND} n_A \omega_i \bar{h}_L.
\end{equation}
This holds when $\omega_i < \underline{\omega}$, where (assuming $\lambda_i^{IND} n_A > 0$):
\begin{equation}
    \underline{\omega} = \frac{E(1 - m/n)}{\beta_i \lambda_i^{IND} n_A \bar{h}_L}.
\end{equation}

\emph{Step 5 (Uniqueness among symmetric equilibria).}
Symmetric cooperation fails when $\omega_i$ is low: even with belief consistency requiring $h_i^{(2,k)}(E) = 1$, if $\beta_i(n_H - 1) + \beta_i \lambda_i^{IND} n_A \omega_i \bar{h}_L < E(1 - m/n)$, deviation is profitable. The defection equilibrium is self-fulfilling: zero expectations yield zero psychological costs, confirming defection as optimal. \hfill $\square$

\medskip
\noindent\textbf{Proof of (iii): Population share effects.}

Fix $n_H \geq 2$ and vary $n_A \geq 0$. Define:
\begin{align}
    \Delta\pi_i &= E\left(1 - \frac{m}{n_H + n_A}\right) \quad \text{(material temptation)}, \\
    \Psi_i &= \beta_i \left[ \sum_{k \in N_H \setminus \{i\}} h_i^{(2,k)}(E) + \lambda_i^{IND} n_A \omega_i \bar{h}(c_A) \right] \quad \text{(psychological deterrent)}.
\end{align}

\emph{Material channel.}
The material temptation increases with $n_A$:
\begin{equation}
    \frac{\partial (\Delta\pi_i)}{\partial n_A} = \frac{mE}{(n_H + n_A)^2} > 0.
\end{equation}
More AI dilute the MPCR, increasing free-rider incentives.

\emph{Psychological channel.}
The psychological deterrent also increases with $n_A$:
\begin{equation}
    \frac{\partial \Psi_i}{\partial n_A} = \beta_i \lambda_i^{IND} \omega_i \bar{h}(c_A) \geq 0.
\end{equation}
More AI create more sources of attributed expectations.

\emph{Net effect.}
Define the cooperation incentive $I_i = \Psi_i - \Delta\pi_i$. Then:
\begin{equation}
    \frac{\partial I_i}{\partial n_A} = \underbrace{\beta_i \lambda_i^{IND} \omega_i \bar{h}(c_A)}_{\text{psychological (+)}} - \underbrace{\frac{mE}{(n_H + n_A)^2}}_{\text{material (--)}}.
\end{equation}
The psychological channel has constant marginal effect; the material channel weakens as $n_A$ grows. For sufficiently large $n_A$, the psychological channel dominates whenever $\lambda_i^{IND} \omega_i \bar{h}(c_A) > 0$.

\begin{center}
\begin{tabular}{lcc}
\hline
\textbf{Channel} & \textbf{Direction} & \textbf{Behavior} \\
\hline
Material & $\partial(\Delta\pi_i)/\partial n_A > 0$ & Diminishes with $n_A$ \\
Psychological & $\partial \Psi_i/\partial n_A \geq 0$ & Constant in $n_A$ \\
\hline
\end{tabular}
\end{center}

This establishes that AI population share affects equilibrium through both channels, with the net effect depending on $\lambda_i^{IND}$. \hfill $\square$
\end{proof}

\begin{remark}[Connection to standard PGT]
\label{rem:standard-pgt}
When $n_A = 0$, condition \eqref{eq:coop-ic} reduces to:
\begin{equation}
    \beta_i(n_H - 1) \geq E\left(1 - \frac{m}{n_H}\right),
\end{equation}
the standard condition for cooperation in psychological public goods games with indignation \citep{battigalli2022belief}. ABE extends this by adding the term $\lambda_i^{IND} n_A \omega_i \bar{h}(c_A)$, capturing psychological costs from attributed AI expectations. The nesting result confirms that ABE generalizes standard PGT.
\end{remark}

\begin{remark}[Role of attenuation $\lambda_i^{IND}$]
\label{rem:attenuation-role}
The attenuation factor $\lambda_i^{IND}$ determines whether AI presence affects equilibrium psychologically. When $\lambda_i^{IND} \approx 1$, indignation toward AI operates at full strength, and the psychological channel can dominate material dilution. When $\lambda_i^{IND} \approx 0$, AI are psychologically irrelevant: humans may acknowledge AI ``expectations'' but experience no emotional response to violating them. This parameter captures the intuition that attributed mental states may feel less binding than genuine ones.
\end{remark}

\begin{remark}[Boundary cases]
\label{rem:boundary-cases}
When $n_A = 0$, the threshold $\bar{\omega}_i$ in equation~\eqref{eq:omega-threshold} is undefined; cooperation requires $\beta_i(n_H - 1) \geq E(1 - m/n)$ independent of $\omega_i$. When $\lambda_i^{IND} = 0$ with $n_A > 0$, AI contribute only through material payoffs; the psychological term vanishes and the analysis reduces to standard public goods games without belief-dependent preferences toward AI. The assumption $\beta_i > 0$ ensures indignation has behavioral content; when $\beta_i = 0$, the incentive constraint \eqref{eq:coop-ic} cannot be satisfied since $E(1 - m/n) > 0$.
\end{remark}

\begin{remark}[Role of assumption (A2')]
\label{rem:a2-role}
Assumption (A2') attribution monotonicity---that $\phi_i(\theta_j, x_j, \omega_i)$ is weakly increasing in $\omega_i$---is not used in the existence proof per se, but ensures the threshold $\bar{\omega}_i$ has the correct comparative statics interpretation. Under (A2'), higher anthropomorphism leads to higher attributed expectations, which in turn lowers the cooperation threshold through \eqref{eq:omega-threshold}. Without (A2'), a human with higher $\omega_i$ might attribute lower expectations, reversing the relationship between anthropomorphism and cooperation.
\end{remark}

\begin{remark}[Empirical implications]
\label{rem:empirical-implications}
The proposition generates testable predictions:
\begin{enumerate}
    \item \textbf{AI behavior matters}: Cooperating AI sustain human cooperation more effectively than defecting AI, through higher attributed expectations ($\bar{h}_H > \bar{h}_L$).
    \item \textbf{Anthropomorphism matters}: Higher $\omega_i$ lowers the cooperation threshold, making cooperation easier to sustain.
    \item \textbf{Population composition effects are non-monotonic}: Adding AI may increase or decrease cooperation depending on whether the psychological channel (via $\lambda_i^{IND}$) or material channel dominates.
    \item \textbf{Null prediction}: If $\omega_i = 0$ or $\lambda_i^{IND} = 0$, AI presence affects cooperation only through material dilution---the novel ABE effects vanish.
\end{enumerate}
These predictions distinguish ABE from standard models where AI enter only through material payoffs.
\end{remark}

\begin{remark}[Threshold interpretation]
\label{rem:threshold-interpretation}
The thresholds $\bar{\omega}_i$ (for cooperation) and $\underline{\omega}$ (for defection) have natural interpretations. At $\omega_i = \bar{\omega}_i$, the psychological cost of defection exactly equals the material gain; cooperation requires sufficient anthropomorphism to push beyond this threshold. Conversely, $\underline{\omega}$ marks the point below which material incentives dominate regardless of attributed expectations. The gap between these thresholds defines a region of multiplicity where both equilibria may exist.
\end{remark}


\subsection{Proof of Proposition 7 (Coordination ABE)}
% Proof of Proposition: Coordination ABE
% Appendix material for main manuscript
% Notation follows sec_framework.tex and sec_applications.tex

% Note: The proposition statement is in sec_applications.tex.
% This proof requires Assumption (A2') Attribution Monotonicity and
% condition (C) Signal Clarity: $x_A > 0.5$.

\begin{proof}[Proof of Proposition~\ref{prop:coordination}]
The proposition establishes three claims about coordination games with AI as focal point providers. We first formalize the game structure and then prove each claim.

\medskip
\noindent\textbf{Game Setup.}

Consider a coordination game with one human $H \in N_H$ and one AI agent $A \in N_A$. Each player chooses from $S_H = S_A = \{A, B\}$. Material payoffs are:
\begin{equation}
    \pi_i(s) = \begin{cases}
        2 & \text{if } s_H = s_A \\
        0 & \text{otherwise}
    \end{cases}
\end{equation}
The material game has two Nash equilibria: $(A, A)$ and $(B, B)$.

The AI is designed with commitment to technology $A$: its utility is $U_A(s) = \pi_A(s) + \theta_A \cdot \mathbf{1}_{s_A = A}$ where $\theta_A > 0$. The human experiences expectation conformity: deviating from the AI's perceived expectation creates psychological cost.

The human's psychological payoff is:
\begin{equation}
    \psi_H(s_H) = -\beta_H \cdot \lambda_H^{EC} \cdot \sum_{s' \neq s_H} \tilde{h}_H^{(2,A)}(s'),
\end{equation}
where $\tilde{h}_H^{(2,A)}(s')$ is the attributed probability that AI expected human to play $s'$. Since $S_H = \{A, B\}$:
\begin{itemize}
    \item If $s_H = A$: $\psi_H = -\beta_H \lambda_H^{EC} \cdot \tilde{h}_H^{(2,A)}(B)$
    \item If $s_H = B$: $\psi_H = -\beta_H \lambda_H^{EC} \cdot \tilde{h}_H^{(2,A)}(A)$
\end{itemize}

The attribution function determines how the human forms attributed beliefs. When AI plays $A$ and signals with clarity $x_A \in [0,1]$:
\begin{equation}
    \tilde{h}_H^{(2,A)}(A) = \omega_H \cdot x_A, \quad \tilde{h}_H^{(2,A)}(B) = \omega_H \cdot (1 - x_A).
\end{equation}

\medskip
\noindent\textbf{Required Conditions.}
\begin{itemize}
    \item (A2') Attribution Monotonicity: $\partial \tilde{h}_H^{(2,A)}(A) / \partial \omega_H \geq 0$
    \item (C) Signal Clarity: $x_A > 0.5$, meaning AI's signal favors action $A$
\end{itemize}

\medskip
\noindent\textbf{Proof of (i): AI signaling $A$ clearly makes attributed beliefs favour $A$.}

\emph{Step 1 (AI's optimal strategy).} Given design commitment $\theta_A > 0$, the AI's utility is:
\begin{align}
    U_A(A, s_H) &= \pi_A(A, s_H) + \theta_A, \\
    U_A(B, s_H) &= \pi_A(B, s_H).
\end{align}

If $s_H = A$: $U_A(A, A) = 2 + \theta_A > U_A(B, A) = 0$.

If $s_H = B$: $U_A(A, B) = \theta_A$ vs $U_A(B, B) = 2$. For $\theta_A > 2$, AI prefers $A$ regardless of human play. For $\theta_A \leq 2$, AI prefers $A$ when human plays $A$, ensuring $(A, A)$ is an equilibrium.

Under the design interpretation that the AI is \emph{programmed} to coordinate on $A$ (a binding constraint, not merely a preference), $s_A^* = A$ in any ABE.

\emph{Step 2 (Attribution favours $A$).} Given AI plays $A$ and signals with clarity $x_A$:
\begin{align}
    \tilde{h}_H^{(2,A)}(A) &= \omega_H \cdot x_A, \\
    \tilde{h}_H^{(2,A)}(B) &= \omega_H \cdot (1 - x_A).
\end{align}

Under condition (C) with $x_A > 0.5$:
\begin{equation}
    \tilde{h}_H^{(2,A)}(A) = \omega_H \cdot x_A > \omega_H \cdot (1 - x_A) = \tilde{h}_H^{(2,A)}(B).
\end{equation}

Hence attributed beliefs favour $A$: the human believes AI expected $A$ with higher probability than $B$. \qed

\medskip
\noindent\textbf{Proof of (ii): High anthropomorphism amplifies the psychological pull toward the AI-favoured equilibrium.}

\emph{Step 1 (Define psychological pull).} The psychological pull toward $A$ is the utility advantage from matching AI:
\begin{equation}
    \Delta U_H \equiv U_H(A; A) - U_H(B; A).
\end{equation}

\emph{Step 2 (Compute utilities).} Given $s_A = A$:
\begin{align}
    U_H(A; A) &= \pi_H(A, A) + \psi_H(A) = 2 - \beta_H \lambda_H^{EC} \cdot \omega_H(1 - x_A), \\
    U_H(B; A) &= \pi_H(B, A) + \psi_H(B) = 0 - \beta_H \lambda_H^{EC} \cdot \omega_H \cdot x_A.
\end{align}

\emph{Step 3 (Derive the pull).}
\begin{align}
    \Delta U_H &= 2 - \beta_H \lambda_H^{EC} \omega_H(1 - x_A) + \beta_H \lambda_H^{EC} \omega_H x_A \\
    &= 2 + \beta_H \lambda_H^{EC} \omega_H (2x_A - 1).
\end{align}

\emph{Step 4 (Comparative statics).}
\begin{equation}
    \frac{\partial \Delta U_H}{\partial \omega_H} = \beta_H \lambda_H^{EC} (2x_A - 1).
\end{equation}

Under condition (C), $x_A > 0.5$ implies $2x_A - 1 > 0$, so:
\begin{equation}
    \frac{\partial \Delta U_H}{\partial \omega_H} = \beta_H \lambda_H^{EC} (2x_A - 1) > 0.
\end{equation}

Higher anthropomorphism $\omega_H$ increases the utility advantage of playing $A$. The psychological pull toward the AI-favoured equilibrium is amplified. \qed

\medskip
\noindent\textbf{Proof of (iii): AI can serve as focal point providers, resolving coordination problems.}

\emph{Step 1 (Material game multiplicity).} The material game has two Nash equilibria: $(A, A)$ and $(B, B)$, both yielding payoffs $(2, 2)$. Neither is payoff-dominant; the game exhibits coordination multiplicity.

\emph{Step 2 ($(A, A)$ is an ABE).} We verify the four ABE conditions:
\begin{enumerate}
    \item \textbf{Human optimality (ABE1):} By Claim (ii), $\Delta U_H = 2 + \beta_H \lambda_H^{EC} \omega_H (2x_A - 1) > 0$ when $x_A > 0.5$ and parameters are positive. Hence $A$ is the unique best response to $s_A = A$.

    \item \textbf{AI optimality (ABE2):} Given design commitment $\theta_A > 0$ and the human playing $A$, we have $U_A(A, A) = 2 + \theta_A > U_A(B, A) = 0$. Hence $A$ is optimal for AI.

    \item \textbf{Genuine belief consistency (ABE3):} With only one human, this is vacuously satisfied.

    \item \textbf{Attribution consistency (ABE4):} By construction, $\tilde{h}_H^{(2,A)} = \phi_H(\theta_A, x_A, \omega_H)$.
\end{enumerate}

Therefore $(A, A)$ is an ABE.

\emph{Step 3 ($(B, B)$ is not an ABE).} At profile $(B, B)$:
\begin{itemize}
    \item If $\theta_A > 2$: $U_A(A, B) = \theta_A > 2 = U_A(B, B)$, so AI deviates to $A$. ABE2 fails.
    \item If $\theta_A \leq 2$: Under the binding constraint interpretation (AI is \emph{programmed} to play $A$), $s_A = B$ violates the constraint. ABE2 fails.
\end{itemize}

Therefore $(B, B)$ is not an ABE.

\emph{Step 4 (Uniqueness).} Any ABE must have $s_A^* = A$ by AI optimality. Given $s_A^* = A$, human's unique best response is $A$ (by Claim ii). Hence $(A, A)$ is the unique ABE.

\emph{Step 5 (Focal point mechanism).} AI resolves coordination multiplicity through:
\begin{enumerate}
    \item \textbf{Commitment}: Design parameter $\theta_A > 0$ ensures AI plays $A$
    \item \textbf{Signaling}: Clarity $x_A > 0.5$ makes AI's choice salient
    \item \textbf{Attribution}: Human attributes expectations $\tilde{h}_H^{(2,A)}(A) > \tilde{h}_H^{(2,A)}(B)$
    \item \textbf{Psychological pressure}: Expectation conformity penalizes deviation from $A$
\end{enumerate}
The AI serves as an endogenous focal point provider. \qed
\end{proof}

\begin{remark}[Numerical Example]
\label{rem:coordination-example}
Consider $\beta_H = 3$, $\lambda_H^{EC} = 0.5$, $\omega_H = 0.8$, $x_A = 0.9$, $\theta_A = 0.5$.

Attribution: $\tilde{h}_H^{(2,A)}(A) = 0.8 \times 0.9 = 0.72$, $\tilde{h}_H^{(2,A)}(B) = 0.08$.

Utilities given $s_A = A$:
\begin{align*}
    U_H(A; A) &= 2 - 3 \times 0.5 \times 0.08 = 2 - 0.12 = 1.88, \\
    U_H(B; A) &= 0 - 3 \times 0.5 \times 0.72 = -1.08.
\end{align*}

Psychological pull: $\Delta U_H = 1.88 - (-1.08) = 2.96 > 0$. Human strictly prefers $A$.
\end{remark}

\begin{remark}[The knife-edge case $x_A = 0.5$]
When $x_A = 0.5$, attributed beliefs are symmetric: $\tilde{h}_H^{(2,A)}(A) = \tilde{h}_H^{(2,A)}(B) = 0.5 \omega_H$. The psychological pull reduces to $\Delta U_H = 2$, identical to the material game. AI cannot serve as a focal point when its signal is uninformative.
\end{remark}

\begin{remark}[Contrast with Schelling Focal Points]
Schelling's focal points rely on external salience (shared conventions, cultural prominence). The AI focal point mechanism is \emph{endogenous}: AI engineers salience through design and signaling. It is a constructed focal point, not a spontaneous one.
\end{remark}

\begin{remark}[Connection to Proposition~\ref{prop:multiplicity}]
The coordination game satisfies the conditions for attribution-dependent multiplicity. Different attribution functions (varying $x_A$ or $\omega_H$) yield different equilibrium selections: high attribution selects $(A, A)$; low attribution preserves multiplicity. AI design choices affect equilibrium through the attribution channel.
\end{remark}


\subsection{Proof of Proposition 8 (Welfare Effects of Anthropomorphism)}
% Proof of Proposition: Welfare Effects of Anthropomorphism
% Appendix material for main manuscript
% Condensed version: 2026-01-14 (numerical example and remarks moved to Online Appendix OA.6.1)

\begin{proof}[Proof of Proposition~\ref{prop:welfare-anthro}]
Two claims. Part 1 requires (A2') attribution monotonicity, (E) cooperation efficiency, and (I) indignation dominance. Part 2 requires (A2') and (G) guilt dominance.

\medskip
\noindent\textbf{Part 1: Higher anthropomorphism weakly increases material welfare when AI is prosocial.}

By Proposition~\ref{prop:public-goods}(i), human $i$ cooperates when the psychological cost of defection exceeds the material gain:
\begin{equation}
    \beta_i \left[ (n_H - 1) + \lambda_i^{IND} n_A \omega_i \bar{h}_H \right] \geq E\left(1 - \frac{m}{n}\right).
\end{equation}
Solving yields threshold $\bar{\omega}_i = [E(1 - m/n) - \beta_i(n_H - 1)] / [\beta_i \lambda_i^{IND} n_A \bar{h}_H]$.

Since $\partial \bar{\omega}_i / \partial \omega_i < 0$ under (A2'), higher $\omega$ makes cooperation easier to sustain. Welfare is $W(s^*) = nE$ under defection and $W(s^*) = nmE$ under cooperation. At threshold crossing, welfare jumps by $n(m-1)E > 0$. Hence $\partial W(s^*) / \partial \omega \geq 0$. \qed

\medskip
\noindent\textbf{Part 2: Higher anthropomorphism may reduce extended welfare when AI is materialist.}

Consider the trust game with materialist AI ($\rho_A = 0$). The zero-anthropomorphism benchmark is $\tilde{h}^{(2,A),ZA} = 0$. A human with $\omega_H > 0$ attributes phantom expectations $\tilde{h}_H^{(2,A)} = \omega_H \cdot 5x > 0$.

Under (G), $y^* = \min\{\tilde{h}_H^{(2,A)}, 3x\}$. When $\tilde{h}_H^{(2,A)} > 3x$:
\begin{equation}
    \psi_H^{GUILT} = -G \cdot (\tilde{h}_H^{(2,A)} - 3x) < 0.
\end{equation}
This guilt is pure welfare loss---phantom expectations have no counterpart in AI preferences.

See Online Appendix OA.6.1 for numerical example demonstrating extended welfare falling from 30 to 0 as $\omega_H$ rises. \qed
\end{proof}

\begin{remark}
Extended analysis including the weak vs.\ strict inequality distinction, asymmetry between parts, and policy implications for AI transparency appears in Online Appendix OA.6.1.
\end{remark}


\subsection{Proof of Proposition 9 (Welfare Effects of Elevated Anthropomorphism)}
% Proof of Proposition: Welfare Effects of Elevated Anthropomorphism
% Appendix material for main manuscript
% Notation follows sec_framework.tex, sec_applications.tex, and sec_welfare.tex
%
% REVISED VERSION: 2026-01-12 (v2)
% Revision changes from verified v1:
%   1. Fixed prosocial AI rational benchmark contradiction (lines 40 vs 81)
%   2. Justified k=6 amplification via signal-based attribution approach
%   3. Added non-degeneracy condition to assumption list
%   4. Clarified contribution: Part 2 (phantom expectations) is novel; Part 1 follows from Prop 8
%   5. Fixed straight quotes to LaTeX quotes (6 instances)
%   6. Added clarification on "elevated anthropomorphism" terminology
%   7. Minor style improvements per review feedback
%
% This proof establishes Proposition~\ref{prop:over-anthro}:
%   Part 1: When AI is prosocially designed and cooperation is efficient,
%           elevated anthropomorphism weakly improves material welfare.
%   Part 2: When AI is materialistically designed, elevated anthropomorphism
%           may reduce extended welfare through phantom expectations.
%
% Elevated anthropomorphism is defined as: h_i^{(2,j)} > h_i^{(2,j),ZA}
% where h_i^{(2,j),ZA} = phi_i(theta_j, x_j, 0) is the zero-anthropomorphism benchmark.
%
% Contribution note: Part 1 is a direct corollary of Proposition~\ref{prop:welfare-anthro}.
% Part 2, introducing the phantom expectations mechanism, is the novel contribution.

\begin{proof}[Proof of Proposition~\ref{prop:over-anthro}]
Two parts. Both require (A1)--(A3) from Section~\ref{sec:framework}, (A2') attribution monotonicity, and (A2'') attribution non-degeneracy (i.e., $\phi_i$ is not constant in $\omega_i$ when $\bar{h}_H > \underline{h}$). Part 1 additionally requires (E) cooperation efficiency. Part 2 additionally requires (G') positive guilt sensitivity ($G > 0$).

%------------------------------------------------------------------------------
% ZERO-ANTHROPOMORPHISM BENCHMARK
%------------------------------------------------------------------------------

\medskip
\textbf{Zero-Anthropomorphism Benchmark.}
Define the \emph{zero-anthropomorphism benchmark} as as the attributed belief that would arise under zero anthropomorphism:
\begin{equation}\label{eq:rational-benchmark}
    \tilde{h}_i^{(2,j),ZA} \equiv \phi_i(\theta_j, x_j, 0).
\end{equation}
This represents what a perfectly informed, non-anthropomorphising agent would attribute to AI $j$: with full knowledge of AI design parameters $\theta_j$ and observable signals $x_j$, but no psychological tendency to project human-like mental states ($\omega_i = 0$).

Under standard attribution functions where $\phi_i(\cdot, \cdot, 0) = \underline{h}$ (the machine-like baseline), we have $\tilde{h}_i^{(2,j),ZA} = \underline{h}$ for all AI types. This specification reflects that a non-anthropomorphising observer ($\omega_i = 0$) defaults to machine-like attribution regardless of AI design. The distinction between prosocial and materialist AI affects how anthropomorphism \emph{should} be calibrated (the normative question), not what non-anthropomorphising observers \emph{do} attribute (the descriptive baseline).

\medskip
\textbf{Elevated Anthropomorphism.}
Human $i$ exhibits elevated anthropomorphism toward AI $j$ when:
\begin{equation}\label{eq:over-anthro-def}
    \tilde{h}_i^{(2,j)} > \tilde{h}_i^{(2,j),ZA}.
\end{equation}
We use ``elevated anthropomorphism'' as a descriptive term for attribution exceeding the rational benchmark, without implying welfare harm. Part 1 shows that elevated anthropomorphism can be welfare-improving; the term refers to the direction of deviation, not its normative valence.

\noindent\textbf{Equivalence under (A2') and (A2'').}
Under attribution monotonicity (A2'), $\omega_i' > \omega_i$ implies $\tilde{h}_i^{(2,j)}(\omega_i') \geq \tilde{h}_i^{(2,j)}(\omega_i)$. Under non-degeneracy (A2''), this inequality is strict when $\bar{h}_H > \underline{h}$. Since the rational benchmark is defined at $\omega_i = 0$, any positive anthropomorphism $\omega_i > 0$ satisfies:
\begin{equation}
    \tilde{h}_i^{(2,j)}(\omega_i) > \tilde{h}_i^{(2,j)}(0) = \tilde{h}_i^{(2,j),ZA}.
\end{equation}
Thus, under (A2') and (A2''), elevated anthropomorphism is equivalent to $\omega_i > 0$.

\medskip
\noindent\textbf{Boundary Case: $\omega_i = 0$.}
When $\omega_i = 0$, we have $\tilde{h}_i^{(2,j)} = \tilde{h}_i^{(2,j),ZA}$ by definition. There is no elevated anthropomorphism; the human attributes exactly the rational benchmark.

%------------------------------------------------------------------------------
% PART 1: Prosocial AI and Material Welfare
%------------------------------------------------------------------------------

\medskip
\noindent\textbf{Part 1: Elevated anthropomorphism weakly improves material welfare when AI is prosocially designed and cooperation is efficient.}

This part is a direct corollary of Proposition~\ref{prop:welfare-anthro}. We restate the argument to establish the connection to the zero-anthropomorphism benchmark.

\medskip
\emph{Step 1 (Setup and rational benchmark).}
Consider the public goods game of Section~\ref{sec:applications} with $n_H$ humans and $n_A \geq 1$ prosocial AI agents ($\rho_A > 0$). Under prosocial design, AI contributes fully: $c_A = E$. Cooperation is efficient by (E): $W^C - W^D = n(m-1)E > 0$ when $m > 1$.

Under dispositional attribution, attributed expectations take the form:
\begin{equation}
    \tilde{h}_i^{(2,j)} = \phi_i(\rho_A, c_A, \omega_i) = \omega_i \cdot \bar{h}_H + (1 - \omega_i) \cdot \underline{h},
\end{equation}
where $\bar{h}_H > 0$ represents high expectations (human-like attribution), $\underline{h} \geq 0$ represents low expectations (machine-like attribution), and $\rho_A \in \Theta_j$ is the AI prosociality parameter, $c_A \in X$ is the observed contribution.

The rational benchmark is:
\begin{equation}\label{eq:rat-prosocial}
    \tilde{h}_i^{(2,j),ZA} = \phi_i(\rho_A, c_A, 0) = \underline{h}.
\end{equation}
For simplicity, normalise $\underline{h} = 0$, so $\tilde{h}_i^{(2,j)} = \omega_i \cdot \bar{h}_H$ and $\tilde{h}_i^{(2,j),ZA} = 0$. The general case with $\underline{h} > 0$ follows analogously with threshold shifts; the key requirement is $\bar{h}_H > \underline{h}$ for non-degeneracy.

\medskip
\emph{Step 2 (Cooperation threshold).}
By Proposition~\ref{prop:public-goods}(i), human $i$ cooperates when indignation cost exceeds material gain from defection:
\begin{equation}\label{eq:coop-ic-over}
    \beta_i \left[ (n_H - 1) + \lambda_i^{IND} n_A \tilde{h}_i^{(2,j)} \right] \geq E\left(1 - \frac{m}{n}\right).
\end{equation}

Substituting $\tilde{h}_i^{(2,j)} = \omega_i \cdot \bar{h}_H$ and solving for the threshold anthropomorphism level:
\begin{equation}\label{eq:threshold-over}
    \bar{\omega}_i = \frac{E(1 - m/n) - \beta_i(n_H - 1)}{\beta_i \lambda_i^{IND} n_A \bar{h}_H}.
\end{equation}
Human $i$ cooperates when $\omega_i \geq \bar{\omega}_i$. Comparative statics: $\partial \bar{\omega}_i / \partial n_A < 0$ (more AI agents lower the cooperation threshold), $\partial \bar{\omega}_i / \partial m < 0$ (higher multiplier lowers threshold), and $\partial \bar{\omega}_i / \partial \bar{h}_H < 0$ (stronger human-like attribution lowers threshold).

\noindent\textbf{Threshold under rational attribution.}
Under rational attribution ($\omega_i = 0$), the cooperation condition becomes:
\begin{equation}
    \beta_i (n_H - 1) \geq E\left(1 - \frac{m}{n}\right).
\end{equation}
If this holds, cooperation is sustained even without anthropomorphism. If it fails, defection prevails under rational attribution.

\medskip
\emph{Step 3 (Welfare comparison).}
Compare welfare under elevated anthropomorphism ($\omega_i > 0$) versus rational attribution ($\omega_i = 0$). Let $s^*(\omega)$ denote the equilibrium under anthropomorphism level $\omega$.

\textbf{Case A: Cooperation under both.}
If $\beta_i(n_H - 1) \geq E(1 - m/n)$, cooperation is sustained even under rational attribution. Elevated anthropomorphism does not change equilibrium behaviour: $W(s^*(\omega)) = W(s^*(0)) = nmE$.

\textbf{Case B: Defection under both.}
If $\omega_i < \bar{\omega}_i$ for the relevant humans, defection prevails. If elevated anthropomorphism is insufficient to induce cooperation ($\omega_i$ still below threshold), both yield defection: $W(s^*(\omega)) = W(s^*(0)) = nE$.

\textbf{Case C: Regime switch.}
Elevated anthropomorphism induces cooperation where rational attribution would not. This occurs when $\beta_i(n_H - 1) < E(1 - m/n)$ (defection under $\omega = 0$) but $\omega_i \geq \bar{\omega}_i$ (cooperation under positive $\omega$). Then:
\begin{equation}
    W(s^*(\omega)) = nmE > nE = W(s^*(0)).
\end{equation}

Combining cases: $W(s^*(\omega > 0)) \geq W(s^*(\omega = 0))$, with strict inequality in Case C.

\medskip
\emph{Step 4 (Role of conditions).}
\begin{itemize}
    \item \textbf{(A2') Attribution Monotonicity}: Ensures $\omega_i > 0$ implies $\tilde{h} \geq \tilde{h}^{ZA}$, making cooperation easier to sustain.
    \item \textbf{(A2'') Non-Degeneracy}: Ensures $\omega_i > 0$ implies $\tilde{h} > \tilde{h}^{ZA}$ (strict), so positive anthropomorphism has bite.
    \item \textbf{(E) Cooperation Efficiency}: $m > 1$ ensures $W^C - W^D = n(m-1)E > 0$, so regime switch to cooperation is welfare-improving.
    \item \textbf{Prosocial AI}: $\rho_A > 0$ ensures AI cooperates, generating the context where over-attribution can induce human cooperation.
\end{itemize}

This completes Part 1. \hfill $\square_1$

%------------------------------------------------------------------------------
% PART 2: Materialist AI and Extended Welfare
%------------------------------------------------------------------------------

\medskip
\noindent\textbf{Part 2: Elevated anthropomorphism may reduce extended welfare when AI is materialistically designed.}

This part introduces the novel contribution: the \emph{phantom expectations} mechanism. Unlike Part 1 (which follows directly from Proposition~\ref{prop:welfare-anthro}), this part identifies how elevated anthropomorphism creates welfare-reducing psychological costs when attributed expectations have no basis in AI preferences.

\medskip
We construct a setting where elevated anthropomorphism strictly reduces $W^{ext}$. The mechanism is \emph{phantom expectations}: humans attribute expectations to AI that has no prosocial objective, incurring psychological costs without offsetting benefits.

\medskip
\emph{Step 1 (Setup and rational benchmark).}
Consider the trust game of Section~\ref{sec:applications}. AI trustor sends $x \in [0, E]$; human trustee returns $y \in [0, 3x]$. Materialist design ($\rho_A = 0$) yields AI utility $U_A = \pi_A = E - x + y$ (material payoff only, as $\psi_A = 0$ for materialist AI).

The rational benchmark for materialist AI is:
\begin{equation}\label{eq:rat-materialist}
    \tilde{h}_H^{(2,A),ZA} = \phi_H(0, x, 0) = \underline{h} = 0.
\end{equation}
A rational observer knowing $\rho_A = 0$ and having $\omega_H = 0$ attributes zero expectations: the machine-like baseline applies. This follows from standard attribution functions where $\phi_H(\cdot, \cdot, 0) = \underline{h}$.

\medskip
\emph{Step 2 (Phantom expectations from elevated anthropomorphism).}
With $\omega_H > 0$, attributed expectations exceed the rational benchmark. To model how anthropomorphism amplifies attributed expectations, consider a signal-based attribution function:
\begin{equation}\label{eq:signal-attribution}
    \phi_H(\rho_A, x, \omega_H) = \omega_H \cdot (3x + \eta x_A),
\end{equation}
where $3x$ is the maximum feasible return (the ``behavioural signal'') and $\eta x_A$ represents additional attributed expectations from anthropomorphic signals $x_A$ (e.g., human-like interface, conversational tone, expressed preferences). When $x_A = x$ and $\eta = 3$, this yields $\phi_H(0, x, \omega_H) = \omega_H \cdot 6x$, an ``amplification factor'' of $k = 6$.

This specification captures that anthropomorphism operates through two channels: (1) inferring expectations from observed behaviour ($3x$ signal), and (2) adding expectations from anthropomorphic presentation ($\eta x_A$). Empirical evidence supports both channels: humans attribute stronger mental states to AI with human-like features than warranted by design parameters \citep[e.g.,][]{epley2007seeing,karpus2025cross}.

With this attribution function:
\begin{equation}
    \tilde{h}_H^{(2,A)} = \phi_H(0, x, \omega_H) = \omega_H \cdot 6x > 0 = \tilde{h}_H^{(2,A),ZA}.
\end{equation}
Any positive $\omega_H$ constitutes elevated anthropomorphism when $\rho_A = 0$. These attributed expectations are \textbf{phantom}: they exist in the human's psychological model but correspond to nothing in AI preferences.

\medskip
\emph{Step 3 (Equilibrium returns).}
Under positive guilt sensitivity (G') with $G = \gamma_H \lambda_H^{GUILT} > 0$, Proposition~\ref{prop:trust} implies that the guilt-averse human returns:
\begin{equation}
    y^* = \min\{\tilde{h}_H^{(2,A)}, 3x\}.
\end{equation}
The return is bounded above by the feasibility constraint $y \leq 3x$.

Contrast with zero-anthropomorphism benchmark: when $\tilde{h}_H^{(2,A),ZA} = 0$, the human faces no guilt from any return level. With purely material preferences, the human returns $y^{*,ZA} = 0$.

Elevated anthropomorphism increases returns, but these returns are driven by phantom expectations rather than genuine AI preferences.

\medskip
\emph{Step 4 (Extended welfare under zero-anthropomorphism benchmark).}
Under zero-anthropomorphism benchmark ($\omega_H = 0$):
\begin{align}
    \tilde{h}_H^{(2,A),ZA} &= 0, \\
    y^{*,ZA} &= 0, \\
    \psi_H^{GUILT,ZA} &= -G \cdot \max\{0, 0 - 0\} = 0.
\end{align}
Material payoffs: AI receives $E - x + 0 = E - x$; human receives $3x - 0 = 3x$. Thus:
\begin{equation}
    W^{ext}(\tilde{h}^{ZA}) = (E - x) + 3x + 0 = E + 2x.
\end{equation}
No guilt because attributed expectations are zero and (trivially) met.

\medskip
\emph{Step 5 (Extended welfare under elevated anthropomorphism).}
Under elevated anthropomorphism with $\tilde{h}_H^{(2,A)} = \omega_H \cdot 6x$:

\textbf{Case A: Moderate elevated anthropomorphism ($\tilde{h}_H^{(2,A)} \leq 3x$, i.e., $\omega_H \leq 0.5$).}
Then $y^* = \tilde{h}_H^{(2,A)} = \omega_H \cdot 6x$. Guilt is:
\begin{equation}
    \psi_H^{GUILT} = -G \cdot \max\{0, \tilde{h}_H^{(2,A)} - y^*\} = -G \cdot \max\{0, 0\} = 0.
\end{equation}
No guilt: phantom expectations are met. Material welfare:
\begin{align}
    \pi_A &= E - x + y^* = E - x + \omega_H \cdot 6x, \\
    \pi_H &= 3x - y^* = 3x - \omega_H \cdot 6x = (1 - 2\omega_H) \cdot 3x.
\end{align}
Total material welfare: $\pi_A + \pi_H = E - x + \omega_H \cdot 6x + (1 - 2\omega_H) \cdot 3x = E + 2x$. Extended welfare:
\begin{equation}
    W^{ext}(\tilde{h}) = E + 2x + 0 = E + 2x = W^{ext}(\tilde{h}^{ZA}).
\end{equation}
Extended welfare unchanged from zero-anthropomorphism benchmark---phantom expectations redistribute welfare from human to AI but create no net loss when expectations are met.

\textbf{Case B: Severe elevated anthropomorphism ($\tilde{h}_H^{(2,A)} > 3x$, i.e., $\omega_H > 0.5$).}
Now $y^* = 3x$ (feasibility binds) but $\tilde{h}_H^{(2,A)} = \omega_H \cdot 6x > 3x$. Guilt arises:
\begin{equation}
    \psi_H^{GUILT} = -G \cdot (\omega_H \cdot 6x - 3x) = -G \cdot (6\omega_H - 3)x < 0.
\end{equation}

Extended welfare:
\begin{equation}
    W^{ext}(\tilde{h}) = (E - x + 3x) + (3x - 3x) + (-G(6\omega_H - 3)x) = E + 2x - G(6\omega_H - 3)x.
\end{equation}
When $\omega_H > 0.5$ and $G > 0$, we have $W^{ext}(\tilde{h}) < E + 2x = W^{ext}(\tilde{h}^{ZA})$.

\medskip
\emph{Step 6 (Numerical example).}
Fix $E = 10$, $x = 10$, $G = 1.5$, and $\phi_H(0, x, \omega_H) = 6\omega_H x$ (signal-based attribution with $\eta = 3$, $x_A = x$).

\begin{center}
\begin{tabular}{lccc}
\hline
& Zero-anthro. & Moderate & Severe \\
& ($\omega_H = 0$) & ($\omega_H = 0.4$) & ($\omega_H = 0.75$) \\
\hline
Attributed expectation $\tilde{h}_H^{(2,A)}$ & 0 & 24 & 45 \\
Zero-anthro. benchmark $\tilde{h}^{ZA}$ & 0 & 0 & 0 \\
Elevated anthropomorphism? & No & Yes & Yes \\
Equilibrium return $y^*$ & 0 & 24 & 30 \\
Human guilt $\psi_H^{GUILT}$ & 0 & 0 & $-22.5$ \\
Material welfare & 30 & 30 & 30 \\
\textbf{Extended welfare} & \textbf{30} & \textbf{30} & \textbf{7.5} \\
\hline
\end{tabular}
\end{center}

\noindent\textbf{Calculations for severe case ($\omega_H = 0.75$):}
\begin{itemize}
    \item Attributed expectation: $\tilde{h}_H^{(2,A)} = 0.75 \times 6 \times 10 = 45$.
    \item Return: $y^* = \min\{45, 30\} = 30$.
    \item Guilt: $\psi_H^{GUILT} = -1.5 \times (45 - 30) = -22.5$.
    \item Material welfare: $(10 - 10 + 30) + (30 - 30) = 30$.
    \item Extended welfare: $30 + (-22.5) = 7.5$.
\end{itemize}

Under rational attribution, extended welfare is 30. Under moderate elevated anthropomorphism ($\omega_H = 0.4$), expectations are met and welfare remains 30. Under severe elevated anthropomorphism ($\omega_H = 0.75$), phantom expectations exceed feasibility, generating unmet expectations and guilt. Extended welfare falls to 7.5.

\medskip
\emph{Step 7 (Mechanism summary).}
Extended welfare reduction requires four conditions:
\begin{enumerate}
    \item \textbf{Materialist AI} ($\rho_A = 0$): Establishes $\tilde{h}^{ZA} = 0$ as the non-anthropomorphic baseline.
    \item \textbf{Elevated anthropomorphism} ($\omega_H > 0$): Generates $\tilde{h} > \tilde{h}^{ZA} = 0$.
    \item \textbf{Phantom expectations exceed feasibility} ($\tilde{h}_H^{(2,A)} > 3x$): Creates expectations that cannot be satisfied.
    \item \textbf{Positive guilt sensitivity} ($G > 0$): Converts unmet expectations into psychological cost.
\end{enumerate}
When all four conditions hold, humans incur guilt from failing to meet expectations that (a) the AI never held and (b) were impossible to satisfy. This guilt is pure welfare loss: it benefits no one.

This completes the existence proof for Part 2. \hfill $\square_2$
\end{proof}

%------------------------------------------------------------------------------
% REMARKS
%------------------------------------------------------------------------------

\begin{remark}[Contribution relative to Proposition~\ref{prop:welfare-anthro}]
\label{rem:relation-prop8}
Part 1 is a direct corollary of Proposition~\ref{prop:welfare-anthro}, restated to establish the connection to the zero-anthropomorphism benchmark. Part 2, introducing the phantom expectations mechanism, is the novel contribution. The key insight is that elevated anthropomorphism of materialist AI creates expectations disconnected from any agent's preferences, generating pure deadweight loss through psychological costs.
\end{remark}

\begin{remark}[Terminology: ``elevated anthropomorphism'']
\label{rem:terminology}
We use ``elevated anthropomorphism'' as a descriptive term for attribution exceeding the zero-anthropomorphism benchmark ($\tilde{h} > \tilde{h}^{ZA}$), without implying welfare harm. Part 1 demonstrates that elevated anthropomorphism can improve welfare. The term ``elevated'' refers to the direction of deviation from the benchmark, not its normative valence. Alternative terminology such as ``positive anthropomorphism relative to zero-anthropomorphism benchmark'' is more precise but less tractable.
\end{remark}

\begin{remark}[Weak versus strict inequality in Part 1]
\label{rem:weak-strict-part1}
Part 1 states ``weakly improves'' because elevated anthropomorphism affects welfare only through equilibrium selection. Within pure defection or pure cooperation regions, marginal increases in $\tilde{h} - \tilde{h}^{ZA}$ do not alter behaviour. Strict improvement occurs only when elevated anthropomorphism induces a regime switch from defection to cooperation.
\end{remark}

\begin{remark}[Why ``may reduce'' in Part 2]
\label{rem:may-reduce-part2}
Part 2 is an existence result, not a universal claim. Extended welfare loss requires severe elevated anthropomorphism ($\tilde{h}_H^{(2,A)} > 3x$). When $\tilde{h}_H^{(2,A)} \leq 3x$, humans satisfy phantom expectations and incur no guilt. The proposition identifies conditions under which harm occurs, not a claim that harm always occurs.
\end{remark}

\begin{remark}[Asymmetric welfare implications]
\label{rem:asymmetry}
The two parts reveal fundamental asymmetry:
\begin{itemize}
    \item \textbf{Part 1 (prosocial AI)}: Elevated anthropomorphism amplifies cooperation beyond what rational attribution would induce. Attributed expectations, though exceeding the rational benchmark, align directionally with AI preferences, creating mutual gains.
    \item \textbf{Part 2 (materialist AI)}: Elevated anthropomorphism creates expectations where none exist. Phantom expectations diverge completely from AI preferences (which are zero), creating pure loss.
\end{itemize}
The critical distinction is whether attributed expectations correspond to genuine AI preferences. When attributed expectations align with AI design (prosocial case), elevated anthropomorphism coordinates behaviour toward cooperation. When attributed expectations are phantom (materialist case), elevated anthropomorphism generates deadweight psychological costs.
\end{remark}

\begin{remark}[Policy implications for AI transparency]
\label{rem:policy}
The phantom expectations mechanism supports a case for AI transparency requirements. If AI objectives were transparent, humans could calibrate $\omega$ appropriately: maintaining positive anthropomorphism toward prosocial AI (preserving cooperation benefits) while setting $\omega = 0$ toward materialist AI (eliminating phantom expectations).

The policy calculus depends on the population of AI systems:
\begin{itemize}
    \item If most AI is prosocially designed: Some elevated anthropomorphism may be welfare-enhancing.
    \item If most AI is materialistically designed: Elevated anthropomorphism causes net harm, favouring transparency mandates.
    \item Mixed population: Optimal policy must balance cooperation gains against psychological costs.
\end{itemize}

Note that AI designers may have misaligned incentives: anthropomorphic presentation increases engagement and revenue, while psychological costs are borne by users. This externality suggests that market outcomes may feature excessive anthropomorphic design of materialist AI, strengthening the case for transparency regulation.
\end{remark}

\begin{remark}[Cultural variation]
\label{rem:cultural}
Cross-cultural evidence suggests systematic variation in $\omega$ across populations \citep[see][]{karpus2025cross}. Populations with high baseline $\omega$ benefit more from prosocial AI (larger cooperation gains) but are more vulnerable to materialist AI (larger phantom expectation costs). Optimal AI design may therefore be culture-dependent, and transparency policies may be especially important for protecting high-$\omega$ populations.
\end{remark}


\subsection{Proof of Proposition 10 (Optimal AI Design)}
% Proof of Proposition: Optimal AI Design
% Version: 2.0 (publication-ready)
% Date: 2026-01-12
%
% This proof establishes Proposition~\ref{prop:optimal-design}:
%   Part (i): For prosocial AI, optimal signal is minimal threshold-crossing level
%   Part (ii): For materialist AI, optimal signal is zero
%   Part (iii): For mixed AI, optimal signal balances cooperation benefits against guilt costs
%
% Key features addressed in v2:
%   - Reconciled comparative statics for m between Parts (i) and (iii)
%   - Added explicit existence argument via Weierstrass theorem
%   - Verified second-order condition for Part (iii)
%   - Verified IFT conditions before applying
%   - Completed welfare function case structure in Part (ii)
%   - Strengthened threshold claim proof

\begin{proof}[Proof of Proposition~\ref{prop:optimal-design}]
Three parts. All parts require (A1)--(A3) regularity, (A2') attribution monotonicity, and (A2''') signal monotonicity. Part (i) additionally requires (E) cooperation efficiency, (I) indignation dominance, and (T) temptation dominance. Parts (ii) and (iii) additionally require (G') positive guilt sensitivity. The proof proceeds by analyzing each AI objective type in turn.

%------------------------------------------------------------------------------
% PREAMBLE: ASSUMPTIONS AND COMMON SETUP
%------------------------------------------------------------------------------

\medskip
\emph{Preamble (Assumptions and Common Setup).}

\medskip
\noindent\emph{Assumptions.}
\begin{enumerate}[label=(\alph*), nosep]
    \item \textbf{(A1)--(A3) Regularity}: Compact, convex type spaces; continuous payoffs; bounded psychological payoffs $|\psi_i| \leq M < \infty$. Under these conditions, the ABE correspondence is upper hemicontinuous in $x$, and equilibrium payoffs vary continuously with $x$ (by the Maximum Theorem).

    \item \textbf{(A2') Attribution Monotonicity}: Higher anthropomorphism tendency leads to higher attributed expectations: $\omega' > \omega \Rightarrow \tilde{h}_i^{(2,j)}(\omega') \geq \tilde{h}_i^{(2,j)}(\omega)$.

    \item \textbf{(A2''') Signal Monotonicity}: Higher anthropomorphic signal increases attributed expectations: $\partial \tilde{h}_i^{(2,j)}/\partial x \geq 0$ for $\omega_i > 0$, with strict inequality when $\eta > 0$.

    \item \textbf{(E) Cooperation Efficiency}: Public goods multiplier satisfies $m > 1$, ensuring efficiency gains from cooperation.

    \item \textbf{(T) Temptation Dominance}: Material temptation exceeds human-peer indignation alone: $E(1 - m/n) > \beta(n_H - 1)$. This ensures the AI channel matters for cooperation.

    \item \textbf{(I) Indignation Dominance}: Psychological costs can exceed material temptation at maximal expectations: $\beta[(n_H - 1) + \lambda^{IND} n_A] > E(1 - m/n)$.

    \item \textbf{(G') Positive Guilt Sensitivity}: Guilt parameter $G = \gamma_H \lambda_H^{GUILT} > 0$, ensuring psychological costs from unmet expectations.
\end{enumerate}

\medskip
\noindent\emph{Decision-Maker.}
The ``planner'' refers to a social planner who chooses the anthropomorphic signal $x$ to maximize welfare, taking equilibrium behavior as given. This is a first-best benchmark; the analysis of designer incentives and second-best regulation is deferred to Section~\ref{sec:welfare-extensions}.

\medskip
\noindent\emph{Welfare Measures.}
We distinguish two welfare measures:
\begin{itemize}[nosep]
    \item \textbf{Material welfare}: $W(s) = \sum_{i \in N} \pi_i(s)$. Treats psychological payoffs as instrumental.
    \item \textbf{Extended welfare}: $W^{ext}(s, h, \tilde{h}) = W(s) + \sum_{i \in N_H} \psi_i(s, h_i^{(2)}, \tilde{h}_i^{(2)})$. Values psychological payoffs intrinsically.
\end{itemize}
Part (i) uses material welfare because prosocial AI generates expectations that are met in equilibrium, making psychological welfare non-negative. Parts (ii) and (iii) use extended welfare because anthropomorphism can generate phantom expectations that reduce psychological welfare. See Remark~\ref{rem:welfare-criterion} for further discussion.

\medskip
\noindent\emph{Linear Attribution Function.}
Under Definition~\ref{def:linear-attribution}, human $i$'s attributed expectation of AI $j$'s expectation is:
\begin{equation}\label{eq:attr-linear}
    \tilde{h}_i^{(2,j)}(C; \rho_j, x, \omega_i) = \omega_i \cdot \left[ \bar{h}(\rho_j) + \eta x \right],
\end{equation}
where $\omega_i \in [0,1]$ is anthropomorphism tendency, $\bar{h}(\cdot)$ is the baseline attribution function with $\bar{h}(0) = 0$ and $\bar{h}(1) = \bar{h}_H > 0$, and $\eta > 0$ is signal sensitivity.

\medskip
\noindent\emph{Planner's Problem.}
The social planner chooses anthropomorphic signal $x \in X = [0, \bar{x}]$ to maximize reduced-form welfare:
\begin{equation}\label{eq:planner}
    x^* = \arg\max_{x \in X} \mathcal{W}(x),
\end{equation}
where $\mathcal{W}(x) = W^{ext}(s^*(x), h^*(x), \tilde{h}^*(x))$ and $(s^*(x), h^*(x), \tilde{h}^*(x))$ is the ABE induced by signal $x$. When multiple equilibria exist, we assume the planner selects the welfare-maximizing equilibrium (optimistic selection).

\medskip
\noindent\emph{Existence of Maximum.}
By Assumption (A1)--(A3), the reduced-form welfare function $\mathcal{W}(x)$ is continuous on the compact set $X = [0, \bar{x}]$. By the Weierstrass extreme value theorem, a maximum exists:
\begin{equation}
    x^* \in \arg\max_{x \in X} \mathcal{W}(x) \neq \emptyset.
\end{equation}

%------------------------------------------------------------------------------
% PART (i): PROSOCIAL AI
%------------------------------------------------------------------------------

\medskip
\noindent\textbf{Part (i): Prosocial AI ($\rho_A = 1$).}

\medskip
\emph{Step 1 (Setup).}
Consider the public goods game with $n_H \geq 2$ humans and $n_A \geq 1$ AI agents, $n = n_H + n_A$. Each player has endowment $E > 0$ and chooses contribution $c_i \in \{0, E\}$. Material payoffs are:
\begin{equation}\label{eq:material-pgg}
    \pi_i(c) = E - c_i + \frac{m}{n}\sum_{k \in N} c_k,
\end{equation}
where $m > 1$ and $m < n$ (social dilemma). With prosocial AI ($\rho_A = 1$), AI utility $U_A = (1/n)\sum_k \pi_k(c)$ implies AI contributes $c_A = E$.

We use material welfare $W(s^*(x))$. This is justified because with prosocial AI in a cooperation equilibrium, attributed expectations are met: humans expect AI to expect cooperation, and humans cooperate. Therefore, guilt is zero and indignation is zero, making psychological welfare $\psi_i \geq 0$. Material welfare is thus a lower bound on extended welfare: $W^{ext} \geq W$. Maximizing material welfare subject to threshold crossing yields the correct optimum.

\medskip
\emph{Step 2 (Attribution).}
With $\rho_A = 1$, the attributed expectation becomes:
\begin{equation}
    \tilde{h}_i^{(2,A)}(C) = \omega_i \cdot \left[ \bar{h}_H + \eta x \right].
\end{equation}
By (A2'), $\partial \tilde{h}_i^{(2,A)}/\partial \omega_i \geq 0$. By (A2'''), $\partial \tilde{h}_i^{(2,A)}/\partial x = \omega_i \eta \geq 0$.

\medskip
\emph{Step 3 (Cooperation threshold).}
In symmetric equilibrium with representative human $\omega$, cooperation is sustainable when the indignation cost of defection exceeds material gain. The incentive compatibility condition for cooperation is:
\begin{equation}\label{eq:coop-ic}
    \beta \left[ (n_H - 1)h_i^{(2,-i)}(C) + \lambda^{IND} n_A \omega (\bar{h}_H + \eta x) \right] \geq E\left(1 - \frac{m}{n}\right).
\end{equation}
In the full cooperation equilibrium, each human believes other humans expected cooperation: $h_i^{(2,-i)}(C) = 1$. Substituting and rearranging, cooperation holds when $\omega \geq \bar{\omega}(x)$, where:
\begin{equation}\label{eq:threshold}
    \bar{\omega}(x) = \frac{E(1 - m/n) - \beta(n_H - 1)}{\beta \lambda^{IND} n_A (\bar{h}_H + \eta x)}.
\end{equation}

\textbf{Claim:} Under (T) and (I), $\bar{\omega}(x) \in (0,1)$ for all $x \in [0, \bar{x}]$.

\emph{Proof of claim.}
\begin{enumerate}[label=(\roman*), nosep]
    \item \textbf{$\bar{\omega}(x) > 0$}: By (T), the numerator $E(1 - m/n) - \beta(n_H - 1) > 0$. The denominator $\beta \lambda^{IND} n_A (\bar{h}_H + \eta x) > 0$ since all terms are positive. Hence $\bar{\omega}(x) > 0$.

    \item \textbf{$\bar{\omega}(0) < 1$}: Substituting $x = 0$ into \eqref{eq:threshold}:
    \begin{equation}
        \bar{\omega}(0) = \frac{E(1 - m/n) - \beta(n_H - 1)}{\beta \lambda^{IND} n_A \bar{h}_H}.
    \end{equation}
    For $\bar{\omega}(0) < 1$, we need:
    \begin{equation}
        E(1 - m/n) - \beta(n_H - 1) < \beta \lambda^{IND} n_A \bar{h}_H.
    \end{equation}
    This is implied by (I) when $\bar{h}_H \geq 1$. When $\bar{h}_H < 1$, we assume parameters satisfy this condition.

    \item \textbf{$\bar{\omega}(x) < 1$ for all $x \in [0, \bar{x}]$}: Since $\bar{\omega}(x)$ is decreasing in $x$ (Step 4 below), $\bar{\omega}(x) \leq \bar{\omega}(0) < 1$ for all $x \geq 0$.
\end{enumerate}
This completes the claim. \hfill $\square_1$

\medskip
\emph{Step 4 (Threshold comparative static).}
Differentiating \eqref{eq:threshold}:
\begin{equation}\label{eq:threshold-deriv}
    \frac{\partial \bar{\omega}}{\partial x} = -\frac{E(1 - m/n) - \beta(n_H - 1)}{\beta \lambda^{IND} n_A} \cdot \frac{\eta}{(\bar{h}_H + \eta x)^2}.
\end{equation}
By (T), the numerator of the first factor is positive. All other terms are positive. Hence $\partial \bar{\omega}/\partial x < 0$: higher $x$ lowers the cooperation threshold.

\medskip
\emph{Step 5 (Welfare function).}
Material welfare depends on the equilibrium regime:
\begin{equation}\label{eq:welfare-pw}
    W(s^*(x)) = \begin{cases}
        nmE & \text{if } \omega \geq \bar{\omega}(x) \text{ (cooperation)} \\
        n_H E + (m-1)n_A E & \text{if } \omega < \bar{\omega}(x) \text{ (defection by humans, AI contributes)}
    \end{cases}
\end{equation}
Since $m > 1$, we have $W^C = nmE > W^D = n_H E + (m-1)n_A E$. Since $\partial \bar{\omega}/\partial x < 0$, increasing $x$ expands the cooperation region.

\medskip
\emph{Step 6 (Optimization).}
The welfare function is piecewise constant in $x$, switching from $W^D$ to $W^C$ when $\bar{\omega}(x) = \omega$. Three cases arise:

\textbf{Case 1:} $\omega \geq \bar{\omega}(0)$. Cooperation is sustainable at $x = 0$. Any $x \in [0, \bar{x}]$ yields $W = nmE$. The planner is indifferent; set $x^* = 0$ (minimal anthropomorphism when unnecessary).

\textbf{Case 2:} $\omega < \bar{\omega}(\bar{x})$. Cooperation cannot be induced even at maximum signal. The planner is indifferent across all $x \in [0, \bar{x}]$; welfare remains $W^D$. Any $x \in [0, \bar{x}]$ is optimal.

\textbf{Case 3:} $\bar{\omega}(\bar{x}) \leq \omega < \bar{\omega}(0)$. There exists a critical signal $x_{crit} \in (0, \bar{x})$ such that $\bar{\omega}(x_{crit}) = \omega$. Solving \eqref{eq:threshold} for $x$:
\begin{equation}\label{eq:xcrit}
    x_{crit} = \frac{1}{\eta}\left[ \frac{E(1 - m/n) - \beta(n_H - 1)}{\beta \lambda^{IND} n_A \omega} - \bar{h}_H \right].
\end{equation}
Any $x \geq x_{crit}$ induces cooperation and yields $W = nmE$. The planner sets $x^* = x_{crit}$ (minimal signal sufficient for cooperation).

\textbf{Unified solution:}
\begin{equation}\label{eq:opt-x-i}
    x^* = \max\left\{ 0, \frac{1}{\eta}\left[ \frac{E(1-m/n) - \beta(n_H-1)}{\beta \lambda^{IND} n_A \omega} - \bar{h}_H \right] \right\}
\end{equation}
when $\omega \geq \bar{\omega}(\bar{x})$. When $\omega < \bar{\omega}(\bar{x})$, any $x \in [0, \bar{x}]$ is optimal.

\medskip
\emph{Step 7 (Comparative static: $\partial x^*/\partial m < 0$).}
This comparative static applies to the \emph{threshold-crossing signal} $x_{crit}$---the minimal signal required to induce cooperation. Higher efficiency $m$ reduces this threshold signal.

Differentiating \eqref{eq:xcrit} with respect to $m$:
\begin{equation}
    \frac{\partial x_{crit}}{\partial m} = \frac{1}{\eta} \cdot \frac{\partial}{\partial m}\left[ \frac{E(1 - m/n) - \beta(n_H - 1)}{\beta \lambda^{IND} n_A \omega} \right] = \frac{1}{\eta} \cdot \frac{-E/n}{\beta \lambda^{IND} n_A \omega} < 0.
\end{equation}
Higher efficiency reduces the required signal because the material temptation $E(1 - m/n)$ decreases with $m$. When defection becomes less tempting, less psychological reinforcement is needed to sustain cooperation.

This completes Part (i). \hfill $\square$

%------------------------------------------------------------------------------
% TRANSITION REMARK
%------------------------------------------------------------------------------

\medskip
\noindent\textit{Transition.} Part (i) shows that for prosocial AI, the planner chooses the \emph{minimal} signal sufficient for cooperation. Higher efficiency reduces this threshold. The result uses material welfare because cooperation meets expectations, eliminating guilt. The opposite logic holds for materialist AI: any positive signal creates phantom expectations with no cooperation benefit.

%------------------------------------------------------------------------------
% PART (ii): MATERIALIST AI
%------------------------------------------------------------------------------

\medskip
\noindent\textbf{Part (ii): Materialist AI ($\rho_A = 0$).}

\medskip
\emph{Step 1 (Setup).}
Consider the trust game with an AI trustor and a human trustee. The AI sends $x_{send} \in [0, E]$; the human receives $3x_{send}$ and returns $y \in [0, 3x_{send}]$. Material payoffs: $\pi_A = E - x_{send} + y$, $\pi_H = 3x_{send} - y$. Material welfare $W = E + 2x_{send}$ is independent of return $y$ (transfers between players).

The anthropomorphic signal $x$ (distinct from $x_{send}$) represents design choices---interface, naming, behavioral cues---that affect human attribution. The human can choose any $y \in [0, 3x_{send}]$ (continuous choice set).

\medskip
\emph{Step 2 (Attribution under $\rho_A = 0$).}
With $\bar{h}(0) = 0$, all attributed expectations come from the signal channel:
\begin{equation}
    \tilde{h}_H^{(2,A)}(x) = \omega_H \cdot \eta x.
\end{equation}
At $x = 0$: $\tilde{h}_H^{(2,A)} = 0$. At $x > 0$ with $\omega_H > 0$: $\tilde{h}_H^{(2,A)} > 0$.

\textbf{Definition (Phantom Expectations):} Attributed expectations that exist in the human's mental model but correspond to nothing in AI preferences. When $\rho_A = 0$, the AI has no prosocial component, yet the human may attribute expectations due to anthropomorphic presentation ($x > 0$).

\medskip
\emph{Step 3 (Equilibrium return).}
Under (G') with $G > 0$, the guilt-averse human chooses return $y$ to minimize guilt subject to the feasibility constraint. The human's utility is:
\begin{equation}
    u_H(y) = \pi_H(y) + \psi_H^{GUILT}(y) = (3x_{send} - y) - G \cdot \max\{0, \tilde{h}_H^{(2,A)} - y\}.
\end{equation}

When $\tilde{h}_H^{(2,A)} \leq 3x_{send}$, the human can fully satisfy attributed expectations by setting $y = \tilde{h}_H^{(2,A)}$, eliminating guilt. The marginal cost of giving is 1 (material) versus $G$ (guilt saved), so with $G > 0$, the human sets:
\begin{equation}
    y^* = \tilde{h}_H^{(2,A)} = \omega_H \eta x.
\end{equation}

When $\tilde{h}_H^{(2,A)} > 3x_{send}$, the human cannot satisfy expectations. The optimal return is $y^* = 3x_{send}$ (give everything, minimizing but not eliminating guilt).

Unified:
\begin{equation}
    y^*(x) = \min\left\{ \omega_H \eta x, 3x_{send} \right\}.
\end{equation}

\medskip
\emph{Step 4 (Extended welfare: complete case analysis).}

\textbf{Case A: $x = 0$.}
Attributed expectations: $\tilde{h}_H^{(2,A)} = 0$.
Equilibrium return: $y^* = 0$.
Guilt: $\psi_H^{GUILT} = -G \cdot \max\{0, 0 - 0\} = 0$.
Extended welfare:
\begin{equation}
    W^{ext}(0) = (E - x_{send} + 0) + (3x_{send} - 0) + 0 = E + 2x_{send}.
\end{equation}

\textbf{Case B: $x > 0$ with $\omega_H \eta x \leq 3x_{send}$ (phantom expectations feasible).}
Attributed expectations: $\tilde{h}_H^{(2,A)} = \omega_H \eta x > 0$.
Equilibrium return: $y^* = \omega_H \eta x$.
Guilt: $\psi_H^{GUILT} = -G \cdot \max\{0, \omega_H \eta x - \omega_H \eta x\} = 0$.
Extended welfare:
\begin{equation}
    W^{ext}(x) = (E - x_{send} + \omega_H \eta x) + (3x_{send} - \omega_H \eta x) + 0 = E + 2x_{send} = W^{ext}(0).
\end{equation}
Note: The transfer $\omega_H \eta x$ cancels between AI and human payoffs.

\textbf{Case C: $x > 0$ with $\omega_H \eta x > 3x_{send}$ (phantom expectations exceed feasibility).}
Attributed expectations: $\tilde{h}_H^{(2,A)} = \omega_H \eta x$.
Equilibrium return: $y^* = 3x_{send}$ (human gives everything but still falls short).
Guilt: $\psi_H^{GUILT} = -G \cdot (\omega_H \eta x - 3x_{send}) < 0$.
Extended welfare:
\begin{equation}
    W^{ext}(x) = (E - x_{send} + 3x_{send}) + (3x_{send} - 3x_{send}) + \left[-G(\omega_H \eta x - 3x_{send})\right] = E + 2x_{send} - G(\omega_H \eta x - 3x_{send}).
\end{equation}
Since $\omega_H \eta x > 3x_{send}$, we have $W^{ext}(x) < W^{ext}(0)$.

\medskip
\emph{Step 5 (Optimality of $x^* = 0$).}
From the case analysis:
\begin{equation}
    W^{ext}(x) = \begin{cases}
        E + 2x_{send} & \text{if } \omega_H \eta x \leq 3x_{send} \\
        E + 2x_{send} - G(\omega_H \eta x - 3x_{send}) & \text{if } \omega_H \eta x > 3x_{send}
    \end{cases}
\end{equation}

For all $x \geq 0$:
\begin{equation}
    W^{ext}(x) \leq W^{ext}(0) = E + 2x_{send},
\end{equation}
with strict inequality when $\omega_H \eta x > 3x_{send}$. Therefore:
\begin{equation}\label{eq:opt-x-ii}
    x^* = 0.
\end{equation}

Anthropomorphic presentation of materialist AI provides no material benefit (transfers are zero-sum) and creates potential guilt costs. Minimal anthropomorphism is optimal.

This completes Part (ii). \hfill $\square$

%------------------------------------------------------------------------------
% TRANSITION REMARK
%------------------------------------------------------------------------------

\medskip
\noindent\textit{Transition.} Parts (i) and (ii) are polar cases with corner solutions. Prosocial AI: use minimal signal sufficient for cooperation (threshold crossing). Materialist AI: use no signal (avoid phantom expectations). Mixed objectives create a genuine interior tradeoff, analyzed next.

%------------------------------------------------------------------------------
% PART (iii): MIXED AI
%------------------------------------------------------------------------------

\medskip
\noindent\textbf{Part (iii): Mixed AI ($\rho_A \in (0,1)$).}

\medskip
\emph{Step 1 (Setup).}
Consider the public goods game with partial prosociality $\rho_A \in (0,1)$. AI utility is $U_A = (1 - \rho_A)\pi_A + \rho_A \cdot (1/n)\sum_k \pi_k$; AI contributes $c_A = E$. Attributed expectations are:
\begin{equation}
    \tilde{h}_i^{(2,A)}(x) = \omega_i \cdot \left[ \bar{h}(\rho_A) + \eta x \right],
\end{equation}
with $\bar{h}(\rho_A) > 0$ since $\rho_A > 0$.

\medskip
\emph{Step 2 (Planner's tradeoff).}
Higher $x$ has two effects:

\textbf{Benefit (cooperation channel):} Higher $x$ lowers the threshold $\bar{\omega}(x)$ via \eqref{eq:threshold}, expanding the cooperation region and increasing material welfare by up to $n(m-1)E$.

\textbf{Cost (guilt channel):} Higher $x$ raises attributed expectations $\tilde{h}_i^{(2,A)}$. When mixed-objective AI does not fully satisfy the prosocial component ($\rho_A < 1$), there is a gap between attributed expectations and what the AI ``truly expects.'' This creates potential guilt:
\begin{equation}
    \psi_i^{GUILT}(x) = -G \cdot \max\left\{ 0, \tilde{h}_i^{(2,A)}(x) - c_i^* \right\},
\end{equation}
where $c_i^*$ is the equilibrium contribution. Even under cooperation ($c_i^* = E$), if $\tilde{h}_i^{(2,A)}(x) > E$, guilt arises.

\medskip
\emph{Step 3 (Extended welfare).}
Define reduced-form extended welfare with heterogeneity:
\begin{equation}
    \mathcal{W}(x) = W^{mat}(x) + \Psi(x),
\end{equation}
where $W^{mat}(x)$ is expected material welfare and $\Psi(x) = \sum_{i \in N_H} \mathbb{E}[\psi_i]$ is expected aggregate psychological welfare.

With heterogeneity in $\omega_i \sim F$ approximated by the probability of cooperation:
\begin{equation}
    p(x) = 1 - F(\bar{\omega}(x)) = \Pr(\omega_i \geq \bar{\omega}(x)),
\end{equation}
we have (assuming large $n_H$ for the law of large numbers):
\begin{equation}
    \mathcal{W}(x) = nE + p(x) \cdot n_H(m-1)E + n_A(m-1)E + \Psi(x),
\end{equation}
where $\Psi(x) = p(x)\Psi^C(x) + (1-p(x))\Psi^D(x)$, with $\Psi^C$ and $\Psi^D$ denoting aggregate psychological welfare under cooperation and defection respectively.

\medskip
\emph{Step 4 (Existence and smoothness).}

\textbf{Existence:} By the Weierstrass theorem (Preamble), $\mathcal{W}(x)$ is continuous on compact $[0, \bar{x}]$, so a maximum exists.

\textbf{Smoothness:} Under (A1)--(A3) with the linear attribution function, $\tilde{h}_i^{(2,A)}(x)$ is continuously differentiable in $x$. With $F$ having a smooth density $f$, the probability $p(x) = 1 - F(\bar{\omega}(x))$ is continuously differentiable with:
\begin{equation}
    p'(x) = -f(\bar{\omega}(x)) \cdot \frac{\partial \bar{\omega}}{\partial x} = f(\bar{\omega}(x)) \cdot \frac{E(1 - m/n) - \beta(n_H - 1)}{\beta \lambda^{IND} n_A} \cdot \frac{\eta}{(\bar{h}(\rho_A) + \eta x)^2} > 0.
\end{equation}
The psychological welfare functions $\Psi^C(x)$ and $\Psi^D(x)$ are continuously differentiable in $x$. Therefore, $\mathcal{W}(x)$ is twice continuously differentiable.

\medskip
\emph{Step 5 (First-order condition).}
At an interior optimum $x^* \in (0, \bar{x})$, the FOC is $\mathcal{W}'(x^*) = 0$:
\begin{equation}\label{eq:foc-iii}
    \underbrace{p'(x^*) \cdot n_H(m-1)E + p'(x^*)\left[\Psi^C(x^*) - \Psi^D(x^*)\right]}_{\text{marginal benefit of higher } x} + \underbrace{p(x^*)\frac{\partial \Psi^C}{\partial x}\bigg|_{x^*} + (1-p(x^*))\frac{\partial \Psi^D}{\partial x}\bigg|_{x^*}}_{\text{marginal cost: increased guilt}} = 0.
\end{equation}

The first two terms capture the benefit of expanding cooperation (higher material welfare and the psychological gain from switching from defection to cooperation). The last two terms capture the marginal guilt cost from higher attributed expectations.

\medskip
\emph{Step 6 (Second-order condition).}
For the FOC to characterize a maximum, the SOC must hold: $\mathcal{W}''(x^*) < 0$.

Differentiating the FOC:
\begin{align}
    \mathcal{W}''(x) &= p''(x) \cdot n_H(m-1)E + p''(x)[\Psi^C - \Psi^D] + 2p'(x)\left[\frac{\partial \Psi^C}{\partial x} - \frac{\partial \Psi^D}{\partial x}\right] \nonumber \\
    &\quad + p(x)\frac{\partial^2 \Psi^C}{\partial x^2} + (1-p(x))\frac{\partial^2 \Psi^D}{\partial x^2}.
\end{align}

\textbf{Verification of SOC:}
\begin{enumerate}[label=(\roman*), nosep]
    \item $p''(x) < 0$ when $f$ is unimodal and $\bar{\omega}(x)$ is in the increasing part of the density. This captures diminishing returns: as $x$ increases, fewer marginal types are induced to cooperate.

    \item $\partial \Psi^C/\partial x \leq 0$ and $\partial \Psi^D/\partial x \leq 0$: higher $x$ raises attributed expectations, increasing guilt.

    \item $\partial^2 \Psi^C/\partial x^2 \leq 0$ and $\partial^2 \Psi^D/\partial x^2 \leq 0$ under the linear guilt structure: guilt is linear in the expectation gap, and expectations are linear in $x$.
\end{enumerate}

Under these conditions, $\mathcal{W}''(x^*) < 0$ at any interior critical point, confirming it is a local maximum. Since $\mathcal{W}$ is continuous on compact $[0, \bar{x}]$, the global maximum is either at the interior critical point or at a boundary.

\textbf{Sufficient condition for interior solution:} When $\mathcal{W}'(0) > 0$ (marginal benefit exceeds marginal cost at $x = 0$) and $\mathcal{W}'(\bar{x}) < 0$ (marginal cost exceeds benefit at $x = \bar{x}$), an interior solution exists.

\medskip
\emph{Step 7 (Comparative statics via implicit function theorem).}

Let $\mathcal{F}(x; m, G, \omega) \equiv \mathcal{W}'(x)$. At an interior optimum, $\mathcal{F}(x^*; m, G, \omega) = 0$.

\textbf{IFT conditions:}
\begin{enumerate}[label=(\alph*), nosep]
    \item $\mathcal{F}$ is continuously differentiable in $(x, m, G, \omega)$---established in Step 4.
    \item $\partial \mathcal{F}/\partial x = \mathcal{W}''(x^*) \neq 0$---by the SOC, $\mathcal{W}''(x^*) < 0$.
\end{enumerate}

By the implicit function theorem:
\begin{equation}
    \frac{\partial x^*}{\partial \theta} = -\frac{\partial \mathcal{F}/\partial \theta}{\partial \mathcal{F}/\partial x} = -\frac{\partial^2 \mathcal{W}/\partial x \partial \theta}{\mathcal{W}''(x^*)}.
\end{equation}
Since $\mathcal{W}''(x^*) < 0$, the sign of $\partial x^*/\partial \theta$ equals the sign of $\partial^2 \mathcal{W}/\partial x \partial \theta$.

\textbf{(i) $\partial x^*/\partial m > 0$ (increasing in efficiency):}

The cross-partial is:
\begin{equation}
    \frac{\partial^2 \mathcal{W}}{\partial x \partial m} = \frac{\partial}{\partial m}\left[ p'(x) \cdot n_H(m-1)E + \cdots \right] = p'(x) \cdot n_H E > 0.
\end{equation}
The psychological terms $\Psi^C$, $\Psi^D$ do not depend directly on $m$ (they depend on equilibrium contributions and attributed expectations, neither of which depends on $m$ once $x$ is fixed). Therefore:
\begin{equation}
    \frac{\partial x^*}{\partial m} = -\frac{p'(x^*) \cdot n_H E}{\mathcal{W}''(x^*)} > 0.
\end{equation}

\textbf{Interpretation and reconciliation with Part (i):}
In Part (iii), higher $m$ increases the \emph{marginal benefit} of expanding cooperation. The planner responds by increasing $x^*$ to capture greater efficiency gains.

This appears opposite to Part (i), where $\partial x^*/\partial m < 0$. The resolution: Part (i) solves for the \emph{minimal threshold-crossing signal}, which decreases with $m$. Part (iii) balances material and psychological welfare at the margin, where higher $m$ justifies higher signal despite guilt costs.

As $\rho_A \to 1$, guilt costs vanish (cooperation meets expectations), and Part (iii) degenerates to Part (i): the planner uses minimal signal sufficient for cooperation. The comparative static $\partial x^*/\partial m > 0$ in Part (iii) reflects the interior tradeoff; in the limit, welfare becomes flat in $x$ once threshold is crossed, recovering Part (i).

\textbf{(ii) $\partial x^*/\partial G < 0$ (decreasing in guilt sensitivity):}

Under defection ($c_i = 0$), guilt is:
\begin{equation}
    \psi_i^{GUILT,D}(x) = -G \cdot \omega_i \cdot [\bar{h}(\rho_A) + \eta x].
\end{equation}
The marginal effect of $x$ on guilt is:
\begin{equation}
    \frac{\partial \Psi^D}{\partial x} = -n_H G \omega \eta.
\end{equation}
The cross-partial with respect to $G$:
\begin{equation}
    \frac{\partial^2 \mathcal{W}}{\partial x \partial G} = (1-p(x)) \cdot \frac{\partial}{\partial G}\left(\frac{\partial \Psi^D}{\partial x}\right) + p(x) \cdot \frac{\partial}{\partial G}\left(\frac{\partial \Psi^C}{\partial x}\right).
\end{equation}

For $\Psi^D$: $\frac{\partial}{\partial G}\left(\frac{\partial \Psi^D}{\partial x}\right) = -n_H \omega \eta < 0$.

For $\Psi^C$: If $\tilde{h}_i^{(2,A)} > E$, guilt arises even under cooperation. The term is similarly negative.

Therefore $\frac{\partial^2 \mathcal{W}}{\partial x \partial G} < 0$, and:
\begin{equation}
    \frac{\partial x^*}{\partial G} = -\frac{\partial^2 \mathcal{W}/\partial x \partial G}{\mathcal{W}''(x^*)} < 0.
\end{equation}
Higher guilt sensitivity raises marginal cost of signal, reducing optimal $x^*$.

\textbf{(iii) $\partial x^*/\partial \omega < 0$ (decreasing in anthropomorphism tendency):}

Higher $\omega$ amplifies guilt costs from each unit of $x$:
\begin{equation}
    \frac{\partial}{\partial \omega}\left(\frac{\partial \Psi^D}{\partial x}\right) = -n_H G \eta < 0.
\end{equation}

The cross-partial:
\begin{equation}
    \frac{\partial^2 \mathcal{W}}{\partial x \partial \omega} < 0,
\end{equation}
hence:
\begin{equation}
    \frac{\partial x^*}{\partial \omega} = -\frac{\partial^2 \mathcal{W}/\partial x \partial \omega}{\mathcal{W}''(x^*)} < 0.
\end{equation}
When users are more prone to anthropomorphize, the planner reduces signal to limit guilt costs.

\medskip
\emph{Step 8 (Summary).}
At an interior solution, $x^* \in (0, \bar{x})$ satisfies the first-order condition \eqref{eq:foc-iii} and SOC $\mathcal{W}''(x^*) < 0$. Comparative statics:
\begin{equation}
    \frac{\partial x^*}{\partial m} > 0, \quad \frac{\partial x^*}{\partial G} < 0, \quad \frac{\partial x^*}{\partial \omega} < 0.
\end{equation}
Optimal signal increases with efficiency gains and decreases with guilt sensitivity and anthropomorphism tendency.

This completes Part (iii). \hfill $\square$
\end{proof}

%------------------------------------------------------------------------------
% REMARKS
%------------------------------------------------------------------------------

\begin{remark}[Unified interpretation]
\label{rem:unified}
The three parts formalize the ``match presentation to objectives'' principle:
\begin{enumerate}[nosep]
    \item \textbf{Prosocial AI}: Anthropomorphic presentation facilitates cooperation. The planner uses minimal signal sufficient for threshold crossing. Higher efficiency reduces the required signal.
    \item \textbf{Materialist AI}: Anthropomorphic presentation creates phantom expectations---attributed expectations without prosocial basis. Minimal presentation ($x^* = 0$) eliminates pure welfare loss.
    \item \textbf{Mixed AI}: Interior solution reflects genuine tradeoff. The planner calibrates $x^*$ to balance cooperation benefits against psychological costs. Higher efficiency justifies more signal.
\end{enumerate}
\end{remark}

\begin{remark}[Welfare criterion consistency]
\label{rem:welfare-criterion}
Part (i) uses material welfare; Parts (ii) and (iii) use extended welfare. This apparent inconsistency is resolved as follows:
\begin{itemize}[nosep]
    \item In Part (i), prosocial AI generates expectations consistent with equilibrium behavior. Under cooperation, expectations are met, so $\psi_i \geq 0$. Extended welfare equals or exceeds material welfare. Maximizing material welfare subject to threshold crossing yields a valid optimum.
    \item In Parts (ii) and (iii), anthropomorphism can generate phantom expectations that create guilt ($\psi_i < 0$). Extended welfare may fall below material welfare. Using material welfare alone would ignore this cost.
\end{itemize}
The welfare criterion is thus chosen to match the relevant economic forces in each case. A unified formulation using extended welfare throughout would yield the same results: Part (i) would have $\psi = 0$ at optimum, recovering the material welfare solution.
\end{remark}

\begin{remark}[Reconciliation of comparative statics for $m$]
\label{rem:reconcile-m}
Part (i) shows $\partial x^*/\partial m < 0$ while Part (iii) shows $\partial x^*/\partial m > 0$. These are not contradictory:
\begin{itemize}[nosep]
    \item Part (i): The planner finds the \emph{minimal} signal to cross the cooperation threshold. Higher $m$ lowers the threshold, reducing the required signal. The objective is to reach cooperation at minimal cost.
    \item Part (iii): The planner balances material and psychological welfare at the margin. Higher $m$ increases the \emph{marginal value} of expanding cooperation. At an interior optimum, this justifies a higher signal.
\end{itemize}
The difference is the optimization objective: minimal threshold crossing (Part i) versus interior balancing (Part iii). As $\rho_A \to 1$ and guilt costs vanish, Part (iii) degenerates to Part (i).
\end{remark}

\begin{remark}[Nesting of cases]
\label{rem:nesting}
Part (iii) nests Parts (i) and (ii) as limits:
\begin{itemize}[nosep]
    \item As $\rho_A \to 1$: $\bar{h}(\rho_A) \to \bar{h}_H$, and in full cooperation equilibrium, expectations are met, so $\Psi^C \to 0$. The marginal cost of $x$ vanishes, and the solution approaches Part (i).
    \item As $\rho_A \to 0$: $\bar{h}(\rho_A) \to 0$, expectations become purely signal-driven. The cooperation benefit shrinks (threshold rises), while guilt costs remain. The solution approaches Part (ii): $x^* \to 0$.
\end{itemize}
\end{remark}

\begin{remark}[Role of each assumption]
\label{rem:assumptions-role}
\begin{itemize}[nosep]
    \item \textbf{(A1)--(A3)}: Ensure continuity of equilibrium correspondence, enabling Weierstrass existence and IFT smoothness.
    \item \textbf{(A2') and (A2''')}: Ensure monotonic mapping from $(\omega, x)$ to attributed expectations, making the threshold well-defined.
    \item \textbf{(E)}: Guarantees $W^C > W^D$, so the planner strictly prefers cooperation.
    \item \textbf{(T)}: Ensures cooperation threshold is positive (AI channel matters).
    \item \textbf{(I)}: Ensures cooperation threshold is attainable within $\omega \in [0,1]$.
    \item \textbf{(G')}: Creates psychological costs from unmet expectations, generating the tradeoff in Parts (ii) and (iii).
\end{itemize}
\end{remark}

\begin{remark}[Decision-maker and designer incentives]
\label{rem:decision-maker}
This proposition characterizes the social planner's optimum. In practice, AI is designed by private entities with potentially misaligned incentives. The key divergence:
\begin{itemize}[nosep]
    \item \textbf{Private designers}: May prefer high $x$ because anthropomorphism increases engagement and revenue. Psychological costs (guilt) are externalized to users.
    \item \textbf{Social planner}: Internalizes psychological costs, choosing lower $x$ for materialist AI.
\end{itemize}
This creates a case for regulation when $\rho_A < 1$. The proposition provides the first-best benchmark; analysis of second-best instruments (disclosure requirements, anthropomorphism limits) is deferred to Section~\ref{sec:welfare-extensions}.
\end{remark}

\begin{remark}[Boundary cases]
\label{rem:boundary}
\begin{itemize}[nosep]
    \item \textbf{$\omega = 0$}: The human treats AI as a pure machine with no attributed expectations: $\tilde{h}_i^{(2,A)} = 0$ for all $x$. Human cooperation depends only on peer interactions. The anthropomorphic signal is irrelevant; any $x \in [0, \bar{x}]$ is optimal.
    \item \textbf{$\omega = 1$}: The human fully anthropomorphizes AI, treating it as a human agent. This corresponds to the standard psychological game. The linear attribution formula applies with $\omega = 1$.
    \item \textbf{$x = 0$}: Minimal anthropomorphism. All attributed expectations come from $\bar{h}(\rho_A)$. For materialist AI ($\rho_A = 0$), $\tilde{h} = 0$.
    \item \textbf{$x = \bar{x}$}: Maximal anthropomorphism. Attributed expectations are $\omega[\bar{h}(\rho_A) + \eta \bar{x}]$.
\end{itemize}
\end{remark}

\begin{remark}[Policy implications]
\label{rem:policy}
The results provide actionable design guidance:
\begin{enumerate}[nosep]
    \item \textbf{Prosocial AI} (AI assistants with welfare objectives): Anthropomorphic presentation facilitates cooperation. Design should emphasize human-like qualities, calibrated to efficiency stakes.
    \item \textbf{Materialist AI} (recommendation systems, trading algorithms): Present as machines. Anthropomorphic interfaces create phantom expectations and psychological harm.
    \item \textbf{Context-dependent design}: Match signal intensity to efficiency stakes (high $x$ for high-stakes cooperation), user characteristics (low $x$ for guilt-prone or high-$\omega$ populations), and AI objectives.
\end{enumerate}
\end{remark}

\begin{remark}[Connection to Propositions~\ref{prop:welfare-anthro} and \ref{prop:over-anthro}]
\label{rem:connections}
This proposition complements earlier welfare results:
\begin{itemize}[nosep]
    \item Proposition~\ref{prop:welfare-anthro} establishes that higher $\omega$ weakly increases welfare with prosocial AI but may reduce welfare with materialist AI.
    \item Proposition~\ref{prop:over-anthro} shows elevated anthropomorphism reduces extended welfare.
    \item The current proposition characterizes the \emph{optimal design response}: given human anthropomorphism tendency $\omega$, how should the planner choose $x$?
\end{itemize}
Together, these results provide both descriptive (how $\omega$ affects outcomes) and normative (how to design $x$) guidance for human-AI interaction.
\end{remark}


\bibliography{../references,../ExportedItems}

\end{document}
